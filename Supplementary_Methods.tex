% Standalone Supplementary Methods for ALIN Framework
\documentclass[11pt]{article}
\usepackage[margin=1in]{geometry}
\usepackage{graphicx}
\usepackage{amsmath}
\usepackage{amssymb}
\usepackage{longtable}
\usepackage{booktabs}
\usepackage{hyperref}
\begin{document}
\sloppy
\section*{Supplementary Methods}
\subsection*{S1. Data provenance and availability}
All primary data were accessed in January 2026. Exact file URLs, local cache filenames, and SHA-256 checksums are provided in \verb|DATA_AVAILABILITY.md| in the project repository. DepMap 24Q4 (CRISPRGeneEffect.csv, Model.csv), OmniPath directed interactions, PRISM repurposing screen, and GDSC drug sensitivity datasets were used. The pipeline's data ingestion script (\verb|scripts/data_fetch.py|) downloads and verifies checksums; use \verb|./run_full_pipeline.sh --fetch| to reproduce.

\subsection*{S2. Algorithms and mathematical definitions}
This document reproduces the equations summarized in the main text and provides additional derivations and implementation notes. Definitions for essentiality thresholds, co-essentiality Jaccard index, signaling path scoring, MHS cost function, ranked triple scoring components, and ODE model equations are provided (see S2--S8). The full ODE simulation detail, including network construction, equations, cross-cancer comparison, pathway shifting dynamics, and epistemic caveats, is presented in S5. Circularity and leakage ablation tests are presented in S8. For reproducibility, we include pseudocode and example command-line invocations for each major module.

\subsection*{S3. Implementation notes}
Key scripts and functions:
- \verb|pan_cancer_xnode.py|: main pipeline entrypoint; accepts flags \verb|--cancer TYPE|, \verb|--all|, \verb|--fetch|, \verb|--no-validate|.
- \texttt{alin/perturbation.py}: parses curated perturbation studies and computes feedback gene lists.
- \texttt{alin/mhs\_solver.py}: greedy and exhaustive MHS solvers; exhaustive search uses branch-and-bound pruning and optional time limit flags.
- \verb|scripts/ode_simulation.py|: constructs cancer-specific ODE networks and runs RK45 integration (\verb|scipy.integrate.solve_ivp|); outputs time-series CSVs and summary viability indices.

\subsection*{S4. Parameter tables and sensitivity analyses}
The full parameter table used in ODE simulations is reproduced, and extended sensitivity analyses (univariate and two-way) are provided as an attached CSV (\verb|supplementary_parameter_sensitivity.csv|). Summary results demonstrate that the tri-axial vs MHS viability advantage is robust across biologically plausible parameter ranges.

\subsection*{S5. ODE-based pathway shifting simulation}

This section presents the full details of the ODE-based systems biology simulation summarized in the main text. The simulation illustrates the deductive consequences of the tri-axial hypothesis under assumed compensatory signaling parameters. \textbf{Important caveat:} the model's parameters are assigned by biological role, not fit to experimental data; the simulation therefore explores the \emph{deductive implications} of the tri-axial hypothesis rather than providing independent evidence for it.

\subsubsection*{S5.1 Network construction}
For five cancer types (PDAC, Melanoma, NSCLC, CRC, Breast), we constructed cancer-specific signaling networks comprising 10--14 nodes and 12--19 interactions, annotated with three biological axes: \emph{upstream} drivers (constitutively active oncogenes: KRAS, BRAF, EGFR), \emph{downstream} effectors (cascade nodes: RAF1, MEK, ERK, CCND1, CDK4), and \emph{orthogonal} survival nodes (STAT3, FYN, MCL1, JAK family). Parameters were assigned by biological role rather than fit to experimental data: constitutively active oncogenes received high basal production rates ($b = 0.25$--$0.50$), reflecting their mutational activation; cascade nodes received low basal rates ($b = 0.03$--$0.08$), making them depend on upstream input; orthogonal nodes received moderate basal rates ($b = 0.05$--$0.10$) with elevated compensatory gains ($g = 0.4$--$0.7$). Because the parameter structure assigns higher compensatory capacity to orthogonal nodes by construction, the resulting tri-axial advantage is a deductive consequence of these assumptions; independent parameterization from time-course phosphoproteomics would be required to treat the simulation as predictive.

\subsubsection*{S5.2 ODE model}
Each node $i$ follows:
\begin{equation}
\frac{dA_i}{dt} = \bigl[\underbrace{b_i + \textstyle\sum_j w_{ji} \cdot H(A_j)}_{\text{production}}\bigr] \cdot \underbrace{(1 - \delta_i(t))}_{\text{drug}} + \underbrace{C_i(\Delta_{\text{axes}})}_{\text{compensation}} - \underbrace{d_i \cdot A_i}_{\text{degradation}} - \underbrace{I_i}_{\text{inhibition}}
\end{equation}
where $H(x) = x^2/(K^2 + x^2)$ is a Hill function (half-max $K = 0.5$, coefficient $n = 2$), $\delta_i(t) = s_i(1 - e^{-\alpha t})$ is the time-dependent drug inhibition with strength $s_i = 0.92$ and onset rate $\alpha = 0.15\,\text{h}^{-1}$, and $I_i$ captures network-level negative feedback (e.g., ERK$\dashv$EGFR). The drug effect is \emph{multiplicative} on production, modeling kinase inhibitors and PROTACs that reduce effective signaling output.

\subsubsection*{S5.3 Compensatory pathway shifting}
The compensation term $C_i$ captures the Liaki de-repression mechanism: when drug treatment depletes axes \emph{other} than node $i$'s axis, uninhibited compensatory sources (e.g., FYN) activate orthogonal survival nodes (e.g., STAT3):
\begin{equation}
C_i = g_i \cdot \max\bigl(0, \bar{\Delta}_{\neg\text{axis}(i)} - 0.1\bigr) + \sum_{k \in \text{comp}(i)} w_{ki}^{(\text{comp})} \cdot H(A_k) \cdot \mathbb{1}[k \notin \mathcal{T}]
\end{equation}
where $\bar{\Delta}_{\neg\text{axis}(i)}$ is the mean fractional activity loss in axes other than node $i$'s own axis (relative to untreated homeostasis), and $\mathbb{1}[k \notin \mathcal{T}]$ ensures that compensation from a node $k$ is suppressed if $k$ itself is drug-inhibited. This captures the key biological insight: tri-axial combinations eliminate both direct activity \emph{and} compensatory sources, while intra-axial MHS combinations leave orthogonal compensators intact.

\subsubsection*{S5.4 Tumor viability}
Following the Liaki principle that a tumor survives if \emph{any} axis maintains sufficient activity:
\begin{equation}
V(t) = 0.6 \cdot \max_{\text{axis}} \bar{A}_{\text{axis}}(t) + 0.4 \cdot \text{mean}_{\text{axis}} \bar{A}_{\text{axis}}(t)
\end{equation}
where $\bar{A}_{\text{axis}}$ is the mean activity of nodes in each axis.

\subsubsection*{S5.5 Treatment strategies}
We compared four strategies per cancer: (1) \emph{no treatment}, (2) \emph{single agent} (one upstream driver), (3) \emph{MHS} (computationally-derived 2-target minimal hitting set, targeting nodes within the same cascade or axis), and (4) \emph{tri-axial combination} (one target per biological axis: upstream + downstream + orthogonal). Simulations ran for 4800\,h (200 days, matching the Liaki et al.\ observation window) using RK45 integration (\verb|scipy.integrate.solve_ivp|).

\subsubsection*{S5.6 Cross-cancer quantitative comparison (under stated assumptions)}
Under the assumed parameter structure, tri-axial combinations achieved a mean final tumor viability of $0.472 \pm 0.069$ compared to $0.691 \pm 0.069$ for MHS strategies, an $\sim$30\% viability reduction. Because this advantage is a deductive consequence of assigning higher compensatory gains to orthogonal nodes, the specific magnitude should not be interpreted as a quantitative prediction; rather, it illustrates the \emph{direction and scale} of the tri-axial advantage \emph{if} compensatory signaling operates as assumed. The pattern was consistent across all five modeled cancer types:

\begin{itemize}
\item \textbf{PDAC}: MHS (KRAS+EGFR, both upstream) = 0.806 vs.\ tri-axial (KRAS+CDK4+STAT3) = 0.516 (36\% advantage). KRAS+EGFR inhibition eliminates the upstream axis but leaves orthogonal STAT3 and downstream CDK4 intact; the tri-axial combination blocks all three escape routes.
\item \textbf{Melanoma}: MHS (BRAF+MEK, upstream+downstream cascade) = 0.605 vs.\ tri-axial (BRAF+CCND1+STAT3) = 0.454 (25\%). BRAF+MEK is the clinical standard (dabrafenib+trametinib), yet invariably develops resistance through NRAS/STAT3 reactivation; the tri-axial combination pre-empts this.
\item \textbf{NSCLC}: MHS (EGFR+KRAS, 2 upstream) = 0.671 vs.\ tri-axial (KRAS+CCND1+MCL1) = 0.568 (15\%). The smallest advantage, reflecting NSCLC's heterogeneous driver landscape where MET amplification provides additional bypass.
\item \textbf{CRC}: MHS (KRAS+BRAF, 2 upstream MAPK drivers) = 0.723 vs.\ tri-axial (KRAS+CCND1+STAT3) = 0.458 (37\%). CRC's WNT/$\beta$-catenin and JAK/STAT convergence require explicit orthogonal axis coverage.
\item \textbf{Breast}: MHS (CDK4+CDK6, 2 downstream) = 0.652 vs.\ tri-axial (CDK4+KRAS+STAT3) = 0.363 (44\%). The largest advantage, consistent with clinical observations that CDK4/6 inhibitor resistance develops through PI3K/STAT3 bypass.
\end{itemize}

\subsubsection*{S5.7 Pathway shifting dynamics}
The simulation illustrates the \emph{hypothesized} mechanism underlying intra-axial MHS failure: when same-axis targets are inhibited, orthogonal axis nodes (STAT3, FYN, MCL1) increase their activity through the compensatory signaling encoded in the model, recapitulating the de-repression mechanism described by Liaki et al. In PDAC, KRAS+EGFR inhibition triggers a pathway shift magnitude of 0.500 (50\% increase in orthogonal axis activity above untreated baseline), while the tri-axial combination constrains the shift to 0.400. Across cancers, MHS combinations showed a mean pathway shift of 0.490 compared to 0.434 for tri-axial combinations, indicating that same-axis targeting provokes stronger compensatory activation.

\subsubsection*{S5.8 Interpretation and epistemic caveats}
These simulation outputs are \emph{deductive consequences} of the model's parameter structure, not independent empirical observations. The model encodes the tri-axial hypothesis through its axioms (elevated compensatory gains for orthogonal nodes, $g = 0.4$--$0.7$ vs.\ $g = 0.1$--$0.2$ for cascade nodes), and the tri-axial advantage follows logically from those axioms. The simulation therefore serves as a \emph{hypothesis illustration}: it demonstrates what the Liaki tri-axial principle quantitatively implies if compensatory signaling operates as assumed, and shows that MHS combinations that target topologically proximal same-axis nodes would leave orthogonal compensators intact under those assumptions. The simulation does not constitute evidence for the tri-axial hypothesis itself; independent validation would require fitting compensatory parameters from experimental time-course data.

The sensitivity analysis confirms that the qualitative ordering (tri-axial $<$ MHS $<$ single $<$ untreated) is preserved across $\pm$25\% parameter perturbations, including when compensatory gains are reduced to $g = 0.3$. This robustness analysis demonstrates that the tautology is stable under perturbation---not that the assumptions themselves are valid. None of the parameters ($b_i$, $g_i$, $d_i$, $s_i$) were fit to experimental data such as time-course phosphoproteomics, dose--response curves, or drug pharmacokinetics. Elevating this simulation from illustrative to predictive would require: (i) fitting compensatory gain parameters from time-course phosphoproteomics (e.g., STAT3 Tyr705 phosphorylation kinetics after MEK/EGFR inhibition); (ii) calibrating drug strength from dose--response data; (iii) validating the ODE dynamics against an independent cancer not used in parameter assignment.

\subsection*{S6. Dual-vs.-triple escape-route validation}

This section documents the computational methods underlying
Supplementary Figure~S12, which compares the functional coverage of
all $\binom{3}{2}$ dual combinations against the full ALIN-predicted
triple for three representative cancers: NSCLC, Melanoma, and CRC.

\subsubsection*{S6.1 PRISM Bliss independence model}

Single-agent viabilities are obtained from the PRISM primary screen
(Corsello et al., 2020). Log-fold-change values $\ell_i$ are converted to
fractional viability $V_i = 2^{\ell_i}$, where $V = 1$ indicates no effect
and $V < 1$ indicates growth inhibition. For combinations, we apply the Bliss
independence assumption:
\[
V_{\mathrm{combo}} = \prod_{i=1}^{k} V_i
\]
where $k$ is the number of drugs (2 for duals, 3 for triples). This assumes
that each drug acts independently; violations (synergy or antagonism) are not
captured but represent a conservative null model.

For CRC, sotorasib (KRAS inhibitor) is not present in the PRISM primary
screen (approved 2021, post-PRISM). KRAS inhibition is instead proxied via
CRISPR Chronos scores: $V_{\mathrm{KRAS}} = 1/(1 + e^{-2 \cdot C_{\mathrm{KRAS}}})$,
where $C$ is the Chronos gene effect score (more negative = more essential).
This sigmoid maps strong essentiality ($C < -1$) to low viability
($V \approx 0.12$) and non-essentiality ($C > 0$) to no effect ($V > 0.5$).

\subsubsection*{S6.2 Network escape-route enumeration}

We construct a canonical signaling network (97 directed edges, 43 nodes)
from literature-validated interactions, covering RTK--RAS--MAPK, PI3K--AKT--mTOR,
JAK--STAT, cell cycle, Wnt, Hippo, and cross-pathway compensation routes
(SRC, AXL, MET, FGFR1, IGF1R bypass edges).

For each cancer type, DepMap CRISPR essentiality data identifies signaling
genes that are essential in $\geq$15\% of cell lines (Chronos $< -0.4$).
``Escape routes'' are defined as directed paths (BFS, max depth 5) from any
uninhibited essential source gene to downstream effector nodes (BCL2L1, MYC,
CCND1, CCNE1, MAPK1, MAPK3, MCL1, BIRC5, E2F1, etc.) that do not traverse
any inhibited target. Fewer escape routes under a given inhibition set indicate
more comprehensive blockade of compensatory signaling.

\subsubsection*{S6.3 Stochastic time-to-escape simulation}

Given $n$ escape routes remaining under a given inhibition set, we model
time to tumor escape as a geometric random variable. At each discrete time step,
each escape route independently reactivates with probability $p = 0.005$.
The probability of escape at step $t$ is:
\[
P(\text{escape at step } t) = 1 - (1 - p)^n
\]
We perform 10{,}000 Monte Carlo iterations per condition and compare
dual vs.\ triple escape-time distributions using the one-sided
Mann--Whitney $U$ test.

The reactivation probability $p = 0.005$ is not calibrated to
experimental data; it serves as a consistent comparative parameter.
The qualitative finding (fewer routes $\Rightarrow$ longer escape time)
is invariant to the choice of $p$ within the range $10^{-4}$--$10^{-2}$.

\subsection*{S7. Outcome-oriented benchmarking}

Beyond target concordance, we evaluate whether ALIN scoring metrics can distinguish clinically \emph{successful} from \emph{failed} molecularly-targeted combination therapies.

\subsubsection*{S7.1 Curation criteria}

We curated 30 molecularly-targeted oncology combination regimens (15 successes, 15 failures), each satisfying:
(i)~at least 2 distinct molecular targets (no chemo-only);
(ii)~published Phase~2/3 efficacy data or FDA regulatory decision;
(iii)~outcomes determined independently from ALIN predictions.
Each entry has a PubMed ID (PMID) or ClinicalTrials.gov identifier (NCT number).

\textbf{Successful combinations (15):} FDA-approved or Phase~3 positive regimens, including:
dabrafenib + trametinib in melanoma (COMBI-d, PMID: 25399551),
vemurafenib + cobimetinib (coBRIM, PMID: 25105994),
encorafenib + cetuximab in CRC (BEACON, PMID: 31566309),
amivantamab in NSCLC (CHRYSALIS, PMID: 34043995),
palbociclib + fulvestrant in breast (PALOMA-3, PMID: 26394241),
venetoclax + gilteritinib in AML (PMID: 35443125),
and the encorafenib + binimetinib + cetuximab triplet in CRC (BEACON triplet arm).

\textbf{Failed combinations (15):} Phase~2/3 trials terminated for futility, toxicity, or no efficacy benefit, including:
erlotinib + figitumumab in NSCLC (NCT00673049, PMID: 21990073),
cetuximab + bevacizumab in CRC (CAIRO2, PMID: 19826119),
trastuzumab + everolimus in breast (BOLERO-1, PMID: 26369892),
vemurafenib + MK-2206 in melanoma (NCT01512251, PMID: 27612945),
selumetinib + MK-2206 in NSCLC (NCT01021748, PMID: 27480128),
and dabrafenib + buparlisib in melanoma (NCT01820364, PMID: 26997144).

\subsubsection*{S7.2 Scoring metrics}

For each curated combination (target set $T$ + cancer type $c$), we compute:
\begin{itemize}
\item \textbf{Co-essentiality}: Mean pairwise Pearson correlation of DepMap CRISPR Chronos scores across cancer-matched cell lines. Higher correlation indicates co-dependent targets.
\item \textbf{Mean essentiality}: Average Chronos effect size (more negative = more essential).
\item \textbf{Fraction essential}: Fraction of (target $\times$ cell line) pairs with Chronos $< -0.5$.
\item \textbf{Escape routes}: Total directed paths (BFS, depth $\leq 5$) from uninhibited essential signaling genes to downstream effectors (same canonical network as S6.2) that bypass all targets in $T$.
\item \textbf{Escape-route ratio}: $n_{\text{routes with } T} / n_{\text{routes without } T}$.
\item \textbf{Pathway coherence}: Fraction of target pairs mapping to the same canonical signaling pathway. Values of 1.0 indicate full within-pathway targeting (e.g., BRAF+MEK); 0.0 indicates all targets in different pathways.
\item \textbf{Composite score}: Weighted combination: $0.35 \times$ co-essentiality $+ 0.15 \times$ fraction-essential $+ 0.20 \times (1 - \text{escape ratio}) + 0.30 \times$ pathway-coherence.
\end{itemize}

\subsubsection*{S7.3 Success vs.\ failure classification}

Discrimination is evaluated by Mann--Whitney $U$ test (two-sided) and AUC (probability that a randomly selected success scores higher than a random failure).
Bootstrap 95\% confidence intervals for AUC are computed from 2{,}000 bootstrap resamples.

Results (Supplementary Figure~S13, panels A--C):
\begin{itemize}
\item Co-essentiality: AUC = 0.69 [0.47--0.88], $p = 0.121$. Successful combos have higher median pairwise $r$ (0.029 vs.\ $-0.053$).
\item Pathway coherence: AUC = 0.66 [0.50--0.82], $p = 0.080$. Successful combos more often employ within-pathway (``vertical'') blockade.
\item Composite: AUC = 0.68 [0.48--0.88], $p = 0.089$.
\item Escape-route ratio: AUC = 0.68 [0.48--0.87], $p = 0.089$. Notably, the \emph{raw escape count} is higher for successes (median 59 vs.\ 47), an apparent paradox explained by successful combos targeting central pathway hubs (e.g., BRAF+MEK) that reside in highly connected network neighborhoods; the normalised escape-route \emph{ratio} corrects this topology-dependent bias.
\end{itemize}

\textbf{Limitations.} With $n = 30$ combinations, statistical power is limited. The 95\% CIs for all AUCs include 0.5, and $p$-values do not reach the conventional 0.05 threshold. A larger curated set ($n \geq 60$) with pre-registered analysis would be needed for definitive conclusions. Scoring weights in the composite are not cross-validated; they reflect domain knowledge about which metrics should favour synergistic combinations.

\subsubsection*{S7.4 Third-target-over-doublet prediction}

For 5 FDA-approved doublets, we independently curated 9 candidate third targets with Phase~1/2 or preclinical evidence from studies not used to train ALIN:
\begin{itemize}
\item Melanoma (BRAF+MEK): CDK4 (ribociclib Phase 1/2, PMID: 32511247), EGFR (feedback reactivation, PMID: 22945257).
\item CRC (BRAF+EGFR): MAP2K1 (BEACON triplet, PMID: 31566309), PIK3CA (PI3K reactivation, PMID: 22588877).
\item NSCLC (EGFR+MET): MAP2K1 (MEK reactivation, PMID: 29751011), CDK4 (sensitisation, PMID: 30755733).
\item Breast (HER2+CDK4/6): PIK3CA (SOLAR-1, PMID: 31091374), MTOR (CDK4/6i resistance, PMID: 27020862).
\item PDAC (KRAS+EGFR): STAT3 (tri-axial blockade, Liaki et al.).
\end{itemize}

ALIN's top-ranked triple matched 5/9 (56\%) of these curated third targets and recovered at least one doublet gene in 7/9 (78\%) of cases (Supplementary Figure~S13, panel D). Misses (CRC: PIK3CA; Breast: PIK3CA, MTOR) reflect ALIN's CRISPR-derived emphasis on RAS--MAPK dependencies over PI3K--AKT signaling.

\subsection*{S8. Circularity and leakage ablation}

Three ablation experiments test whether literature-curated scoring features inflate benchmark concordance or whether performance derives entirely from DepMap CRISPR data and OmniPath network topology (Supplementary Figure~S14).

\subsubsection*{S8.1 No-literature-features ablation}

The ALIN scoring formula (Equation~2, main text) includes five literature-curated components:
\begin{enumerate}
\item \textbf{KNOWN\_SYNERGIES}: 25 curated synergy pairs with clinical-evidence scores (0.60--0.95), affecting the synergy sub-score (weight 0.18).
\item \textbf{RESISTANCE\_MECHANISMS}: Literature-curated dict of bypass genes per target, affecting the resistance sub-score (weight 0.18).
\item \textbf{Perturbation bonus}: Curated from $\sim$15 published inhibitor studies (feedback coverage $\times$ 0.1).
\item \textbf{Combination toxicity}: KNOWN\_DDI (50\%) and DRUG\_TOXICITY\_PROFILE (35\%) from drug labels/DrugBank (weight 0.18).
\item \textbf{Evidence exemptions}: Hard-coded hub-penalty override for STAT3 in PDAC when paired with KRAS/EGFR.
\end{enumerate}

We disabled all five components simultaneously: synergy was computed from pathway diversity only (no known-pair lookup); resistance used a uniform $1/(1 + n \times 0.3)$ heuristic independent of curated bypass genes; combination toxicity and perturbation bonus were set to 0; evidence exemptions were cleared. The retained components---DepMap co-essentiality, tumor specificity, pan-essential penalty, OmniPath path coverage, hub penalty, and druggability clinical stage---are data-derived or factual rather than literature-curated.

\textbf{Result.} Benchmark concordance was unchanged: 61.5\% any-overlap, 38.5\% pair-overlap, 0\% exact match under both full-pipeline and no-literature conditions ($\Delta = 0.0$ pp for all metrics). The identical predictions confirm that the literature-curated features contribute zero marginal information to the target combinations evaluated by the gold standard.

\subsubsection*{S8.2 Degree-matched null}

For each of 17 cancer types, we generated 1{,}000 random triples by sampling 3 genes matched on OmniPath degree quartile (Q1--Q4 based on 25th/50th/75th percentiles of total degree across 71 network nodes) and druggability (binary: druggability score $\geq 0.4$), yielding $4 \times 2 = 8$ bins. Each gene was drawn uniformly from its corresponding bin to preserve the degree and druggability profile of ALIN's actual top-1 prediction.

\textbf{Result.} Null distributions: any-overlap $29.1\% \pm 14.3\%$, pair-overlap $9.1\% \pm 9.6\%$, exact $0.0\% \pm 0.7\%$. Empirical $p$-values (fraction of null iterations $\geq$ observed): $p = 0.030$ (any-overlap), $p = 0.013$ (pair-overlap). This rules out the hypothesis that selecting high-degree druggable genes from the OmniPath network is sufficient to match the gold standard.

\subsubsection*{S8.3 Network-scramble null}

Under 20 degree-preserving edge-swap permutations ($10 \times |E| = 940$ swap attempts per permutation, with 81\% success rate preserving all node in- and out-degrees), we re-enumerated X-nodes by scrambled degree, identified cancer-specific essential genes from DepMap CRISPR data, and scored all candidate triples using the simplified formula (path coverage, pathway diversity, hub penalty, druggability) with BFS-based reachability (depth $\leq 3$) on the scrambled adjacency.

\textbf{Result.} None of the 20 permuted networks produced viable cancer-specific triples (0 cancers per permutation), yielding 0\% concordance across all metrics ($p = 0.048$). The failure to produce any predictions demonstrates that the scrambled networks disrupt the causal chain from network topology through X-node identification to survival-mechanism inference. This confirms that ALIN's performance depends critically on OmniPath's specific biological wiring, not merely its degree distribution.

\textbf{Interpretation.} Together, these three ablations establish that (i)~literature-curated features contribute no benchmark information (zero leakage), (ii)~degree and druggability matching alone is insufficient ($p = 0.030$), and (iii)~specific network wiring is essential ($p = 0.048$). Benchmark concordance is driven by the combination of DepMap essentiality data and OmniPath signaling topology.

\subsection*{S9. How to generate the Supplementary Methods PDF}
Run the following commands in the repository root:

\begin{verbatim}
pdflatex Supplementary_Methods.tex
bibtex Supplementary_Methods
pdflatex Supplementary_Methods.tex
pdflatex Supplementary_Methods.tex
\end{verbatim}

The generated \verb|Supplementary_Methods.pdf| is a reviewer-friendly standalone document containing full mathematical definitions, parameter tables, sensitivity analysis figures, and exact data provenance metadata.

\subsection*{S10. Notes for reviewers}
- While the code repository contains executable scripts and notebooks, reviewers often expect a human-readable Supplementary Methods PDF; this file is provided for that purpose.
- For full reproducibility, reviewers can re-run the pipeline in a clean environment; see \verb|README.md| and \verb|requirements.txt| for environment setup.

\end{document}
