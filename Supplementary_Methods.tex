% Standalone Supplementary Methods for ALIN Framework
\documentclass[11pt]{article}
\usepackage[margin=1in]{geometry}
\usepackage{graphicx}
\usepackage{amsmath}
\usepackage{amssymb}
\usepackage{longtable}
\usepackage{booktabs}
\usepackage{hyperref}
\begin{document}
\sloppy
\section*{Supplementary Methods}
\subsection*{S1. Data provenance and availability}
All primary data were accessed in January 2026. Exact file URLs, local cache filenames, and SHA-256 checksums are provided in \verb|DATA_AVAILABILITY.md| in the project repository. DepMap 24Q4 (CRISPRGeneEffect.csv, Model.csv), OmniPath directed interactions, PRISM repurposing screen, and GDSC drug sensitivity datasets were used. The pipeline's data ingestion script (\verb|scripts/data_fetch.py|) downloads and verifies checksums; use \verb|./run_full_pipeline.sh --fetch| to reproduce.

\subsection*{S2. Algorithms and mathematical definitions}
This document reproduces the equations summarized in the main text and provides additional derivations and implementation notes. Definitions for essentiality thresholds, co-essentiality Jaccard index, signaling path scoring, MHS cost function, ranked triple scoring components, and ODE model equations are provided (see S2--S5). The full ODE simulation detail, including network construction, equations, cross-cancer comparison, pathway shifting dynamics, and epistemic caveats, is presented in S5. For reproducibility, we include pseudocode and example command-line invocations for each major module.

\subsection*{S3. Implementation notes}
Key scripts and functions:
- \verb|pan_cancer_xnode.py|: main pipeline entrypoint; accepts flags \verb|--cancer TYPE|, \verb|--all|, \verb|--fetch|, \verb|--no-validate|.
- \texttt{alin/perturbation.py}: parses curated perturbation studies and computes feedback gene lists.
- \texttt{alin/mhs\_solver.py}: greedy and exhaustive MHS solvers; exhaustive search uses branch-and-bound pruning and optional time limit flags.
- \verb|scripts/ode_simulation.py|: constructs cancer-specific ODE networks and runs RK45 integration (\verb|scipy.integrate.solve_ivp|); outputs time-series CSVs and summary viability indices.

\subsection*{S4. Parameter tables and sensitivity analyses}
The full parameter table used in ODE simulations is reproduced, and extended sensitivity analyses (univariate and two-way) are provided as an attached CSV (\verb|supplementary_parameter_sensitivity.csv|). Summary results demonstrate that the tri-axial vs MHS viability advantage is robust across biologically plausible parameter ranges.

\subsection*{S5. ODE-based pathway shifting simulation}

This section presents the full details of the ODE-based systems biology simulation summarized in the main text. The simulation illustrates the deductive consequences of the tri-axial hypothesis under assumed compensatory signaling parameters. \textbf{Important caveat:} the model's parameters are assigned by biological role, not fit to experimental data; the simulation therefore explores the \emph{deductive implications} of the tri-axial hypothesis rather than providing independent evidence for it.

\subsubsection*{S5.1 Network construction}
For five cancer types (PDAC, Melanoma, NSCLC, CRC, Breast), we constructed cancer-specific signaling networks comprising 10--14 nodes and 12--19 interactions, annotated with three biological axes: \emph{upstream} drivers (constitutively active oncogenes: KRAS, BRAF, EGFR), \emph{downstream} effectors (cascade nodes: RAF1, MEK, ERK, CCND1, CDK4), and \emph{orthogonal} survival nodes (STAT3, FYN, MCL1, JAK family). Parameters were assigned by biological role rather than fit to experimental data: constitutively active oncogenes received high basal production rates ($b = 0.25$--$0.50$), reflecting their mutational activation; cascade nodes received low basal rates ($b = 0.03$--$0.08$), making them depend on upstream input; orthogonal nodes received moderate basal rates ($b = 0.05$--$0.10$) with elevated compensatory gains ($g = 0.4$--$0.7$). Because the parameter structure assigns higher compensatory capacity to orthogonal nodes by construction, the resulting tri-axial advantage is a deductive consequence of these assumptions; independent parameterization from time-course phosphoproteomics would be required to treat the simulation as predictive.

\subsubsection*{S5.2 ODE model}
Each node $i$ follows:
\begin{equation}
\frac{dA_i}{dt} = \bigl[\underbrace{b_i + \textstyle\sum_j w_{ji} \cdot H(A_j)}_{\text{production}}\bigr] \cdot \underbrace{(1 - \delta_i(t))}_{\text{drug}} + \underbrace{C_i(\Delta_{\text{axes}})}_{\text{compensation}} - \underbrace{d_i \cdot A_i}_{\text{degradation}} - \underbrace{I_i}_{\text{inhibition}}
\end{equation}
where $H(x) = x^2/(K^2 + x^2)$ is a Hill function (half-max $K = 0.5$, coefficient $n = 2$), $\delta_i(t) = s_i(1 - e^{-\alpha t})$ is the time-dependent drug inhibition with strength $s_i = 0.92$ and onset rate $\alpha = 0.15\,\text{h}^{-1}$, and $I_i$ captures network-level negative feedback (e.g., ERK$\dashv$EGFR). The drug effect is \emph{multiplicative} on production, modeling kinase inhibitors and PROTACs that reduce effective signaling output.

\subsubsection*{S5.3 Compensatory pathway shifting}
The compensation term $C_i$ captures the Liaki de-repression mechanism: when drug treatment depletes axes \emph{other} than node $i$'s axis, uninhibited compensatory sources (e.g., FYN) activate orthogonal survival nodes (e.g., STAT3):
\begin{equation}
C_i = g_i \cdot \max\bigl(0, \bar{\Delta}_{\neg\text{axis}(i)} - 0.1\bigr) + \sum_{k \in \text{comp}(i)} w_{ki}^{(\text{comp})} \cdot H(A_k) \cdot \mathbb{1}[k \notin \mathcal{T}]
\end{equation}
where $\bar{\Delta}_{\neg\text{axis}(i)}$ is the mean fractional activity loss in axes other than node $i$'s own axis (relative to untreated homeostasis), and $\mathbb{1}[k \notin \mathcal{T}]$ ensures that compensation from a node $k$ is suppressed if $k$ itself is drug-inhibited. This captures the key biological insight: tri-axial combinations eliminate both direct activity \emph{and} compensatory sources, while intra-axial MHS combinations leave orthogonal compensators intact.

\subsubsection*{S5.4 Tumor viability}
Following the Liaki principle that a tumor survives if \emph{any} axis maintains sufficient activity:
\begin{equation}
V(t) = 0.6 \cdot \max_{\text{axis}} \bar{A}_{\text{axis}}(t) + 0.4 \cdot \text{mean}_{\text{axis}} \bar{A}_{\text{axis}}(t)
\end{equation}
where $\bar{A}_{\text{axis}}$ is the mean activity of nodes in each axis.

\subsubsection*{S5.5 Treatment strategies}
We compared four strategies per cancer: (1) \emph{no treatment}, (2) \emph{single agent} (one upstream driver), (3) \emph{MHS} (computationally-derived 2-target minimal hitting set, targeting nodes within the same cascade or axis), and (4) \emph{tri-axial combination} (one target per biological axis: upstream + downstream + orthogonal). Simulations ran for 4800\,h (200 days, matching the Liaki et al.\ observation window) using RK45 integration (\verb|scipy.integrate.solve_ivp|).

\subsubsection*{S5.6 Cross-cancer quantitative comparison (under stated assumptions)}
Under the assumed parameter structure, tri-axial combinations achieved a mean final tumor viability of $0.472 \pm 0.069$ compared to $0.691 \pm 0.069$ for MHS strategies, an $\sim$30\% viability reduction. Because this advantage is a deductive consequence of assigning higher compensatory gains to orthogonal nodes, the specific magnitude should not be interpreted as a quantitative prediction; rather, it illustrates the \emph{direction and scale} of the tri-axial advantage \emph{if} compensatory signaling operates as assumed. The pattern was consistent across all five modeled cancer types:

\begin{itemize}
\item \textbf{PDAC}: MHS (KRAS+EGFR, both upstream) = 0.806 vs.\ tri-axial (KRAS+CDK4+STAT3) = 0.516 (36\% advantage). KRAS+EGFR inhibition eliminates the upstream axis but leaves orthogonal STAT3 and downstream CDK4 intact; the tri-axial combination blocks all three escape routes.
\item \textbf{Melanoma}: MHS (BRAF+MEK, upstream+downstream cascade) = 0.605 vs.\ tri-axial (BRAF+CCND1+STAT3) = 0.454 (25\%). BRAF+MEK is the clinical standard (dabrafenib+trametinib), yet invariably develops resistance through NRAS/STAT3 reactivation; the tri-axial combination pre-empts this.
\item \textbf{NSCLC}: MHS (EGFR+KRAS, 2 upstream) = 0.671 vs.\ tri-axial (KRAS+CCND1+MCL1) = 0.568 (15\%). The smallest advantage, reflecting NSCLC's heterogeneous driver landscape where MET amplification provides additional bypass.
\item \textbf{CRC}: MHS (KRAS+BRAF, 2 upstream MAPK drivers) = 0.723 vs.\ tri-axial (KRAS+CCND1+STAT3) = 0.458 (37\%). CRC's WNT/$\beta$-catenin and JAK/STAT convergence require explicit orthogonal axis coverage.
\item \textbf{Breast}: MHS (CDK4+CDK6, 2 downstream) = 0.652 vs.\ tri-axial (CDK4+KRAS+STAT3) = 0.363 (44\%). The largest advantage, consistent with clinical observations that CDK4/6 inhibitor resistance develops through PI3K/STAT3 bypass.
\end{itemize}

\subsubsection*{S5.7 Pathway shifting dynamics}
The simulation illustrates the \emph{hypothesized} mechanism underlying intra-axial MHS failure: when same-axis targets are inhibited, orthogonal axis nodes (STAT3, FYN, MCL1) increase their activity through the compensatory signaling encoded in the model, recapitulating the de-repression mechanism described by Liaki et al. In PDAC, KRAS+EGFR inhibition triggers a pathway shift magnitude of 0.500 (50\% increase in orthogonal axis activity above untreated baseline), while the tri-axial combination constrains the shift to 0.400. Across cancers, MHS combinations showed a mean pathway shift of 0.490 compared to 0.434 for tri-axial combinations, indicating that same-axis targeting provokes stronger compensatory activation.

\subsubsection*{S5.8 Interpretation and epistemic caveats}
These simulation outputs are \emph{deductive consequences} of the model's parameter structure, not independent empirical observations. The model encodes the tri-axial hypothesis through its axioms (elevated compensatory gains for orthogonal nodes, $g = 0.4$--$0.7$ vs.\ $g = 0.1$--$0.2$ for cascade nodes), and the tri-axial advantage follows logically from those axioms. The simulation therefore serves as a \emph{hypothesis illustration}: it demonstrates what the Liaki tri-axial principle quantitatively implies if compensatory signaling operates as assumed, and shows that MHS combinations that target topologically proximal same-axis nodes would leave orthogonal compensators intact under those assumptions. The simulation does not constitute evidence for the tri-axial hypothesis itself; independent validation would require fitting compensatory parameters from experimental time-course data.

The sensitivity analysis confirms that the qualitative ordering (tri-axial $<$ MHS $<$ single $<$ untreated) is preserved across $\pm$25\% parameter perturbations, including when compensatory gains are reduced to $g = 0.3$. This robustness analysis demonstrates that the tautology is stable under perturbation---not that the assumptions themselves are valid. None of the parameters ($b_i$, $g_i$, $d_i$, $s_i$) were fit to experimental data such as time-course phosphoproteomics, dose--response curves, or drug pharmacokinetics. Elevating this simulation from illustrative to predictive would require: (i) fitting compensatory gain parameters from time-course phosphoproteomics (e.g., STAT3 Tyr705 phosphorylation kinetics after MEK/EGFR inhibition); (ii) calibrating drug strength from dose--response data; (iii) validating the ODE dynamics against an independent cancer not used in parameter assignment.

\subsection*{S6. How to generate the Supplementary Methods PDF}
Run the following commands in the repository root:

\begin{verbatim}
pdflatex Supplementary_Methods.tex
bibtex Supplementary_Methods
pdflatex Supplementary_Methods.tex
pdflatex Supplementary_Methods.tex
\end{verbatim}

The generated \verb|Supplementary_Methods.pdf| is a reviewer-friendly standalone document containing full mathematical definitions, parameter tables, sensitivity analysis figures, and exact data provenance metadata.

\subsection*{S7. Notes for reviewers}
- While the code repository contains executable scripts and notebooks, reviewers often expect a human-readable Supplementary Methods PDF; this file is provided for that purpose.
- For full reproducibility, reviewers can re-run the pipeline in a clean environment; see \verb|README.md| and \verb|requirements.txt| for environment setup.

\end{document}
