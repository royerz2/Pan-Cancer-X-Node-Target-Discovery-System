% Supplementary Information for:
% ALIN Framework (Adaptive Lethal Intersection Network)
% Roy Erzurumluoğlu

\documentclass[11pt]{article}
\usepackage[margin=1in]{geometry}
\usepackage{graphicx}
\usepackage{booktabs}
\usepackage{multirow}
\usepackage{amsmath}
\usepackage{amssymb}
\usepackage{microtype}
\usepackage{needspace}
\usepackage[section]{placeins}
\usepackage{caption}
\usepackage{multicol}
\usepackage{tikz}
\usepackage{tcolorbox}
\usetikzlibrary{arrows.meta,positioning,fit,calc}
\usepackage[colorlinks=false,pdfborder={0 0 0},breaklinks=true]{hyperref}
\usepackage{cleveref}
\usepackage{longtable}
\usepackage{array}

\begin{document}

\title{Supplementary Information: ALIN Framework (Adaptive Lethal Intersection Network)}
\author{Roy Erzurumluoğlu\\
  Maastricht University, Bachelor of Natural Sciences, Maastricht, The Netherlands.}
\maketitle

\tableofcontents
\newpage

%% ============================================================
%% SUPPLEMENTARY RESULTS (moved from main text)
%% ============================================================

\section{Supplementary Results}

\subsection{MHS non-uniqueness and biological plausibility}
\label{sec:mhs_nonunique}
For a combinatorial optimization problem, the uniqueness of the optimal solution is not guaranteed. We therefore enumerated up to 20 near-optimal MHS solutions per cancer type using iterative ILP with solution-exclusion constraints (k-best approach, cost tolerance $\leq 1.5\times$ optimal; Supplementary Figure~S16). Across 17 evaluable cancer types, 4~cancers (Anaplastic Thyroid, HNSCC, Melanoma, NSCLC) yielded the maximum 20 near-optimal solutions (mean pairwise Jaccard distance 0.40--0.55), indicating substantial solution degeneracy. The remaining 11~cancers yielded only 1 near-optimal solution, reflecting a tight cost landscape where the optimal MHS is effectively unique. Mean diversity across all cancers was Jaccard~$= 0.14$ (range 0.00--0.55).

To assess biological plausibility, we applied four hard constraints: (i)~$\geq$1 druggable target (druggability~$\geq 0.6$), (ii)~exclusion of pan-essential genes (essential in $>$90\% of DepMap lines), (iii)~mean combination toxicity~$< 0.8$, and (iv)~same-pathway redundancy penalty (max pairwise Jaccard of pathway memberships~$\leq 0.5$). No solutions were removed by the pan-essential or toxicity filters; pathway redundancy was the dominant constraint (mean 0.4 solutions removed per cancer type, removing both solutions for Diffuse Glioma and Renal Cell Carcinoma). Under these constraints, the top-1 MHS changed in \textbf{5/17 (29.4\%)} cancer types (HNSCC, NSCLC, Esophagogastric, Diffuse Glioma, Renal Cell Carcinoma), with mean Jaccard distance of 0.49 between unconstrained and constrained top-1. The non-uniqueness finding underscores that MHS predictions should be interpreted as representative solutions from a constrained landscape, not as uniquely determined optima.

\subsection{Novel combinations for rare cancers}
Notable predictions for cancers with limited treatment options: Chondrosarcoma (MCL1 alone), Retinoblastoma (OTX2+STAT3), Nerve Sheath Tumor (HNRNPH1+STAT3), and Pleural Mesothelioma (CCND1+FGFR1+MDM2+STAT3, 4-target). Five of 10 priority combinations had no partial clinical trial matches, indicating complete novelty.

\subsection{Validation and patient stratification}
Multi-source validation (PubMed, STRING, PRISM, ClinicalTrials.gov) was run on 10 priority combinations. All showed strong STRING PPI support. PRISM concordance scores ranged from 0.38 to 0.55. Patient stratification identified that 70--75\% of patients are addressable by existing biomarkers.

\subsection{Dual-vs.-triple escape-route validation}
\label{sec:dual_triple}
To test whether a third target provides functional value beyond the best dual,
we performed three complementary analyses across NSCLC ($n=98$),
Melanoma ($n=67$), and CRC ($n=63$) using PRISM primary-screen
viability data and a canonical signaling network (97~edges, 43~nodes;
see Supplementary Methods~S6).

\emph{(i)~PRISM Bliss independence model.} Single-agent viabilities
($V = 2^{\textrm{logFC}}$) from the PRISM primary screen were combined
under the Bliss independence assumption ($V_{\mathrm{combo}} = \prod V_i$)
for all $\binom{3}{2}$ dual and the full triple combinations.
In CRC, the triple (BRAF+EGFR+KRAS) achieved estimated viability
$V=0.046$ vs.\ best dual $V=0.034$ (BRAF+KRAS); KRAS
essentiality was proxied via CRISPR Chronos scores where no PRISM
drug column was available.

\emph{(ii)~Network escape-route enumeration.} Using DepMap-calibrated
essential signaling genes (Chronos $< -0.4$ in $\geq$15\% of lines per
cancer type) and BFS path enumeration through the canonical network,
we counted signaling routes from uninhibited essential genes to
downstream effectors (BCL2L1, MYC, CCND1, MAPK1, etc.)\ that bypass
all inhibited targets. Across all three cancers, the triple achieves
escape-route counts $\leq$ the best dual: NSCLC 73 vs.\ worst-dual 84
($-$13.1\%), Melanoma 45 vs.\ 47 ($-$4.3\%), CRC 45 vs.\ 73
($-$38.4\%). In NSCLC and Melanoma, the marginal gain from adding
EGFR to CDK6+MAP2K1 is zero because EGFR is upstream of MAP2K1 on the
same cascade; this supports the distinction between algorithmic triads
and mechanistic tri-axiality (Definition~1 in main text).

\emph{(iii)~Stochastic time-to-escape simulation.} Using per-route
reactivation probability $p=0.005$ per time step and 10{,}000 Monte
Carlo iterations, the triple produced significantly longer median
escape times than the worst dual in all three cancers (NSCLC
$p=1.4\times10^{-23}$; Melanoma $p=0.017$; CRC $p<10^{-100}$,
Mann--Whitney $U$ test, one-sided). This confirms that even when the
best dual matches the triple, the wrong dual choice produces
substantially faster escape.

\subsection{Exploratory: success vs.\ failure discrimination}
\label{sec:exploratory_auc}
As an exploratory analysis, we tested whether ALIN-compatible metrics can discriminate clinically \emph{successful} from \emph{failed} molecularly-targeted combinations.
We curated 30 oncology combination regimens (15 successes: FDA-approved or Phase~3 positive; 15 failures: Phase~2/3 terminated for futility or toxicity) independently of ALIN predictions, each with PMID or NCT identifiers (Supplementary Methods~S7, Table~S6). We pre-specified \textbf{co-essentiality AUC} as the primary metric before examining results.
Co-essentiality showed a suggestive but non-significant discriminative signal (AUC~$= 0.69$, 10{,}000-iteration stratified bootstrap 95\% CI~$[0.46$--$0.88]$, permutation $p = 0.058$, $n = 15$ vs.\ 15); 94.4\% of bootstrap resamples exceed AUC~$= 0.5$, and leave-one-out CV AUC was stable ($0.686 \pm 0.023$, range 0.657--0.734). Successful combinations had higher pairwise co-essentiality (median $r = 0.029$ vs.\ $-0.053$).
Pathway coherence provided complementary signal (AUC~$= 0.66$, perm.~$p = 0.063$). The composite score (AUC~$= 0.68$, perm.~$p = 0.042$) is statistically indistinguishable from co-essentiality alone ($\Delta$AUC~$= -0.001$, paired bootstrap $p = 0.98$), confirming no performance inflation from heuristic weighting.
This analysis is underpowered ($n = 30$; 95\% CIs cross 0.5) and should be treated as hypothesis-generating for future validation in larger outcome-labelled datasets, not as evidence of clinical-grade discriminative ability.

\subsection{Hub-handling strategy comparison}
\label{sec:hub_strategies}
The ad hoc proportional hub penalty ($1.5\times$ excess over median frequency) was introduced to correct STAT3 over-representation but was never systematically validated against alternatives. We compared five strategies on 17 evaluable cancer types against the 43-entry gold standard:
\begin{enumerate}
\item \textbf{No hub penalty}: baseline without any frequency correction.
\item \textbf{Proportional penalty} (current): $1.5 \times (f_g - \tilde{f})$ for genes above median.
\item \textbf{PageRank-normalized coverage}: path coverage weighted by inverse PageRank centrality, replacing the separate hub penalty with implicit degree correction.
\item \textbf{Lean scoring}: removes all literature-curated features (resistance, combo-toxicity, evidence exemptions), retaining only DepMap essentiality, co-essentiality synergy, path coverage, druggability, and the proportional hub penalty.
\item \textbf{PageRank + lean}: combines PageRank normalization with literature-feature removal.
\end{enumerate}

\noindent Key findings (Table~\ref{tab:hub_strategies_supp}):

\begin{table}[!htb]
\centering
\scriptsize
\caption{\textbf{Hub-handling strategy comparison.} Any-overlap and pair-overlap against 43-entry gold standard; hub dominance metrics across 17 cancer types. Gini = gene-frequency Gini coefficient (lower = more equitable).}
\label{tab:hub_strategies_supp}
\begin{tabular}{@{}lccccc@{}}
\toprule
Strategy & Any & Pair & Max\% & STAT3\% & Gini \\
\midrule
No penalty & 34.9 & 25.6 & 64.7 & 64.7 & 0.357 \\
Proportional & 23.3 & 18.6 & 82.4 & 5.9 & 0.414 \\
\textbf{PageRank} & \textbf{34.9} & \textbf{23.3} & \textbf{70.6} & \textbf{47.1} & \textbf{0.318} \\
Lean & 23.3 & 9.3 & 52.9 & 0.0 & 0.386 \\
PageRank+lean & 27.9 & 11.6 & 100.0 & 100.0 & 0.488 \\
\bottomrule
\end{tabular}
\end{table}

\noindent The proportional hub penalty is \emph{counterproductive}: it reduces benchmark concordance ($-$11.6~pp any-overlap, $-$7.0~pp pair-overlap vs.\ no penalty) while merely substituting STAT3 dominance (64.7\%) with MAP2K1 dominance (82.4\%, Gini 0.414 = least equitable). PageRank-normalized coverage matches the no-penalty baseline at any-overlap (34.9\%) and nearly matches pair-overlap (23.3\% vs.\ 25.6\%), while achieving the lowest Gini coefficient (0.318) and reducing STAT3 from 64.7\% to 47.1\%---a principled balance of biological relevance and target diversity.

\textbf{Literature-feature decorativeness.} Lean scoring (literature features removed, proportional penalty retained) achieves identical concordance to the proportional-penalty baseline (23.3\% any-overlap), confirming the circularity ablation finding: literature-curated features---\texttt{KNOWN\_SYNERGIES}, \texttt{RESISTANCE\_MECHANISMS}, combo-toxicity, and perturbation bonus---contribute $\Delta = 0.0$~pp across all hub-handling strategies. ALIN's empirically active scoring components are: DepMap essentiality/specificity (via cost function), OmniPath path coverage, co-essentiality synergy, druggability, and the choice of hub-handling strategy. The remaining literature features are retained for reporting transparency (evidence tiers, toxicity alerts) but should not be interpreted as contributing to predictive concordance.

Based on these results, we recommend PageRank-normalized coverage as the principled replacement for the proportional hub penalty. This approach has a clear statistical foundation (degree normalization via random walk), does not require arbitrary multiplier tuning, and achieves the best trade-off between benchmark concordance and target diversity.

\subsection{Evidence calibration and de-correlation}
\label{sec:evidence_calibration}
The combined score (Equation~3 in main text) sums eight weighted components with hand-tuned weights. To assess whether these features are redundant, miscalibrated, or decorative, we collected all 25{,}667 scored triple and doublet combinations across 50 cancer types and labelled each by overlap with the 43-entry gold standard (2{,}687 positives, 10.5\%).

\textbf{Feature correlations.} Pairwise Pearson correlation reveals one highly collinear pair: \emph{coverage} and \emph{hub\_penalty} ($r = 0.84$, Spearman $\rho = 0.87$). This is expected: both are determined by gene--path frequency. The remaining $\binom{8}{2} - 1 = 27$ pairs have $|r| < 0.48$. Removing \emph{hub\_penalty} from the de-correlated set eliminates the redundancy without discarding information, since coverage already captures path-frequency content.

\textbf{Calibrated model vs.\ heuristic.} A logistic regression (L2-regularized, balanced class weights) with leave-one-cancer-out (LOCO) cross-validation achieves AUROC = 0.552 on the held-out cancers, indistinguishable from the heuristic combined score (AUROC = 0.553). Isotonic calibration on top of the LR overfits to the small positive fraction (AUROC = 0.353), confirming that the marginal positive rate per cancer is too sparse for nonparametric calibration. On the independent 30-combo outcome benchmark, a composite-vs-individual ablation (10{,}000 paired bootstrap) confirms $\Delta$AUC~$= -0.001$ between the composite and the best individual feature (co-essentiality; 95\% CI~$[-0.24, +0.25]$, $p = 0.98$). The composite thus does not inflate discriminative performance; it is retained for multi-dimensional reporting transparency. Neither heuristic nor data-fit model has strong discriminative power (random baseline AUPRC = 10.5\%, heuristic AUPRC = 13.2\%, LR AUPRC = 8.0\%), but the heuristic is not pathologically miscalibrated.

\textbf{Per-feature importance (SHAP and ablation).} SHAP analysis (linear explainer on the full LR model) ranks features: druggable count (mean $|$SHAP$|$ = 0.274), hub penalty (0.188), resistance (0.122), coverage (0.109), combo toxicity (0.091), synergy (0.071), cost (0.012), and perturbation bonus (0.006). Leave-one-feature-out ablation (LOCO AUROC drop when feature removed) confirms this hierarchy: only druggable count produces a meaningful $\Delta$AUROC ($+$0.030); perturbation bonus ($\Delta = +0.001$) and total cost ($\Delta = -0.001$) are negligible, consistent with the earlier $\Delta = 0.0$ circularity finding. Hub penalty and coverage ablation each \emph{improve} AUROC slightly ($-$0.008 and $-$0.012 respectively), consistent with their collinearity: removing either reduces noise from the redundant pair.

\textbf{Recommendation.} The analysis supports retaining four empirically active features---druggable count, coverage, resistance risk, and synergy---while demoting hub penalty (subsumed by coverage), combo toxicity (modest contribution), total cost (negligible), and perturbation bonus (negligible) to reporting-only status. This de-correlated feature set should guide any future weight re-optimization or model-based scoring replacement.

\subsection{Negative controls and failure analysis}
\label{sec:negative_controls}
To assess whether ALIN's predictions are biologically plausible beyond the gold-standard benchmark---which measures recall but cannot detect false positives---we performed two complementary analyses: (i)~identification of implausible high-confidence predictions (negative controls), and (ii)~evaluation against known-failed clinical combinations.

\textbf{Implausible predictions.} We manually reviewed all 17 benchmark cancer predictions and the 77-cancer pan-cancer output, flagging top-1 predictions that conflict with known biology (Table~\ref{tab:neg_controls_supp}). Eight predictions are identified as biologically suspect, four at high severity:

\begin{table}[htbp]
\centering
\caption{\textbf{Implausible high-confidence ALIN predictions (negative controls).} Eight predictions flagged as biologically suspect. Severity: high = target class has no clinical rationale in the cancer type; moderate = target selection reflects algorithmic artefact (hub bias, wrong isoform) rather than cancer-specific biology.}
\label{tab:neg_controls_supp}
\resizebox{\columnwidth}{!}{%
\begin{tabular}{@{}llp{5.5cm}l@{}}
\toprule
\textbf{Cancer} & \textbf{Prediction} & \textbf{Implausibility} & \textbf{Sev.} \\
\midrule
RCC & EGFR+MET & EGFR not a validated RCC target; erlotinib Phase~2 ORR $<$5\% & High \\
Prostate & CDK2+EGFR+MAP2K1 & AR pathway missed; EGFR failed in CRPC trials & High \\
Liposarcoma & EGFR+FGFR1+MET & MDM2+CDK4 driver axis missed; generic RTK triple & High \\
Non-Cancerous & CDK6+FYN+STAT3 & Non-malignant tissue; hub artefact & High \\
AML & CDK4+CDK6+MCL1 & MCL1 cardiotoxicity (Phase~1 DLT); FLT3/BCL2 axis missed & Mod. \\
HNSCC & FYN+STAT3 & Both targets are network-degree artefacts; no single-agent activity & Mod. \\
Endometrial & MAP2K1+STAT3 & PI3K/MTOR pathway missed; MAPK hub artefact & Mod. \\
HCC & FGFR1+STAT3 & FGFR1 wrong isoform (FGFR2 in cholangio); KDR invisible to CRISPR & Mod. \\
\bottomrule
\end{tabular}%
}
\end{table}

\noindent Common failure modes fall into four categories: (1)~\textbf{CRISPR-invisible targets}---hormonal (AR/CYP17A1 in prostate), anti-angiogenic (KDR), and immune-checkpoint targets are not cell-autonomous essentialities and cannot be recovered by a CRISPR-based pipeline; (2)~\textbf{hub-gene artefacts}---STAT3 and FYN appear in 52/77 and 28/77 predictions respectively due to high OmniPath degree, inflating their selection beyond cancer-specific evidence; (3)~\textbf{wrong isoform/lineage}---FGFR1 substitutes for the clinically relevant FGFR2 in liver cancers, and EGFR is selected in RCC/prostate where it lacks activating alterations; (4)~\textbf{missing lineage drivers}---MDM2 (liposarcoma) and FLT3/IDH (AML) are absent from \texttt{GENE\_TO\_DRUGS} or not CRISPR-essential.

\textbf{Known failed combinations.} We curated 8 clinically documented combination failures where targets overlap with ALIN's prediction space. For each, we checked whether ALIN's top-1 prediction for the same cancer fully contains, partially overlaps, or correctly avoids the failed target set:

\begin{itemize}\setlength{\itemsep}{0pt}
\item \textbf{4/8 predicted} (false positives): ALIN predicts EGFR in RCC (erlotinib ORR $<$5\%), EGFR in prostate (failed Phase~2), MCL1 in AML (cardiotoxicity DLT), and BRAF without MEK in melanoma (rapid resistance, PFS 5--8~mo).
\item \textbf{2/8 partial overlap}: melanoma predictions overlap with the failed BRAF+MEK+CDK4/6 triple (encorafenib+binimetinib+ribociclib; ORR 52\% $<$ 64\% doublet) and with CDK4/6+BRAF without MEK.
\item \textbf{2/8 correctly avoided}: ALIN does not predict STAT3 (napabucasin) for CRC, which failed Phase~3 (CanStem111P), nor the PI3K+MEK combination for ovarian cancer (ORR 4.7\%).
\end{itemize}

\noindent The 4/8 false-positive rate against known failures---and the 8/77 implausible prediction rate---quantify the precision cost of a CRISPR-only target discovery pipeline. Proposed post-hoc filters include: (i)~excluding non-malignant lineages, (ii)~a hub-gene frequency cap (reject targets appearing in $>$60\% of predictions without cancer-specific mutation/amplification evidence), (iii)~lineage-specific driver-pathway priors (require $\geq$1 target from the cancer's known driver pathway), (iv)~clinical toxicity cross-check (flag combinations where single-agent DLTs overlap), and (v)~negative-evidence integration (down-weight targets from failed indication-specific trials). These filters are not implemented in the current pipeline to preserve its purely algorithmic nature, but they are recommended for any translational application.

\subsection{Translational feasibility scoring}
\label{sec:feasibility}
To bridge algorithmic predictions to clinical actionability, we developed a post-hoc translational feasibility score that integrates four components: (i)~\textbf{drug availability}---clinical stage and number of approved/investigational agents per target (approved $= 1.0$, Phase~3 $= 0.80$, Phase~2 $= 0.55$, Phase~1 $= 0.30$, preclinical $= 0.10$, no drug $= 0$; averaged across targets); (ii)~\textbf{selectivity}---penalties for shared drugs between targets (e.g.\ CDK4+CDK6 share palbociclib; $-0.25$), pan-kinase inhibitors ($-0.08$ per target), and narrow therapeutic windows ($-0.12$ per target); (iii)~\textbf{combination toxicity feasibility}---penalties for overlapping organ toxicities (myelosuppression/cardiotoxicity overlap: $-0.25$ per additional target; GI/dermatologic overlap: $-0.05$--$0.10$) and known drug--drug interactions from a curated 11-pair DDI database; (iv)~\textbf{clinical precedent}---whether any target subset has been tested in the same cancer type (from the 43-entry gold standard; FDA-approved $= 1.0$, Phase~2 $= 0.6$, partial overlap $= 0.5\times$). The composite score $F = 0.30 \times \text{availability} + 0.25 \times \text{tox\_feasibility} + 0.25 \times \text{selectivity} + 0.20 \times \text{precedent}$ ranges from 0 (infeasible) to 1 (highly feasible).

Across the 17 benchmark cancer predictions, the mean feasibility score is 0.783 (range 0.580--0.988). 16/17 predictions achieve high feasibility ($F \geq 0.60$), reflecting that most triples comprise approved-drug targets. All 17 predictions map every target to at least one drug; 11/17 have all three targets with FDA-approved agents. 21 red flags are identified across 17 predictions, predominantly moderate-severity overlapping organ toxicities (dermatologic: EGFR+MAP2K1 rash in 7 cancers; gastrointestinal: EGFR+MET/KRAS nausea/diarrhea in 5 cancers). Three high-severity flags involve overlapping myelosuppression (CDK4+CDK6 in AML and HNSCC) and a known moderate DDI (erlotinib+alpelisib in esophagogastric cancer). No critical-severity flags (e.g.\ additive cardiotoxicity or QT prolongation) are present in the benchmark set.

The feasibility-adjusted ranking, computed as $S_{\text{adj}} = S_{\text{combined}} / \max(F, 0.20)$, re-orders 5/17 predictions by $\geq$3 ranks (Table~\ref{tab:feasibility_supp}). The largest upward moves are \textbf{NSCLC} (CDK6+EGFR+MAP2K1; \#12 $\to$ \#3, $F = 0.988$) and \textbf{hepatocellular carcinoma} (EGFR+FGFR1+MET; \#15 $\to$ \#6, $F = 0.945$). The largest downward moves are \textbf{endometrial carcinoma} (CDK2+EGFR+MET; \#11 $\to$ \#15, $F = 0.650$) and \textbf{esophagogastric adenocarcinoma} (CDK6+EGFR+PIK3CA; \#9 $\to$ \#13, $F = 0.732$). The \textbf{AML} prediction (CDK4+CDK6+MCL1) has the lowest feasibility ($F = 0.580$) due to shared CDK4/6 drugs, MCL1 cardiotoxicity dose-limitation, and overlapping myelosuppression.

\begin{table}[htbp]
\centering
\caption{\textbf{Feasibility-adjusted re-ranking of top triple predictions.} $F$: composite feasibility score (0--1). $S_{\text{adj}} = S_{\text{combined}} / \max(F, 0.20)$; lower $=$ better. $\Delta$Rank: positive $=$ improved by feasibility.}
\label{tab:feasibility_supp}
\resizebox{\columnwidth}{!}{%
\begin{tabular}{@{}lllccccc@{}}
\toprule
\textbf{Cancer} & \textbf{Triple} & \textbf{Drugs} & $S_{\text{comb}}$ & $F$ & $S_{\text{adj}}$ & \textbf{Rank} & $\Delta$ \\
\midrule
Pancreatic & BRAF+KRAS+STAT3 & vem+sot+nap & 0.390 & 0.738 & 0.529 & 1$\to$1 & 0 \\
Colorectal & BRAF+EGFR+KRAS & vem+erl+sot & 0.919 & 0.952 & 0.965 & 4$\to$2 & +2 \\
\textbf{NSCLC} & \textbf{CDK6+EGFR+MAP2K1} & \textbf{pal+erl+tra} & \textbf{0.989} & \textbf{0.988} & \textbf{1.002} & \textbf{12$\to$3} & \textbf{+9} \\
ATC & CDK6+EGFR+MAP2K1 & pal+erl+tra & 0.896 & 0.887 & 1.010 & 2$\to$4 & $-$2 \\
Melanoma & CDK6+EGFR+MAP2K1 & pal+erl+tra & 0.898 & 0.887 & 1.012 & 3$\to$5 & $-$2 \\
\textbf{HCC} & \textbf{EGFR+FGFR1+MET} & \textbf{erl+erd+cap} & \textbf{1.044} & \textbf{0.945} & \textbf{1.105} & \textbf{15$\to$6} & \textbf{+9} \\
\textbf{Esophagogastric} & \textbf{CDK6+EGFR+PIK3CA} & \textbf{pal+erl+alp} & \textbf{0.968} & \textbf{0.732} & \textbf{1.322} & \textbf{9$\to$13} & $\mathbf{-4}$ \\
\textbf{Endometrial} & \textbf{CDK2+EGFR+MET} & \textbf{din+erl+cap} & \textbf{0.979} & \textbf{0.650} & \textbf{1.506} & \textbf{11$\to$15} & $\mathbf{-4}$ \\
HNSCC & CDK4+CDK6+ERBB2 & pal+pal+tra & 1.060 & 0.675 & 1.570 & 16$\to$16 & 0 \\
AML & CDK4+CDK6+MCL1 & pal+pal+AMG & 1.140 & 0.580 & 1.966 & 17$\to$17 & 0 \\
\bottomrule
\end{tabular}%
}
\end{table}

\noindent \textbf{Key selectivity caveats.} (1)~CDK4 and CDK6 are targeted by the \emph{same} drugs (palbociclib, ribociclib, abemaciclib), so the CDK4+CDK6+MCL1 triple in AML is pharmacologically a doublet, not a triplet. (2)~FYN and SRC share dasatinib, a pan-kinase inhibitor hitting $>$40 kinases at clinical concentrations; predictions containing both targets (28/77 pan-cancer) overstate pharmacological independence. (3)~CDK2 inhibition with dinaciclib is complicated by pan-CDK activity (CDK1/2/5/9), contributing to severe myelosuppression that limits combination potential.

\subsection{End-to-end case studies}
\label{sec:case_studies}
To illustrate how the full pipeline progresses from raw DepMap/OmniPath inputs to a ranked, drug-annotated, feasibility-scored prediction, we trace three cancer types end to end: colorectal adenocarcinoma (strong clinical concordance), NSCLC (highest feasibility), and AML (instructive failure). Each case reports inferred survival mechanisms, top MHS solutions, ranked triples, external validation signals, and at least one negative control.

\subsubsection{Case 1: Colorectal adenocarcinoma --- validated convergence}

\textbf{Inferred mechanisms.} From 96 CRC cell lines in DepMap 24Q4, the pipeline infers $\sim$50 viability paths across four modules: (i)~co-essentiality modules (Jaccard-based Ward clustering of 96 Chronos profiles; conf.~0.9) identify coordinately essential gene sets including a MAPK-pathway cluster (BRAF, KRAS, EGFR, SOS1) and a cell-cycle cluster (CCND1, CDK4, CDK6); (ii)~OmniPath signaling paths (directed 2--4-hop traversals from 14 driver genes to 10 effector genes, scored by mean Chronos dependency) recover KRAS$\to$BRAF$\to$MAP2K1$\to$CCND1, EGFR$\to$KRAS$\to$BRAF, and PIK3CA$\to$AKT1$\to$MTOR among others; (iii)~a lineage-aware cancer-specific module (Welch $t$-test, FDR $q < 0.05$, Cohen's $d > 0.3$ vs.\ non-CRC lineages) identifies CRC-selective dependencies including $\beta$-catenin (CTNNB1) and APC-pathway genes; (iv)~curated perturbation-response paths capture EGFR inhibition$\to$MAPK reactivation signatures.

\textbf{MHS solutions.} The ILP solver returns a single optimal MHS: \{BRAF, CCND1, CTNNB1, STAT3\} (cost 4.60, coverage 76.0\%). This 4-gene set intersects every inferred viability path at minimum cost. No alternative solutions are found (gene entropy $= 0$), indicating that CRC's mechanism structure is highly constrained.

\textbf{Ranked triples.} The triple combination scorer selects \textbf{BRAF + EGFR + KRAS} (combined score 0.919) as the top-1 triple, with drugs vemurafenib + erlotinib + sotorasib. The transition from MHS to triple is instructive: STAT3 is removed by the hub-gene penalty; CCND1 (not directly targetable) is replaced by its downstream effectors; CTNNB1 (undruggable) is replaced by an upstream targetable driver.

\textbf{Ablation stability.} Removing the statistical (lineage-aware) module yields the \emph{same} BRAF+EGFR+KRAS prediction. Removing co-essentiality degrades prediction to BCL2L1+ERBB2+MET; removing perturbation yields BRAF+CDK4+ERBB2; disabling the hub penalty substitutes STAT3 for EGFR (CDK4+KRAS+STAT3). CRC's prediction is thus jointly determined by co-essentiality and OmniPath signaling, with the hub penalty performing a critical swap (STAT3$\to$EGFR).

\textbf{External validation and negatives.} All three pairwise subsets have clinical evidence: BRAF+EGFR is FDA-approved (encorafenib+cetuximab; BEACON Phase~3, OS HR 0.60); the BRAF+EGFR+MAP2K1 triplet arm achieved ORR 26\% in Phase~3; KRAS+EGFR is in Phase~2 (sotorasib+panitumumab; CodeBreaK~101, ORR 30\%). Feasibility score: $F = 0.952$ (\#2 of 17). \textbf{Negative:} ALIN correctly avoids the failed STAT3/napabucasin monotherapy in CRC (CanStem111P Phase~3).

\subsubsection{Case 2: Non-small cell lung cancer --- highest feasibility}

\textbf{Inferred mechanisms.} From 165 NSCLC cell lines (the largest lineage in DepMap), the pipeline infers $\sim$57 viability paths. NSCLC's rich mutational landscape generates dense signaling path coverage.

\textbf{MHS solutions.} NSCLC produces 20 near-optimal MHS solutions (the enumeration ceiling). The unconstrained top-1 is \{CCND1, CDK4, CDK6, STAT3\} (cost 4.63, coverage 78.9\%); after constraint filtering, the constrained top-1 shifts to \{CCND1, CDK4, CDK6, MCL1\} (cost 5.04, coverage 80.7\%).

\textbf{Ranked triples.} The triple scorer selects \textbf{CDK6 + EGFR + MAP2K1} (combined score 0.989), with drugs palbociclib + erlotinib + trametinib.

\textbf{Ablation stability.} The CDK6+EGFR+MAP2K1 prediction is the most stable in the benchmark: it survives removal of the statistical (lineage-aware) module identically.

\textbf{External validation and negatives.} Individual target pairs have extensive clinical precedent: BRAF+MAP2K1 is FDA-approved in NSCLC (dabrafenib+trametinib); EGFR+MET is FDA-approved (amivantamab). Feasibility score: $F = 0.988$ (highest of all 17 predictions). \textbf{Negative:} In melanoma, the BRAF+MEK+CDK4/6 triplet failed Phase~1/2 (ORR 52\% $<$ 64\% doublet).

\subsubsection{Case 3: Acute myeloid leukemia --- instructive failure}

\textbf{Inferred mechanisms.} From 64 AML cell lines, the pipeline infers $\sim$29 viability paths---fewer than CRC or NSCLC. Critically, AML's canonical oncogenic drivers (FLT3, IDH1/2, NPM1, KMT2A) are \emph{not} in the curated 14-gene driver set.

\textbf{MHS solutions.} The ILP solver returns a single optimal MHS: \{CDK4, CDK6, SOS1, STAT3\} (cost 4.17, coverage 58.6\%). Coverage is the lowest among the three case studies.

\textbf{Ranked triples.} The triple scorer selects \textbf{CDK4 + CDK6 + MCL1} (combined score 1.140, the worst in the benchmark). This prediction has three compounding problems: (i)~CDK4 and CDK6 are pharmacologically redundant (both inhibited by palbociclib); (ii)~MCL1 inhibitors caused dose-limiting cardiotoxicity in Phase~1; (iii)~overlapping myelosuppression between CDK4/6 inhibitors and AML's inherent cytopenias.

\textbf{Ablation stability.} The prediction is unstable across ablation conditions, further confirming low confidence.

\textbf{External validation and negatives.} The validated AML combinations target fundamentally different biology: FLT3+BCL2 (venetoclax+gilteritinib; composite CR in Phase~2), IDH1+BCL2 (ivosidenib+venetoclax; CR 71\%). These operate through lineage-specific oncogene addiction and apoptotic priming---mechanisms invisible to a CRISPR-essentiality pipeline. Feasibility score: $F = 0.580$ (lowest of 17). \textbf{Negative:} ALIN predicts the failed MCL1 inhibitor class (Phase~1 DLT: cardiotoxicity), making this a documented false positive.

\subsection{Pathway shifting simulation}
\label{sec:pathway_sim}
An ODE-based systems biology model illustrating the deductive consequences of the tri-axial hypothesis under assumed compensatory signaling parameters is presented in Supplementary Methods~S5 and Figure~S4. Under these assumptions, tri-axial combinations yield $\sim$30\% lower modeled tumor viability than intra-axial combinations; however, this result is a logical consequence of the model's axioms, not an independent prediction.

\subsection{Additional limitations (7--13)}
\label{sec:additional_limitations}

\textbf{7. MHS non-uniqueness.} The weighted hitting set problem admits multiple near-optimal solutions; the reported MHS is not uniquely determined. Iterative ILP enumeration shows that 4/17 cancer types yield $\geq$20 near-optimal solutions (mean Jaccard 0.40--0.55), and biological-plausibility constraints change the top-1 MHS in 29.4\% of cancer types.

\textbf{8. Network degree bias.} OmniPath is literature-curated, with known degree biases toward well-studied proteins (STAT3, TP53, AKT, MAPK family). PageRank-normalized path coverage provides principled degree correction in ranked triples; MHS predictions remain susceptible.

\textbf{9. Proliferation assay confound.} CRISPR screens measure growth/viability, so genes essential for proliferation (cell cycle regulators, DNA replication factors) are inherently favored. The ``downstream effector'' role assigned to CCND1/CDK4/CDK6 may partly reflect this assay property. Future analyses should control for core proliferation signatures.

\textbf{10. Heuristic scoring weights.} The eight-component combined score uses domain-informed but hand-tuned weights. A calibrated logistic regression achieves indistinguishable AUROC (0.552 vs.\ 0.553), and a composite-vs-individual ablation confirms $\Delta$AUC~$= -0.001$ ($p = 0.98$). Coverage and hub penalty are collinear ($r = 0.84$); perturbation bonus and total cost contribute negligibly (SHAP $< 0.013$).

\textbf{11. Druggability.} Several MHS components lack clinical-stage drugs, though 88\% (68/77) of predictions include at least one druggable target.

\textbf{12. False positives and CRISPR-invisible targets.} A systematic negative-control analysis identifies 8 implausible predictions and shows that ALIN fully predicts 4/8 known failed clinical combinations. Key blind spots include hormonal pathway targets (AR axis in prostate), anti-angiogenic targets (KDR), and targets with narrow therapeutic windows (MCL1 cardiotoxicity).

\textbf{13. Feasibility scoring is post-hoc and heuristic.} The translational feasibility score uses expert-informed but hand-tuned component weights and a curated 11-pair DDI database. The score cannot replace formal PK/PD modeling, clinical safety review, or regulatory assessment.

%% ============================================================
%% SUPPLEMENTARY METHODS (original)
%% ============================================================

\section{Supplementary Methods}

\subsection{S1. Data sources and version control}
All data were accessed in January 2026 and cached locally for reproducibility. DepMap release 24Q4 was downloaded from \url{https://depmap.org/portal/download/all/} (files: CRISPRGeneEffect.csv, Model.csv, OmicsDefaultModelProfiles.csv). The OmniPath directed signaling network was obtained via the OmniPath Python client (v1.0.8; \url{https://omnipathdb.org}). PRISM secondary repurposing screen data (19Q4) were downloaded from \url{https://depmap.org/repurposing/}. GDSC drug sensitivity data (IC$_{50}$ and AUC) were obtained from \url{https://www.cancerrxgene.org}. Exact file URLs, SHA-256 checksums, and download dates are recorded in the repository file DATA\_AVAILABILITY.md to enable precise data provenance tracking.

\subsection{S2. Survival mechanism inference: mathematical definitions}

\paragraph{Essentiality threshold.} Gene $g$ is essential in cell line $\ell$ if its Chronos score satisfies $D_{g,\ell} < \theta_{\text{dep}}$ where $\theta_{\text{dep}} = -0.5$ (standard DepMap threshold for gene essentiality). Gene $g$ is selective for cancer type $c$ if it is essential in at least fraction $\theta_{\text{sel}} = 0.30$ of cell lines within $c$:
\[
\resizebox{\columnwidth}{!}{$\displaystyle
\text{Selective}(g, c) = \mathbb{1}\!\left[\frac{|\{\ell \in L_c : D_{g,\ell} < \theta_{\text{dep}}\}|}{|L_c|} \geq \theta_{\text{sel}}\right]
$}
\]
where $L_c$ is the set of cell lines annotated to cancer type $c$. Pan-essential genes, those essential in $>$90\% of all cell lines across all cancer types, are excluded to remove housekeeping dependencies.

\paragraph{Co-essentiality (Jaccard index).} For genes $g_i, g_j$ in cancer $c$, let $E_i = \{\ell \in L_c : D_{g_i,\ell} < \theta_{\text{dep}}\}$. The co-essentiality score is:
\[
\text{CoEss}(g_i, g_j) = \frac{|E_i \cap E_j|}{|E_i \cup E_j|}
\]
This Jaccard index ranges from 0 (no shared essentiality) to 1 (identical essentiality profiles). The resulting pairwise distance matrix ($1 - \text{CoEss}$) is clustered using Ward's agglomerative method with a dynamic tree cut producing 3--15 modules per cancer type. Each module constitutes a survival mechanism.

\paragraph{Pearson co-essentiality comparison.} As an alternative to Jaccard binarization, we computed pairwise Pearson correlation on continuous Chronos dependency scores for each cancer type, converting to a distance matrix $d_{ij} = 1 - |r_{ij}|$. Both Jaccard and Pearson distance matrices were clustered with Ward's method for $k \in \{3, 4, \ldots, 15\}$, and agreement was measured by normalized mutual information (NMI) between the two partitions. Silhouette scores were computed for each clustering to assess internal cohesion. Across 10 cancer types with $\geq$74 cell lines (range 74--165; 949--1,340 selective genes), mean NMI was $0.20 \pm 0.07$, increasing modestly with $k$ (0.16 at $k{=}3$ to 0.22 at $k{=}15$). Per-cancer NMI ranged from 0.08 (Lung Neuroendocrine Tumor) to 0.30 (Diffuse Glioma). Jaccard clusterings yielded higher silhouette scores than Pearson at all $k$ values (mean 0.054 vs.\ 0.025), indicating tighter module boundaries under discretized profiles.

\paragraph{Signaling path scoring.} For a directed path $p = (g_1 \to g_2 \to \cdots \to g_k)$ in the OmniPath network (maximum $k = 4$ hops), the dependency score is:
\[
S_{\text{path}}(p) = \frac{1}{k} \sum_{i=1}^{k} |\bar{D}_{g_i,c}|
\]
where $\bar{D}_{g_i,c}$ is the mean Chronos score of gene $g_i$ across cell lines of cancer $c$. Paths with $S_{\text{path}} < 0.3$ are pruned as insufficiently essential.

\paragraph{Cancer-specific statistical testing.} Gene $g$ is cancer-specific for cancer $c$ if it satisfies:
\begin{enumerate}
\item Welch's $t$-test: $q_{g,c} < 0.05$ (Benjamini--Hochberg FDR-corrected; comparing Chronos scores of $g$ in $L_c$ vs.\ all other cell lines);
\item Effect size: Cohen's $d_{g,c} > 0.3$ (small-to-medium effect size threshold).
\end{enumerate}

\paragraph{Lineage-aware regression alternative.} To control for shared lineage dependencies, we implemented an OLS regression model for each gene $g$:
\[
D_{g,\ell} = \beta_0 + \sum_{j=1}^{33} \beta_j \, \text{lin}_{j,\ell} + \beta_{\text{cancer}} \, \mathbb{1}_{\ell \in L_c} + \varepsilon_{g,\ell}
\]
where $\text{lin}_{j,\ell}$ are OncotreeLineage dummy variables (34 lineages, one dropped) and $\mathbb{1}_{\ell \in L_c} = 1$ if cell line $\ell$ belongs to the target cancer type. Across 17 gold-standard cancer types, the lineage-aware model identifies substantially different gene sets from the Welch $t$-test (median Jaccard similarity = 0.051, mean = 0.118). Final triple predictions change for only 2/17 cancer types (11.8\%): Colorectal (EGFR+KRAS+MET $\to$ BRAF+EGFR+KRAS) and Breast (BCL2L1+EGFR+MAP2K1 $\to$ CDK4+ERBB2+PIK3CA). Benchmark concordance increases modestly from 46.5\% to 48.8\% any-overlap.

\paragraph{Perturbation-response signatures (literature-curated).} We manually curate published perturbation data from $\sim$15 kinase-inhibitor studies to identify feedback and bypass genes. Combinations that target perturbation-identified feedback genes receive a perturbation bonus $\beta_{\text{pert}} = 0.05$ per feedback gene covered, up to a maximum of 0.15. This module does not integrate systematic high-throughput perturbation databases (e.g., LINCS L1000).

\subsection{S3. MHS cost function components}

The weighted cost for including gene $g$ in the MHS for cancer $c$ (Equation~2 in main text) comprises four terms:

\textbf{Toxicity} $\tau(g)$: Derived from OpenTargets safety data and DrugTargetDB adverse event profiles. Scores range from 0.0 (no known toxicity) to 1.0 (severe dose-limiting toxicities in clinical use).

\textbf{Tumor specificity} $s(g,c)$: The selectivity of gene $g$ for cancer $c$ relative to all cancers:
\[
s(g,c) = \frac{f_{\text{ess}}(g,c) - \bar{f}_{\text{ess}}(g)}{\max_c f_{\text{ess}}(g,c) - \min_c f_{\text{ess}}(g,c) + \epsilon}
\]

\textbf{Druggability} $d(g)$: Binary-graded assessment: $d(g) = 1.0$ if an FDA-approved inhibitor exists; $d(g) = 0.6$ for Phase 2/3; $d(g) = 0.3$ for Phase 1; $d(g) = 0.2$ for preclinical-only; $d(g) = 0.0$ for undruggable targets.

\textbf{Pan-essentiality penalty} $\mathbb{1}_{\text{pan}}$: Equals 1 if the gene is essential in $>$90\% of all cell lines; 0 otherwise.

\subsection{S4. Ranked triple scoring: component definitions}

\textbf{Synergy score.} Derived from: (1) known synergistic pairs from curated clinical data and (2) pathway diversity. Evidence-adaptive weighting ensures that FDA-approved combinations receive appropriate credit.

\textbf{Resistance probability.} Estimated from curated resistance mechanisms: for each target, if known bypass mechanisms exist and the bypass gene is \emph{not} covered by another target, resistance probability increases by 0.15 per uncovered bypass.

\textbf{Combination toxicity (combo-tox).} Computed as:
\[
\text{combo-tox} = \sum_{\text{DDI pairs}} w_{\text{DDI}} + \sum_{\text{shared tox}} w_{\text{overlap}}
\]

\textbf{Path coverage.} Fraction of all inferred survival mechanisms intersected by at least one gene in the triple. Minimum 70\% for inclusion.

\textbf{Druggability count} $n_{\text{drugs}}$: Number of genes with FDA-approved or clinical-stage inhibitors.

\subsection{S5. ODE model parameters and justification}

\begin{table}[htbp]
\centering
\small
\caption{ODE model parameter values and biological justification.}
\begin{tabular}{@{}llll@{}}
\toprule
\textbf{Parameter} & \textbf{Symbol} & \textbf{Value} & \textbf{Justification} \\
\midrule
Basal production (drivers) & $b_i$ & 0.25--0.50 & Constitutive oncogene activation \\
Basal production (cascade) & $b_i$ & 0.03--0.08 & Dependent on upstream signaling \\
Basal production (orthogonal) & $b_i$ & 0.05--0.10 & Independent but regulatable \\
Degradation rate & $d_i$ & 0.05 h$^{-1}$ & Protein half-life $\sim$14 h \\
Hill coefficient & $n$ & 2 & Standard cooperativity \\
Hill half-max & $K$ & 0.5 & Normalized activity scale \\
Drug strength & $s_i$ & 0.92 & 92\% maximal inhibition \\
Drug onset rate & $\alpha$ & 0.15 h$^{-1}$ & $t_{1/2} \approx 4.6$ h to steady state \\
Compensatory gain (orthogonal) & $g_i$ & 0.4--0.7 & FYN$\to$STAT3 de-repression \\
Compensatory gain (cascade) & $g_i$ & 0.1--0.2 & Limited compensatory capacity \\
Simulation duration & $T$ & 4800 h & 200 days (Liaki observation window) \\
Integration method & & RK45 & Adaptive step-size (scipy) \\
\bottomrule
\end{tabular}
\end{table}

\textbf{Parameter sensitivity.} Because the ODE parameters are assigned by biological role rather than fit to data, the sensitivity analysis characterizes the \emph{robustness of the tautology}, not the validity of the underlying assumptions. We performed univariate sensitivity analysis, varying each parameter $\pm$25\% from its nominal value. The qualitative ordering (tri-axial $<$ MHS $<$ single $<$ untreated) was preserved across all parameter perturbations.

\subsection{S6. Benchmark methodology}

The 43 multi-target ($\geq$2 gene) gold-standard combinations were independently curated from clinical sources without reference to ALIN outputs. Each entry requires at least two distinct gene targets to ensure combination-level evaluation.

\textbf{Match criteria.} The primary metric is \emph{any-overlap recall}: a predicted set $T$ matches gold-standard entry $G$ if $|G \cap T| \geq 1$. A stricter \emph{pair-overlap recall} requires $|G \cap T| \geq 2$. Gene equivalences (MAP2K1$\leftrightarrow$MAP2K2, CDK4$\leftrightarrow$CDK6) are applied during matching.

%% ============================================================
%% SUPPLEMENTARY FIGURES
%% ============================================================

\section{Supplementary Figures}

\textbf{Fig.\ S1: ALIN pipeline detailed flowchart.}
End-to-end computational workflow from DepMap 24Q4 data ingestion through five survival mechanism inference methods, MHS optimization, ranked triple scoring, and multi-source validation. Full pipeline executes in $\sim$45 minutes on a single CPU (Apple M2, 16~GB RAM).

\textbf{Fig.\ S2: Co-essentiality clustering methodology.}
\textbf{(A)} Binary essentiality matrix. \textbf{(B)} Pairwise Jaccard co-essentiality. \textbf{(C)} Ward's hierarchical clustering. \textbf{(D)} Example melanoma modules.

\textbf{Fig.\ S3: Network path inference from OmniPath.}
\textbf{(A)} OmniPath directed subgraph for melanoma. \textbf{(B)} Dependency-weighted path scoring. \textbf{(C)} Top melanoma paths.

\textbf{Fig.\ S4: Benchmark rank distribution.}
\textbf{(A)} Distribution of match types. \textbf{(B)} Baseline comparison. \textbf{(C)} Unmatched entries.

\textbf{Fig.\ S5: MHS combination size distribution.}
\textbf{(A)} MHS sizes across 77 cancers. \textbf{(B)} Lineage comparison. \textbf{(C)} Correlation with cell line count.

\textbf{Fig.\ S6: Perturbation-response path coverage.}
\textbf{(A)} Perturbation-derived survival mechanisms per cancer type. \textbf{(B)} Overlap with co-essentiality modules. \textbf{(C)} Ablation results.

\textbf{Fig.\ S7: Driver mutation landscape heatmap.}
Mutation frequencies of 10 driver genes across the 20 most well-powered cancer types.

\textbf{Fig.\ S8: Jaccard vs.\ Pearson co-essentiality comparison.}
\textbf{(A)} NMI between Jaccard-binarized and Pearson-correlation Ward clusterings as a function of cluster count $k$ (3--15). \textbf{(B)} Per-cancer mean NMI. \textbf{(C)} Silhouette score comparison.

\textbf{Fig.\ S9: Lineage-stratified benchmark concordance.}
\textbf{(A)} ALIN recall stratified by lineage. \textbf{(B)} Testable entries only. \textbf{(C)} Per-cancer match level.

\textbf{Fig.\ S10: Lineage-aware vs.\ Welch cancer-specific dependency comparison.}
\textbf{(A)} Per-cancer Jaccard similarity. \textbf{(B)} Number of cancer-specific genes per method. \textbf{(C)} Benchmark concordance comparison.

\textbf{Fig.\ S11: MHS-to-triple target redistribution.}
\textbf{(A)} Overlap category distribution. \textbf{(B)} STAT3 tracking across pipeline stages. \textbf{(C)} Hub penalty comparison. \textbf{(D)} MHS cost vs.\ triple combined score scatter. \textbf{(E)} Per-cancer target changes. \textbf{(F)} Gene frequency comparison.

%% ============================================================
%% SUPPLEMENTARY TABLES
%% ============================================================

\section{Supplementary Tables}

\begin{table}[htbp]
\centering
\caption{\textbf{Table S1.} Representative MHS combinations for 15 of 77 cancer types. Full results in Supplementary Data File 1.}
\scriptsize
\setlength{\tabcolsep}{4pt}
\begin{tabular}{@{}p{2.8cm}lcp{5.8cm}r@{}}
\toprule
\textbf{Cancer Type} & \textbf{MHS Targets} & \textbf{Size} & \textbf{Druggable Targets} & \textbf{Cost} \\
\midrule
NSCLC & CCND1+CDK2+MCL1 & 3 & CDK2 (dinaciclib), MCL1 (AMG-176) & 4.15 \\
Melanoma & BRAF+CCND1+STAT3 & 3 & BRAF (vemurafenib), STAT3 (napabucasin) & 2.96 \\
Colorectal Adeno. & CDK4+CTNNB1+KRAS+STAT3 & 4 & CDK4 (palbociclib), KRAS (sotorasib), STAT3 (napabucasin) & 3.96 \\
PDAC & CCND1+KRAS & 2 & KRAS (sotorasib) & 1.94 \\
Breast (Invasive) & CDK4+PPP1R15B+STAT3 & 3 & CDK4 (palbociclib), STAT3 (napabucasin) & 3.17 \\
AML & CDK6+DNM2+STAT3 & 3 & CDK6 (palbociclib), STAT3 (napabucasin) & 2.89 \\
Diffuse Glioma & CDK6+CHMP4B+STAT3 & 3 & CDK6 (palbociclib), STAT3 (napabucasin) & 3.34 \\
Ewing Sarcoma & CDK4+FLI1+STAT3 & 3 & CDK4 (palbociclib), STAT3 (napabucasin) & 2.71 \\
Pleural Mesothel. & CCND1+FGFR1+MDM2+STAT3 & 4 & FGFR1 (erdafitinib), STAT3 (napabucasin) & 4.57 \\
Chondrosarcoma & MCL1 & 1 & MCL1 (AMG-176) & 0.96 \\
MPN & ABL1+CDK4+STAT3 & 3 & CDK4 (palbociclib), STAT3 (napabucasin) & 3.10 \\
Retinoblastoma & OTX2+STAT3 & 2 & STAT3 (napabucasin) & 2.03 \\
HCC & GRB2+STAT3 & 2 & STAT3 (napabucasin) & 2.28 \\
Prostate Adeno. & CDK4+STAT3 & 2 & CDK4 (palbociclib), STAT3 (napabucasin) & 1.68 \\
Cervical (Mixed) & ERBB2 & 1 & ERBB2 (trastuzumab) & 0.90 \\
\bottomrule
\end{tabular}
\normalsize
\end{table}

\begin{table}[htbp]
\centering
\caption{\textbf{Table S2.} Gold-standard benchmark: 43 independently curated multi-target clinically validated combinations spanning 25 cancer types. Gene equivalences applied during matching.}
\footnotesize
\begin{tabular}{@{}p{2.5cm}p{2.7cm}lp{3.8cm}p{2.4cm}@{}}
\toprule
\textbf{Cancer} & \textbf{Gold Targets} & \textbf{Evidence} & \textbf{Ranked Triple} & \textbf{Match} \\
\midrule
Melanoma & BRAF+MAP2K1 & FDA & BRAF+MAP2K1+STAT3 & Yes (superset) \\
Melanoma & BRAF+MAP2K2 & FDA & BRAF+MAP2K1+STAT3 & Yes (superset) \\
NSCLC & EGFR+MET & Breakthrough & \textemdash & No \\
NSCLC & ALK & FDA & \textemdash & No \\
Lung Neuroendocrine & KRAS & FDA & KRAS+MET+STAT3 & Yes (superset) \\
Breast (Invasive) & CDK4+CDK6 & FDA & CDK4+KRAS+STAT3 & Yes (superset) \\
Breast (Invasive) & CDK4+KRAS & Trial & CDK4+KRAS+STAT3 & Yes (superset) \\
Breast & ERBB2 & FDA & \textemdash & No \\
Colorectal & EGFR & FDA & \textemdash & No \\
Colorectal & BRAF+EGFR & Trial & \textemdash & No \\
Colorectal & KRAS & Trial & BRAF+KRAS+STAT3 & Yes (superset) \\
Ampullary & KRAS+STAT3 & Preclin & EGFR+KRAS+STAT3 & Yes (superset) \\
PDAC & FYN+SRC+STAT3 & Preclin & CDK6+FYN+STAT3 & Yes (pair-overlap) \\
Ampullary & KRAS & FDA & EGFR+KRAS+STAT3 & Yes (superset) \\
AML & FLT3 & FDA & \textemdash & No \\
AML & CDK6+KRAS & Trial & CDK6+KRAS+STAT3 & Yes (superset) \\
RCC & MTOR & FDA & \textemdash & No \\
RCC & MTOR+VEGFR2 & Trial & \textemdash & No \\
HNSCC & EGFR+MET & Trial & CDK6+EGFR+MET & Yes (superset) \\
Diffuse Glioma & CDK4+CDK6 & Trial & CDK6+FYN+STAT3 & Yes (superset) \\
Synovial Sarcoma & CDK4+CDK6 & Trial & CDK6+FYN+STAT3 & Yes (superset) \\
Ampullary & EGFR & Trial & EGFR+KRAS+STAT3 & Yes (superset) \\
HCC & EGFR+MET & Trial & \textemdash & No \\
\bottomrule
\end{tabular}
\normalsize
\end{table}

\begin{table}[htbp]
\centering
\caption{\textbf{Table S3.} Priority MHS combinations for preclinical validation.}
\scriptsize
\setlength{\tabcolsep}{4pt}
\begin{tabular}{@{}p{2.2cm}lrp{6.8cm}@{}}
\toprule
\textbf{Cancer Type} & \textbf{MHS Targets} & \textbf{Cost} & \textbf{Rationale} \\
\midrule
Chondrosarcoma & MCL1 & 0.96 & Simplest MHS; single target; chemo-resistant tumor type \\
Ewing Sarcoma & CDK4+FLI1+STAT3 & 2.71 & Includes cancer-defining EWS-FLI1 fusion target \\
Melanoma & BRAF+CCND1+STAT3 & 2.96 & Extends BRAF inhibition; 68.7\% BRAF V600E \\
PDAC & CCND1+KRAS & 1.94 & 2-target MHS; add STAT3 as third axis per Liaki et al. \\
Retinoblastoma & OTX2+STAT3 & 2.03 & Cancer-specific OTX2; rare pediatric tumor \\
\bottomrule
\end{tabular}
\normalsize
\end{table}

\begin{table}[htbp]
\centering
\caption{\textbf{Table S4.} MHS-to-triple target redistribution across 17 gold-standard cancer types. 94.1\% of cancer types show fully disjoint MHS and triple targets.}
\label{tab:s4_mhs_triple_supp}
\resizebox{\textwidth}{!}{%
\begin{tabular}{lcrlllll}
\toprule
\textbf{Cancer Type} & \textbf{MHS Targets} & \textbf{Cost} & \textbf{Top Triple} & \textbf{Added} & \textbf{Removed} & \textbf{Overlap} & \textbf{STAT3} \\
\midrule
AML & CDK6+SOS1+SPI1+STAT3 & 4.16 & CDK4+CDK6+MCL1 & CDK4, MCL1 & SOS1, SPI1, STAT3 & partial & mhs only \\
Anapl.\ Thyroid & BRAF+STAT3 & 2.06 & CDK6+EGFR+MAP2K1 & CDK6, EGFR, MAP2K1 & BRAF, STAT3 & disjoint & mhs only \\
Bladder & CCND1+STAT3 & 2.52 & CDK2+EGFR+MAP2K1 & CDK2, EGFR, MAP2K1 & CCND1, STAT3 & disjoint & mhs only \\
Colorectal & CCND1+STAT3 & 2.52 & EGFR+KRAS+MET & EGFR, KRAS, MET & CCND1, STAT3 & disjoint & mhs only \\
Diff.\ Glioma & CCND1+STAT3 & 2.52 & CDK2+EGFR+MAP2K1 & CDK2, EGFR, MAP2K1 & CCND1, STAT3 & disjoint & mhs only \\
Endometrial & CCND1+STAT3 & 2.52 & CDK2+EGFR+MET & CDK2, EGFR, MET & CCND1, STAT3 & disjoint & mhs only \\
Esophagogastric & CCND1+STAT3 & 2.34 & CDK6+EGFR+PIK3CA & CDK6, EGFR, PIK3CA & CCND1, STAT3 & disjoint & mhs only \\
HNSCC & EGFR+STAT3 & 2.14 & CDK4+CDK6+ERBB2 & CDK4, CDK6, ERBB2 & EGFR, STAT3 & disjoint & mhs only \\
HCC & CCND1+CDK4 & 2.28 & BCL2L1+EGFR+MAP2K1 & BCL2L1, EGFR, MAP2K1 & CCND1, CDK4 & disjoint & neither \\
Breast & CCND1+STAT3 & 2.30 & BCL2L1+EGFR+MAP2K1 & BCL2L1, EGFR, MAP2K1 & CCND1, STAT3 & disjoint & mhs only \\
Liposarcoma & CCND1+CDK4 & 1.84 & EGFR+FGFR1+MET & EGFR, FGFR1, MET & CCND1, CDK4 & disjoint & neither \\
Melanoma & CCND1+CDK4 & 2.13 & CDK6+EGFR+MAP2K1 & CDK6, EGFR, MAP2K1 & CCND1, CDK4 & disjoint & neither \\
NSCLC & CCND1+STAT3 & 2.46 & CDK6+EGFR+MAP2K1 & CDK6, EGFR, MAP2K1 & CCND1, STAT3 & disjoint & mhs only \\
Ovarian & CCND1+CDK4+STAT3 & 3.59 & CDK6+EGFR+MET & CDK6, EGFR, MET & CCND1, CDK4, STAT3 & disjoint & mhs only \\
PDAC & CCND1+CDK6 & 2.24 & FYN+KRAS+STAT3 & FYN, KRAS, STAT3 & CCND1, CDK6 & disjoint & triple only \\
Prostate & CCND1+CDK4 & 1.54 & CDK2+EGFR+MAP2K1 & CDK2, EGFR, MAP2K1 & CCND1, CDK4 & disjoint & neither \\
Renal & CCND1+ZNF316 & 2.32 & CDK6+EGFR+MAP2K1 & CDK6, EGFR, MAP2K1 & CCND1, ZNF316 & disjoint & neither \\
\bottomrule
\end{tabular}}
\end{table}

\begin{table}[htbp]
\centering
\caption{\textbf{Table S5.} Known failed drug combinations evaluated against ALIN predictions.}
\label{tab:s5_failed_combos_supp}
\scriptsize
\setlength{\tabcolsep}{3pt}
\begin{tabular}{@{}p{1.8cm}p{2.2cm}p{1.5cm}p{2.4cm}lp{3.8cm}@{}}
\toprule
\textbf{Cancer} & \textbf{Failed Combo} & \textbf{Trial} & \textbf{ALIN Prediction} & \textbf{Verdict} & \textbf{Outcome / Failure Reason} \\
\midrule
RCC & EGFR (erlotinib) & Phase~2 & EGFR+MET & Predicts & ORR $<$5\%; EGFR not a RCC driver \\
Prostate & EGFR (erlotinib) & Phase~2 & CDK2+EGFR+MAP2K1 & Predicts & No responses in CRPC; EGFR non-driver \\
AML & MCL1 (AMG-176) & Phase~1 & CDK4+CDK6+MCL1 & Predicts & Dose-limiting cardiotoxicity \\
Melanoma & BRAF mono (no MEK) & Phase~3 & BRAF+EGFR (combo) & Predicts & PFS 5--8~mo; paradoxical MAPK reactivation \\
\midrule
Melanoma & BRAF+MEK+CDK4/6 & Phase~1/2 & CDK6+EGFR+MAP2K1 & Partial & ORR 52\% $<$ 64\% doublet \\
Melanoma & CDK4/6+BRAF & Phase~1/2 & BRAF+EGFR (combo) & Partial & Excess haematotoxicity \\
\midrule
CRC & STAT3 (napabucasin) & Phase~3 & BRAF+EGFR+KRAS & Avoids & Failed primary OS endpoint \\
Ovarian & PIK3CA+MAP2K1 & Phase~1/2 & EGFR+MET & Avoids & ORR 4.7\% \\
\bottomrule
\end{tabular}
\normalsize
\end{table}

\begin{table}[htbp]
\centering
\caption{\textbf{Table S7.} Translational feasibility scoring for all 17 benchmark cancer predictions.}
\label{tab:s7_feasibility_supp}
\scriptsize
\setlength{\tabcolsep}{3pt}
\begin{tabular}{@{}p{2cm}p{2.5cm}p{2.6cm}cccccp{3.3cm}@{}}
\toprule
\textbf{Cancer} & \textbf{Triple} & \textbf{Drug mapping} & \textbf{Avail} & \textbf{Select} & \textbf{ToxF} & \textbf{Prec} & $F$ & \textbf{Red flags} \\
\midrule
Pancreatic & BRAF+KRAS+STAT3 & vem(A)+sot(A)+nap(P2) & 0.85 & 1.00 & 0.85 & 0.10 & 0.74 & OOT-GI: KRAS+STAT3 \\
Colorectal & BRAF+EGFR+KRAS & vem(A)+erl(A)+sot(A) & 0.97 & 1.00 & 0.90 & 1.00 & 0.95 & OOT-derm: EGFR+BRAF \\
NSCLC & CDK6+EGFR+MAP2K1 & pal(A)+erl(A)+tra(A) & 1.00 & 1.00 & 0.95 & 1.00 & 0.99 & OOT-derm: EGFR+MAP2K1 \\
ATC & CDK6+EGFR+MAP2K1 & pal(A)+erl(A)+tra(A) & 1.00 & 1.00 & 0.95 & 0.50 & 0.89 & OOT-derm: EGFR+MAP2K1 \\
Melanoma & CDK6+EGFR+MAP2K1 & pal(A)+erl(A)+tra(A) & 1.00 & 1.00 & 0.95 & 0.50 & 0.89 & OOT-derm: EGFR+MAP2K1 \\
Breast & BCL2L1+EGFR+MAP2K1 & nav(P2)+erl(A)+tra(A) & 0.85 & 0.88 & 0.95 & 0.50 & 0.79 & OOT-derm: EGFR+MAP2K1 \\
RCC & CDK6+EGFR+MAP2K1 & pal(A)+erl(A)+tra(A) & 1.00 & 1.00 & 0.95 & 0.00 & 0.79 & OOT-derm: EGFR+MAP2K1 \\
Ovarian & CDK6+EGFR+MET & pal(A)+erl(A)+cap(A) & 1.00 & 1.00 & 0.90 & 0.00 & 0.76 & OOT-GI: EGFR+MET \\
Liposarcoma & EGFR+FGFR1+MET & erl(A)+erd(A)+cap(A) & 1.00 & 1.00 & 0.90 & 0.00 & 0.75 & OOT-GI: EGFR+MET \\
Glioma & EGFR+FGFR1+MET & erl(A)+erd(A)+cap(A) & 1.00 & 1.00 & 0.90 & 0.00 & 0.75 & OOT-GI: EGFR+MET \\
Bladder & CDK2+EGFR+MAP2K1 & din(P2)+erl(A)+tra(A) & 0.87 & 0.80 & 0.95 & 0.30 & 0.75 & CDK2 narrow TW \\
Esophago. & CDK6+EGFR+PIK3CA & pal(A)+erl(A)+alp(A) & 1.00 & 1.00 & 0.65 & 0.00 & 0.73 & DDI(mod): erl+alp \\
HCC & EGFR+FGFR1+MET & erl(A)+erd(A)+cap(A) & 1.00 & 1.00 & 0.90 & 0.50 & 0.95 & OOT-GI: EGFR+MET \\
Prostate & CDK2+EGFR+MAP2K1 & din(P2)+erl(A)+tra(A) & 0.87 & 0.80 & 0.95 & 0.00 & 0.69 & CDK2 narrow TW \\
Endometrial & CDK2+EGFR+MET & din(P2)+erl(A)+cap(A) & 0.87 & 0.88 & 0.90 & 0.00 & 0.65 & CDK2 narrow TW \\
HNSCC & CDK4+CDK6+ERBB2 & pal(A)+pal(A)+tras(A) & 0.93 & 0.75 & 0.75 & 0.00 & 0.68 & OOT-myelo: CDK4+CDK6 \\
AML & CDK4+CDK6+MCL1 & pal(A)+pal(A)+AMG(P1) & 0.73 & 0.63 & 0.75 & 0.00 & 0.58 & MCL1 cardiotox \\
\bottomrule
\end{tabular}
\normalsize
\end{table}

\end{document}
