% Prefer the fallback `article` class unless an explicit marker enables the
% vendor class. To force the vendor class create an empty file named
% `USE_VENDOR_CLASS` in the project root. This prevents class-provided package
% load-order issues (e.g., cleveref vs hyperref) during automatic builds.
\IfFileExists{zHenriquesLab-StyleBioRxiv.cls}{%
  \IfFileExists{USE_VENDOR_CLASS}{%
    \documentclass[times, twoside, watermark]{zHenriquesLab-StyleBioRxiv}
  }{%
    \documentclass[11pt]{article}
    \usepackage[margin=1in]{geometry}
    % Provide compatibility shims when the vendor class is absent (Overleaf/fallback)
    \usepackage{authblk}
    \providecommand{\leadauthor}[1]{}
    \providecommand{\shorttitle}[1]{}
    \providecommand{\corrauthor}[1]{}
    \providecommand{\Letter}{}
    \renewcommand\Affilfont{\small}
  }
}{%
  \documentclass[11pt]{article}
  \usepackage[margin=1in]{geometry}
  % Provide compatibility shims when the vendor class is absent (Overleaf/fallback)
  \usepackage{authblk}
  \providecommand{\leadauthor}[1]{}
  \providecommand{\shorttitle}[1]{}
  \providecommand{\corrauthor}[1]{}
  \providecommand{\Letter}{}
  \renewcommand\Affilfont{\small}
}

% Core packages
\usepackage{graphicx}
\usepackage{booktabs}
\usepackage{multirow}
\usepackage{amsmath}
\usepackage{amssymb}
\usepackage{microtype}
\usepackage{needspace}
\usepackage[section]{placeins}
\usepackage{caption}
\usepackage{multicol}
\usepackage[colorlinks=false,pdfborder={0 0 0},breaklinks=true]{hyperref}
\usepackage{cleveref}
\usepackage{longtable}
\usepackage{array}

% Compatibility shims for environments defined only by the vendor class
\makeatletter
\@ifundefined{keywords}{%
  \newenvironment{keywords}{\noindent\textbf{Keywords:} }{\par}%
}{}
\@ifundefined{endacknowledgements}{%
  \newenvironment{acknowledgements}{\section*{Acknowledgements}}{}
}{}
\makeatother

% Note: if you need journal-specific styling, provide the class file locally
% (ignored by git) or switch to the journal template during submission.

\leadauthor{Erzurumluoğlu}

\begin{document}

\title{ALIN Framework (Adaptive Lethal Intersection Network): Computational Hypothesis Generation for Tri-Axial Combination Therapy Across Cancer Types}
\shorttitle{ALIN Framework}

% Use a single, robust author block compatible with both the vendor class
% and the fallback article+authblk combination. The corresponding author
% email is included via \thanks{} to avoid class-specific macros.
\author{Roy Erzurumluoğlu\thanks{Corresponding author: \href{mailto:roy.erzurumluoglu@gmail.com}{roy.erzurumluoglu@gmail.com}}$^{,}$\thanks{ORCID: \href{https://orcid.org/0009-0000-9523-881X}{0009-0000-9523-881X}}\\
  Maastricht University, Bachelor of Natural Sciences, Maastricht, The Netherlands. Current affiliation: none.}

\maketitle

\begin{abstract}
Cancer drug resistance arises when tumors shift signaling through compensatory pathways. Liaki et al.~\cite{Liaki2025} demonstrated that simultaneously targeting three independent signaling axes---downstream, upstream, and orthogonal---prevented resistance in pancreatic cancer for $>$200 days. We present ALIN (Adaptive Lethal Intersection Network), a computational hypothesis-generation pipeline that extends this tri-axial principle across cancer types. ALIN infers cancer-specific survival mechanisms from DepMap CRISPR data and OmniPath signaling networks, identifies minimal hitting sets (MHS) covering all inferred mechanisms, and generates ranked triple combinations scored for predicted synergy, resistance risk, toxicity, and druggability.

Across 96 DepMap cancer types, ALIN produces candidate predictions for 77 cancers. Target concordance with 43 independently curated multi-target gold-standard combinations (25 cancer types) yields 44.2\% any-overlap, 30.2\% pair-overlap, and 7.0\% exact concordance (3/43; $p < 0.001$ vs.\ random-candidate null, permutation test with 1{,}000 random-draw iterations across 17 cancer types). On 25 testable entries, concordance rises to 64.0\% any-overlap and 44.0\% pair-overlap. A recurrent pattern consistent with tri-axial architecture is observed: STAT3 as a candidate orthogonal node (78\% of cancers), cell cycle regulators as downstream effectors, and cancer-specific oncogenes as upstream drivers---though STAT3's dominance partly reflects algorithmic hub preference, which a hub-gene penalty corrects in ranked triples. All predictions are computational hypotheses requiring experimental validation; target concordance measures partial overlap with historical drug targets, not clinical efficacy.
\end{abstract}

\begin{keywords}
cancer | drug resistance | tri-axial inhibition | combination therapy | DepMap | STAT3 | pathway shifting | precision oncology
\end{keywords}

\noindent\textbf{Corresponding author:} \href{mailto:roy.erzurumluoglu@gmail.com}{roy.erzurumluoglu@gmail.com}

\newpage
\twocolumn
\raggedbottom
\sloppy

% Aggressive float placement — reduce white gaps around figure*/table*
\renewcommand{\dbltopfraction}{0.9}      % up to 90% of page for top floats
\renewcommand{\textfraction}{0.07}       % min 7% text (default 20%)
\renewcommand{\dblfloatpagefraction}{0.7}% float-only pages need 70% fill
\setcounter{dbltopnumber}{2}             % allow 2 double floats per page top
\setcounter{topnumber}{3}
\setcounter{totalnumber}{5}

%TC:break main

\section*{Introduction}

Single-agent cancer therapies fail because tumors adapt: when one signaling pathway is inhibited, compensatory pathways sustain survival. This \emph{pathway shifting} is the fundamental mechanism of acquired drug resistance~\cite{Bergholz2021,Xin2024}. Combination therapy addresses this by simultaneously blocking multiple survival axes, but the vast combinatorial space makes rational design essential.

Liaki et al.~\cite{Liaki2025} demonstrated a solution in pancreatic ductal adenocarcinoma (PDAC): tumor cells maintain viability through three independent signaling nodes---\textbf{downstream} (RAF1), \textbf{upstream} (EGFR), and \textbf{orthogonal} (STAT3) to KRAS signaling. Ablating two nodes triggers compensatory activation of the third (e.g., RAF1+EGFR ablation activates STAT3 via FYN~\cite{Chougule2016,Zhang2020FYN}). Only simultaneous targeting of all three axes induced complete tumor regression with no resistance for $>$200 days; dual combinations failed~\cite{Liaki2025}.

This tri-axial inhibition principle provides a framework for combination therapy design, but generalizing it beyond PDAC requires answering a cancer-specific question: \emph{which genes occupy the downstream, upstream, and orthogonal positions for each cancer type?}

We present ALIN (Adaptive Lethal Intersection Network), a computational pipeline that addresses this question (\cref{fig:pipeline}). ALIN infers cancer-specific survival mechanisms from DepMap~\cite{Dempster2021} and OmniPath~\cite{Turei2026} data, identifies target combinations covering every axis via \textbf{minimal hitting set (MHS)} optimization~\cite{VeraLicona2013,Murakami2017}, and generates ranked triple combinations scored for estimated synergy, resistance risk, and toxicity. Our contributions: (1)~computational extension of the tri-axial hypothesis to 77 cancer types as a hypothesis-generation tool; (2)~identification of a recurrent target architecture consistent with tri-axial organization, with a hub-gene penalty addressing STAT3's algorithmic over-representation; (3)~target-concordance evaluation against 43 independently curated gold-standard combinations with four baselines and circularity ablation; (4)~multi-source pharmacological evidence integration. We distinguish \textbf{algorithmic triads} (three nodes that collectively cover many inferred survival mechanisms) from \textbf{mechanistic tri-axiality} (three functionally orthogonal axes whose simultaneous inhibition prevents adaptive rewiring), claiming only the former.

\section*{Methods}

\subsection*{Data sources}
We use DepMap~\cite{Dempster2021,Tsherniak2017} release 24Q4 (accessed January 2026) for gene dependency $D_{g,\ell}$ (Chronos score~\cite{Dempster2021} from CRISPRGeneEffect.csv), cell line metadata (Model.csv), and cancer type mapping via OncoTree (OncotreePrimaryDisease). OmniPath~\cite{Turei2026} provides the directed signaling network $G = (V, E)$ (accessed January 2026; cached for reproducibility). PRISM~\cite{Corsello2020} and GDSC~\cite{Yang2013,Iorio2016} supply drug sensitivity data for validation. All data versions and access dates are documented in VERSION\_INFO.md (Supplementary Methods, Section~S1).

\subsection*{Survival mechanism inference}
We infer cancer-specific survival mechanisms---heterogeneous objects representing independent routes by which tumor cells maintain viability---through four complementary approaches (mathematical definitions in Supplementary Methods~S2). We use ``survival mechanism'' rather than ``path'' as a collective term, since these objects have different semantics; the term ``path'' is reserved for directed OmniPath network paths below.

\textbf{1. Essentiality and selectivity.} A gene is essential if its Chronos score $< -0.5$ and selective if essential in $\geq$30\% of cell lines for that cancer type~\cite{Dempster2021,Behan2019}. Pan-essential genes ($>$90\% of all cell lines) are excluded. A threshold sensitivity sweep (500 configurations, 5 cancers) confirmed robustness (see Sensitivity).

\textbf{2. Co-essentiality modules.} Pairwise Jaccard co-essentiality on binarized dependency profiles (essential/non-essential per cell line), followed by Ward's hierarchical clustering, groups genes into 3--15 survival modules~\cite{Pan2018}. Jaccard is used because the relevant structure is shared essentiality membership rather than continuous Chronos score magnitude; however, this binarization discards information about dependency strength. To quantify how much information is lost, we compared Jaccard-binarized vs.\ Pearson-correlation distance matrices across 10 cancer types ($\geq$74 cell lines each), clustering both with Ward's method for $k = 3$--$15$ and computing normalized mutual information (NMI) between the resulting partitions. Agreement was low (mean NMI = $0.20 \pm 0.07$; Supplementary Methods~S2, Supplementary Figure~S8), confirming that Jaccard and Pearson capture complementary co-essentiality structure. Jaccard clusterings yielded consistently higher silhouette scores (mean 0.054 vs.\ 0.025), supporting their use for discrete module recovery. Clusters are not formally enrichment-tested against pathway databases (see Limitations).

\textbf{3. Signaling paths (OmniPath).} Directed paths from driver genes to essential effectors are extracted from OmniPath~\cite{Turei2026} ($\leq$4 hops; sensitivity to this cutoff in Supplementary Methods~S2), scored by mean dependency, and pruned if weakly essential (mean Chronos $> -0.3$). OmniPath is literature-curated and has known degree biases toward well-studied proteins (STAT3, TP53, AKT); the hub-gene penalty (Equation~3) partially addresses this, but network priors may still drive results more than essentiality data for high-degree nodes.

\textbf{4. Cancer-specific statistical dependencies.} The default method identifies genes significantly more essential in the target cancer via Welch $t$-test (cancer-type cell lines vs.\ all other lines, BH-corrected $q < 0.05$, Cohen's $d > 0.3$~\cite{Benjamini1995,Cohen1988}). Because this confounds lineage-specific and cancer-specific effects, we also implement a \textbf{lineage-aware alternative}: for each gene, we fit an OLS model $D_g \sim \text{lineage} + \text{is\_target\_cancer}$ across all cell lines with lineage annotations from OncotreeLineage (34~lineages), where lineage is encoded as dummy variables and the coefficient on $\text{is\_target\_cancer}$ captures cancer-type-specific essentiality after controlling for shared lineage dependencies. Genes with BH-corrected $q < 0.05$ and $|\beta_{\text{cancer}}| > 0.3$ form the cancer-specific survival mechanism.

The lineage-aware model identifies substantially different gene sets (median Jaccard~$= 0.051$, mean~$= 0.118$ across cancer types), yet final triple predictions change for only 2/17 gold-standard cancer types (11.8\%), with benchmark concordance increasing modestly from 46.5\% to 48.8\% any-overlap. The two changes are biologically notable: Colorectal (EGFR+KRAS+MET $\to$ BRAF+EGFR+KRAS, replacing MET with the established CRC target BRAF) and Breast (BCL2L1+EGFR+MAP2K1 $\to$ CDK4+ERBB2+PIK3CA, matching the clinical standard of care). These results indicate that the MHS + scoring stages are robust to gene-level input perturbation, while the lineage-controlled model better isolates cancer-specific vulnerabilities for the cancer types where lineage effects are strongest (Supplementary Methods~S2; Supplementary Figure~S10).

\textbf{5. Perturbation-response paths.} Manually curated summaries from $\sim$15 kinase-inhibitor studies identify feedback and bypass genes (e.g., EGFR reactivation upon KRAS inhibition). Combinations targeting these feedback genes receive a perturbation bonus $\beta_{\text{pert}}$. This does not constitute systematic perturbation profiling (see Limitations).

\subsection*{Minimal hitting set (MHS) optimization}
We frame target selection as a \textbf{minimal hitting set} problem~\cite{VeraLicona2013,Murakami2017}: find a minimal gene set $T$ intersecting every inferred survival mechanism. \textbf{Important framing:} MHS coverage is a topological property of the inferred mechanism graph, not a guarantee of therapeutic efficacy. ``Hitting'' a mechanism by containing a gene on its path does not ensure pharmacological inhibition, nor that inhibition causes the intended shutdown. Resistance is often driven by phenotypic plasticity, tumor microenvironment, pharmacodynamics, and subclonal selection---not only network-topological path redundancy. We therefore position MHS as a \emph{hypothesis generator} for candidate targets; ranked triple combinations (below) serve as the primary therapeutic recommendation, since static mechanism coverage may underestimate the targets needed to prevent dynamic pathway shifting~\cite{Liaki2025}. The core constraint is:
\begin{equation}
\forall p \in P: \quad T \cap \mathcal{N}(p) \neq \emptyset
\end{equation}

We minimize a weighted cost penalizing toxic and pan-essential genes while rewarding specificity and druggability:
\begin{equation}
\resizebox{\columnwidth}{!}{$\displaystyle
c(g, c) = \underbrace{\tau(g)}_{\text{tox}} - 0.5\underbrace{s(g,c)}_{\text{spec}} - 0.3\underbrace{d(g)}_{\text{drug}} + 2\underbrace{\mathbb{1}_{\text{pan}}}_{\text{pen}} + 1
$}
\end{equation}
Component definitions are provided in Supplementary Methods~S3. We employ a solver hierarchy: greedy weighted set cover ($\ln n$-approximation), ILP via \texttt{scipy.optimize.milp} for pools $\leq$500 genes, and exhaustive enumeration for pools $\leq$20 genes. The combination size (1--4 targets in practice) emerges from viability network complexity.

\begin{figure*}[!htb]
\centering
\includegraphics[width=0.95\linewidth]{figures/fig1_pipeline_schematic}
\vspace{-6pt}
\caption{\textbf{ALIN Framework overview.} \textbf{(A)} Computational pipeline: DepMap CRISPR essentiality data and OmniPath signaling networks are integrated to infer cancer-specific survival mechanisms. Minimal hitting sets (MHS) identify the smallest target set covering all inferred mechanisms; ranked triples are scored for estimated synergy, resistance risk, combo-toxicity, and pathway coverage. Multi-source evidence integration (PubMed, STRING, ClinicalTrials.gov, PRISM) provides orthogonal support. \textbf{(B)} Tri-axial inhibition principle (Liaki et al., 2025): tumors maintain viability through three independent signaling axes. Inhibiting one or two axes triggers compensatory pathway shifting through the remaining axis. Only simultaneous tri-axial blockade prevented resistance in PDAC; ALIN tests whether this principle generalizes computationally.}
\label{fig:pipeline}
\end{figure*}

\subsection*{Ranked triple combination scoring}
We enumerate triples from the candidate pool and score each by five criteria: \textbf{path coverage} (fraction of survival mechanisms hit; minimum 70\%), \textbf{estimated synergy} (blending co-essentiality correlation with pathway diversity; this estimates pathway coupling, not pharmacological synergy \textit{sensu stricto}), \textbf{resistance risk} (uncovered bypass mechanisms~\cite{Bergholz2021}), \textbf{combination toxicity} (drug-drug interaction severity from DrugBank, overlapping toxicity classes, and FAERS signal counts; see Supplementary Methods~S4 for operational definitions of each sub-score), and \textbf{druggability} (count of targets with approved drugs). These combine into:
\begin{equation}
\resizebox{\columnwidth}{!}{$\displaystyle
\boxed{\begin{aligned}
S_{\text{comb}} &= 0.22\!\cdot\!\text{cost} + 0.18\!\cdot\!(1\!-\!\text{syn}) + 0.18\!\cdot\!\text{res} \\
&+ 0.18\!\cdot\!\text{tox} + 0.14\!\cdot\!(1\!-\!\text{cov}) \\
&- 0.10\!\cdot\!n_{\text{drugs}} - \beta_{\text{pert}} + \text{hub}_{\text{pen}}
\end{aligned}}
$}
\end{equation}
Lower scores indicate better combinations. The \textbf{hub-gene penalty} corrects for algorithmic over-representation of high-connectivity genes (particularly STAT3): genes whose survival-mechanism frequency exceeds the candidate median receive a proportional penalty of $1.5 \times (f_g - \tilde{f})$. Weights were set by domain-informed heuristic and are robust to $\pm$20\% perturbation ($<$15\% of cancer types change top-5 rankings). Detailed scoring formulas with input/output ranges for each sub-score are in Supplementary Methods~S4.

\subsection*{Sensitivity and robustness}
MHS compositions for major cancers remained stable across tested parameter ranges ($\theta \in \{0.2, 0.3, 0.4\}$, $\tau_{\text{dep}} \in \{-0.4, -0.5, -0.6\}$). A systematic threshold sweep (500 configurations, 5 cancers) showed candidate pool size varying $\sim$23-fold, with the dependency threshold exerting the largest effect. Co-essentiality clustering calibrated against KEGG/Reactome annotations yielded NMI=0.58 at the default $k$=15 (see Limitations). LOCO partitioned evaluation yields exact recall of 12.5\% $\pm$ 33.1\% (note: this tests consistency of predictions when one cancer type is held out, not true generalization via held-out data; see Limitations). Per-cancer sample sizes range from 1 to 165 cell lines; predictions from low-$n$ cancers ($<$10 cell lines) should be treated with additional caution.

\subsection*{Validation and benchmarking}
We curated 43 independently sourced multi-target ($\geq$2 gene) clinically validated combinations spanning 25 cancer types~\cite{Long2014,Planchard2016,Kopetz2019,Larkin2014,Park2021,Sequist2020,Baselga2012,Turner2015,Motzer2015,Daver2022,Ciruelos2024,Bhardwaj2019,Liaki2025}. Of these, 18 are structurally unmatchable by any CRISPR-based pipeline (absent cancer types or non-CRISPR targets such as ESR1, KDR, PSMB5); the criteria for ``unmatchable'' were defined \textit{a priori} as: (i)~cancer type absent from DepMap OncoTree mapping, or (ii)~all gold-standard targets absent from the CRISPR gene panel. We report concordance on both the full 43-entry set and the 25 testable entries. The primary metric is \textbf{any-overlap target concordance} ($|G \cap T| \geq 1$)~\cite{Julkunen2023}---measuring partial overlap between predicted gene targets and historical drug targets, not clinical efficacy or synergy. Secondary metrics include pair-overlap ($\geq$2 genes), superset ($G \subseteq T$), exact ($G = T$), and cancer-level precision (fraction of evaluable cancer types with $\geq$1 gold-standard target recovered). Four baselines: (1)~random-global (3 genes from $\sim$18,400; 1{,}000 permutations); (2)~random-candidate (3 genes from the per-cancer selective pool; correct null; 1{,}000 permutations; $p$-values by permutation test: for each cancer, gold-standard targets are shuffled among the candidate pool; 1{,}000 iterations yield an empirical null distribution, avoiding the independence assumption of a binomial test); (3)~global-frequency (top-3 most frequent pipeline genes); (4)~driver-gene (per-cancer top-3 from TCGA/COSMIC/OncoKB).

\subsection*{Pharmacological validation and evidence tiering}
To address the CRISPR--drug gap~\cite{Lin2017,Goncalves2020}: (1)~a \emph{concordance filter} correlates DepMap CRISPR scores with PRISM drug sensitivity across matched cell lines, flagging discordant targets; (2)~a \emph{co-essentiality-based synergy estimator} replaces the original heuristic, blending Jaccard similarity (60\%) with pathway diversity (40\%)~\cite{Norman2019,Shen2017combCRISPR}; (3)~an \emph{evidence tier system} classifies predictions from Tier~1 (experimental tri-axial validation) through Tier~4 (computational only).

\subsection*{Permutation test}
We assess benchmark significance via a permutation test rather than a binomial test (which assumes independence across cancer types---violated because cancers share gene pools and DepMap cell lines). The protocol is:
\begin{enumerate}
\item \textbf{Candidate pool (fixed).} For each cancer type $c$, define $\mathcal{P}_c$ as the set of selective-essential genes (Chronos $< -0.5$ in $\geq$30\% of cell lines for $c$, excluding pan-essentials). Pool sizes range from 247 to 3{,}891 genes (median 1{,}717; $n = 17$ evaluable cancer types).
\item \textbf{Gold standard (fixed).} The 43-entry curated benchmark set $\{G_c\}$ and all match criteria are held constant across all iterations; only predictions are randomized.
\item \textbf{Null sampling (randomized).} In each of $B = 1{,}000$ iterations, for every evaluable cancer type $c$ simultaneously, draw 3~genes uniformly at random \emph{without replacement} from $\mathcal{P}_c$ to form a null triple $T_c^{(b)}$, replacing the pipeline's predicted triple while preserving the cancer--pool mapping. Gold-standard target labels are never permuted.
\item \textbf{Statistic (computed).} Evaluate each null prediction set $\{T_c^{(b)}\}_{c=1}^{17}$ against the full 43-entry gold standard using the same match criteria (any-overlap, pair-overlap, exact) and gene-alias equivalences as the observed evaluation. The test statistic $S_b$ is the \emph{pooled} concordance rate across all cancer types (not per-cancer).
\item \textbf{$p$-value.} The empirical $p$-value is $p = |\{b : S_b \geq S_{\text{obs}}\}| / B$, the fraction of null iterations achieving concordance $\geq$ observed.
\end{enumerate}
Because the test statistic is a single pooled value across all 17 cancer types per iteration, no per-cancer multiple-testing correction is required; the permutation directly generates the joint null distribution over the cancer panel. Critically, the candidate pool $\mathcal{P}_c$ is the \emph{same} pool from which the pipeline selects targets, so the null tests whether scoring---not DepMap pre-filtering---drives performance.

\subsection*{Degree-preserving network null}
To test whether STAT3's MHS recurrence reflects biological signal or OmniPath hub degree, we perform a degree-preserving edge-swap randomization:
\begin{enumerate}
\item \textbf{Edge swaps.} Given the OmniPath directed graph $G = (V, E)$, for each of $N = 20$ permutations, perform $10 \times |E|$ swap attempts: select two edges $(u \to v, x \to y)$ uniformly at random and propose swapping to $(u \to y, x \to v)$, accepting only if no self-loops or multi-edges are created. This preserves the in-degree and out-degree of every node.
\item \textbf{Re-run pipeline.} For each shuffled graph $G'$, re-run the full MHS inference pipeline on 5 test cancers (NSCLC, Melanoma, CRC, PDAC, Breast) and record which genes appear in MHS solutions.
\item \textbf{Null distribution.} Compute STAT3 frequency across the 5 test cancers under each permutation, yielding a null distribution of $N$ values.
\item \textbf{$p$-value.} One-sided permutation $p = |\{i : f_i^{\text{null}} \geq f^{\text{obs}}\}| / N$.
\end{enumerate}

\subsection*{Systems biology simulation}
An ODE-based model simulates pathway shifting dynamics under different treatment strategies. Parameters are assigned by biological role (not fit to data), so results are deductive consequences of the tri-axial hypothesis, not independent predictions. Details in Supplementary Methods~S5.

\section*{Results}

\subsection*{Pan-cancer MHS discovery}
We analyzed \textbf{96 cancer types} from DepMap (OncoTree primary disease mapping), spanning solid tumors, hematologic malignancies, and rare cancers; \textbf{77 cancer types} yielded MHS combinations (19 had insufficient cell lines or no inferred survival mechanisms). \cref{tab:summary} summarizes key metrics. A central finding is that the optimal MHS size varies by cancer (\cref{fig:mhs_dist}): 11 cancers require only \textbf{1 target} (e.g., Chondrosarcoma: MCL1; Mucosal Melanoma: CCND1), 50 cancers require \textbf{2 targets}, 14 cancers require \textbf{3 targets}, and 2 cancers require \textbf{4 targets} (Colorectal Adenocarcinoma: CDK4+CTNNB1+KRAS+STAT3; Pleural Mesothelioma: CCND1+FGFR1+MDM2+STAT3). All 77 MHS sets achieve \textbf{100\% survival mechanism coverage} by construction: the MHS solver is defined to find a set intersecting every inferred survival mechanism, so complete coverage is a mathematical guarantee of the algorithm, not an empirical finding about biological pathway blockade (see Limitations~\#6). Component ablation (Table~\ref{tab:ablation}) quantifies each module's contribution; a permutation test and degree-preserving network null confirm significance.

Cell line counts per cancer type ranged from 1 (Ovarian Germ Cell Tumor, Glassy Cell Carcinoma) to 165 (Non-Small Cell Lung Cancer). Survival mechanisms ranged from 4 (Chondrosarcoma, Mucosal Melanoma) to 103 (NSCLC), reflecting differences in tumor complexity. Cost scores ranged from 0.90 (Mixed Cervical Carcinoma, single ERBB2 target) to 4.57 (Pleural Mesothelioma, 4-target set).

\begin{table*}[htbp]
\centering
\caption{Summary of pan-cancer MHS discovery metrics.}
\label{tab:summary}
\begin{tabular}{@{}lr@{}}
\toprule
\textbf{Metric} & \textbf{Value} \\
\midrule
Cancer types analyzed & 96 \\
Cancer types with MHS predictions & 77 \\
  \quad 1-target MHS & 11 \\
  \quad 2-target MHS & 50 \\
  \quad 3-target MHS & 14 \\
  \quad 4-target MHS & 2 \\
  Survival mechanism coverage (all) & 100\% (by construction) \\
STAT3 frequency (of 77 cancers) & 78\% (60) \\
CCND1 frequency & 23\% (18) \\
MCL1 frequency & 13\% (10) \\
CDK4 frequency & 10\% (8) \\
CDK6 frequency & 8\% (6) \\
Benchmark concordance (ranked combinations) & 44.2\% any-overlap, 30.2\% pair-overlap (19/43) \\
Benchmark cancer-level precision & 47.1\% (8/17 evaluable cancers) \\
Benchmark exact concordance & 7.0\% (3/43) \\
\bottomrule
\end{tabular}
\end{table*}

\begin{figure*}[htbp]
\centering
\includegraphics[width=0.95\linewidth]{figures/fig6_mhs_distribution}
\caption{\textbf{MHS size and target frequency.} \textbf{(A)} Distribution of optimal MHS sizes across 77 cancer types. Most cancers (50/77, 65\%) require exactly 2 targets for complete survival mechanism coverage; 11 require only 1 target, 14 require 3, and 2 require 4. \textbf{(B)} Top 10 most frequent MHS target genes, colored by assigned tri-axial role: candidate orthogonal survival hub (green), downstream cell cycle effector (blue), upstream oncogenic driver (orange). STAT3 dominates (78\% of cancers), followed by CCND1 (23\%). Tri-axial role assignments are post-hoc functional annotations, not algorithmically derived.}
\label{fig:mhs_dist}
\end{figure*}

\subsection*{Target frequency architecture}
The target frequency distribution (\cref{fig:xnode}) reveals a recurrent pattern consistent with tri-axial organization, mapping onto the Liaki et al.~\cite{Liaki2025} principle:
\begin{itemize}
\item \textbf{STAT3} (60 cancers, 78\%): The most frequent MHS target. Constitutively activated in $>$70\% of solid tumors~\cite{Yu2004,Frank2007}, STAT3 sits downstream of multiple oncogenic drivers. Its high frequency reflects both genuine pan-cancer activation and algorithmic hub preference (see Limitations).
\item \textbf{CCND1} (18 cancers, 23\%): Cyclin D1 drives G1/S transition and is targetable via CDK4/6 inhibitors~\cite{Spring2020}. Its emergence as the second most frequent MHS component suggests a pan-cancer cell cycle entry dependence.
\item \textbf{MCL1} (10 cancers), \textbf{CDK4} (8), \textbf{CDK6} (6): Anti-apoptotic and cell cycle targets with clinical-stage inhibitors.
\item \textbf{EGFR} (3), \textbf{KRAS} (2), \textbf{BRAF} (1): Rare as MHS components despite oncogenic driver roles; recovered through ranked triple scoring.
\end{itemize}

These targets map onto the tri-axial framework in a pattern consistent with STAT3 as a candidate \textbf{orthogonal} survival hub, cell cycle regulators (CCND1, CDK4/6) as \textbf{downstream} effectors, and cancer-specific oncogenes (KRAS, BRAF, EGFR) as \textbf{upstream} drivers. Whether this recurrent architecture reflects genuine functional orthogonality or is partly driven by network degree biases and CRISPR assay properties requires experimental validation (see Limitations).

\begin{figure*}[htbp]
\centering
\includegraphics[width=0.85\linewidth]{figures/fig2_xnode_concept}
\caption{\textbf{Pan-cancer target architecture.} \textbf{(A)} Target frequency across 77 cancer types with MHS predictions. Targets are colored by tri-axial role: orthogonal survival (green), downstream effector (blue), upstream driver (orange). STAT3 is the most frequent orthogonal-position target (78\% of cancers), consistent with its known constitutive activation~\cite{Yu2004} and algorithmic hub preference (see Limitations). Cell cycle regulators (CCND1, CDK4, CDK6) function as downstream effectors, and cancer-specific oncogenes (KRAS, EGFR, BRAF) serve as upstream drivers. \textbf{(B)} Tri-axial classification of the most frequent targets, showing gene counts per functional category.}
\label{fig:xnode}
\end{figure*}

\subsection*{Cancer-specific MHS predictions}
Representative cancer-specific MHS predictions (\cref{fig:heatmap}; full results in Supplementary Table~S1):
\begin{itemize}
\item \textbf{NSCLC} (165 cell lines): CCND1+CDK2+MCL1 --- three functionally distinct axes (cell cycle entry, cell cycle, anti-apoptosis).
\item \textbf{Melanoma} (135 lines): BRAF+CCND1+STAT3 --- extends BRAF inhibition with STAT3 and cell cycle targets.
\item \textbf{Colorectal} (96 lines): CDK4+CTNNB1+KRAS+STAT3 (4-target) --- the most complex common-cancer MHS, reflecting WNT/MAPK/JAK-STAT/cell cycle convergence.
\item \textbf{PDAC} (64 lines): CCND1+KRAS (2-target) --- notably, the static MHS misses STAT3, which Liaki et al.~\cite{Liaki2025} showed is essential for preventing dynamic resistance. This divergence highlights the limitation of static path coverage.
\item \textbf{Ewing Sarcoma} (51 lines): CDK4+FLI1+STAT3 --- uniquely includes the EWS-FLI1 fusion product.
\item \textbf{Chondrosarcoma} (9 lines): MCL1 alone --- the simplest MHS, suggesting a single dominant survival node in this chemotherapy-resistant tumor.
\end{itemize}

\begin{figure*}[htbp]
\centering
\includegraphics[width=0.95\linewidth]{figures/fig5_cancer_target_heatmap}
\caption{\textbf{Cancer--target heatmap.} Binary presence/absence of MHS targets (columns) across the 20 cancer types with the most cell lines (rows). Cells are colored by assigned tri-axial role: candidate orthogonal (green), downstream (blue), upstream (orange). Column headers show overall target frequency (number of cancers). The heatmap illustrates STAT3's recurrence as a candidate orthogonal node alongside cancer-specific target diversity. Role assignments are post-hoc functional annotations.}
\label{fig:heatmap}
\end{figure*}

\subsection*{Ranked triple combinations}
Ranked triples serve as the primary therapeutic hypothesis, enforcing tri-axial diversity via the hub-gene penalty (Equation~3). The two-stage architecture (MHS $\to$ ranked triples) produces substantial target redistribution: across 17 gold-standard cancer types, MHS and ranked triples are fully disjoint in 94.1\% (16/17) of cases, with only 1 partial overlap (Table~S4; Supplementary Figure~S11). MHS solutions are dominated by high-connectivity hub genes---STAT3 appears in 64.7\% of MHS predictions but only 5.9\% of ranked triples (removed in 11/17 cancer types), while CCND1 appears in 76.5\% of MHS solutions but 0\% of triples. The hub-gene penalty (Equation~3) drives this redistribution, replacing network hubs with cancer-specific druggable targets (EGFR in 14/17, MAP2K1 in 7/17, MET in 5/17 triples). Mean targets shared per cancer: 0.1; added: 2.9; removed: 2.1. Despite this near-complete transformation, the MHS stage is not redundant: it defines the survival mechanism landscape from which the triple scoring stage draws candidates (Supplementary Methods~S4). MHS cost and triple combined score are uncorrelated (Pearson $r = 0.21$, $p = 0.43$; Spearman $\rho = 0.24$, $p = 0.36$), confirming that scoring criteria---not cost minimization---drive the final ranking.

\textbf{Worked examples.}
\emph{Pancreatic Adenocarcinoma:} MHS selects \{CCND1, CDK6\} (2-gene cost-optimal coverage). The ranked triple is \{FYN, KRAS, STAT3\}---a complete disjoint transformation. CCND1 receives the highest hub penalty (0.92; path frequency far above median); STAT3, despite its own penalty (0.69), is retained uniquely for PDAC via the Tier~1 evidence exemption (Liaki et al.~\cite{Liaki2025}). The KRAS+FYN+STAT3 triple recovers the experimentally validated tri-axial combination.
\emph{Melanoma:} MHS selects \{CCND1, CDK4\} (both cell cycle regulators). The ranked triple is \{CDK6, EGFR, MAP2K1\}---again fully disjoint. CCND1's hub penalty (0.88) removes it; EGFR and MAP2K1 enter via synergy scoring and druggability. The triple captures the clinically relevant MAPK pathway targets (BRAF/MEK/EGFR axis).
\emph{Breast Carcinoma:} MHS selects \{CCND1, STAT3\}. The ranked triple is \{BCL2L1, EGFR, MAP2K1\}. STAT3 (penalty 0.64) and CCND1 (penalty 0.78) are both removed; BCL2L1, EGFR, and MAP2K1 are selected for cancer-specific essentiality and druggability.

Target concordance against the 43-entry gold standard (\cref{fig:benchmark}; Table~\ref{tab:benchmark}) yields:

\begin{table*}[htbp]
\centering
\caption{Target concordance against 43 gold-standard combinations (25 cancer types). Testable-only metrics exclude 18 structurally unmatchable entries (criteria defined \textit{a priori}: absent cancer type or all gold targets absent from CRISPR panel). Concordance measures partial target overlap with historical drug targets, not clinical efficacy.}
\label{tab:benchmark}
\resizebox{\textwidth}{!}{%
\begin{tabular}{@{}lcccccc@{}}
\toprule
\textbf{Method} & \textbf{AnyOvlp} & \textbf{PairOvlp} & \textbf{Precision} & \textbf{AnyOvlp\textsuperscript{T}} & \textbf{PairOvlp\textsuperscript{T}} & \textbf{Precision\textsuperscript{T}} \\
 & \multicolumn{2}{c}{\emph{(recall, n=43)}} & \emph{(cancer)} & \multicolumn{2}{c}{\emph{(recall, n=25)}} & \emph{(cancer)} \\
\midrule
ALIN & 44.2\% & 30.2\% & 47.1\% (8/17) & 64.0\% & 44.0\% & 58.3\% (7/12) \\
Driver genes & 44.2\% & 16.3\% & 100\% (8/8) & 60.0\% & 20.0\% & 100\% (7/7) \\
Random-global (1000$\times$) & 22.0\% & 8.8\% & 34.1\% & 32.2\% & 12.1\% & 43.3\% \\
Random-candidate (1000$\times$) & 0.2\% & 0.0\% & 0.3\% & 0.3\% & 0.0\% & 0.4\% \\
Global frequency & 48.8\% & 27.9\% & 58.8\% (10/17) & 76.0\% & 40.0\% & 83.3\% (10/12) \\
\midrule
\multicolumn{7}{@{}l}{\emph{ALIN lineage-stratified (all entries / testable):}} \\
\quad Carcinoma ($n{=}32$/$18$) & 59.4\% & 40.6\% & --- & 88.9\% & 61.1\% & --- \\
\quad Hematologic ($n{=}7$/$4$) & 0.0\% & 0.0\% & --- & 0.0\% & 0.0\% & --- \\
\quad CNS ($n{=}2$/$2$) & 0.0\% & 0.0\% & --- & 0.0\% & 0.0\% & --- \\
\quad Sarcoma ($n{=}2$/$1$) & 0.0\% & 0.0\% & --- & 0.0\% & 0.0\% & --- \\
\bottomrule
\end{tabular}%
}
\\[2pt]
{\footnotesize \textsuperscript{T}Testable entries only (25/43). Bottom panel: lineage-stratified ALIN recall. All concordance derives from carcinomas (Fisher's exact $p = 0.0008$; heterogeneity $\chi^2 = 11.7$, $p = 0.009$; Supplementary Figure~S9). Random-global samples from all $\sim$18{,}400 genes; Random-candidate from the per-cancer selective-essential pool. Driver baseline covers fewer cancer types (8 evaluable vs.\ ALIN's 17).}
\end{table*}

ALIN achieves exact target concordance for 3/43 entries (7.0\%): BRAF+EGFR in colorectal cancer~\cite{Kopetz2019}, EGFR+MET in NSCLC~\cite{Sequist2020}, and CDK4+ERBB2 in breast cancer~\cite{Ciruelos2024}---the only method in this comparison achieving exact concordance, though this claim depends on the specific baseline set and null model. ALIN's pair-overlap (30.2\%) exceeds all baselines including the driver-gene baseline (16.3\%; $p < 0.001$ vs.\ random-candidate null, permutation test: gold-standard targets shuffled within each cancer's candidate pool across 1{,}000 iterations). The candidate-pool random baseline (0.2\% any-overlap) confirms that scoring---not DepMap pre-filtering---drives performance. On testable entries, ALIN achieves 64.0\% any-overlap and 44.0\% pair-overlap, surpassing the driver-gene baseline (60.0\%/20.0\%). A \texttt{KNOWN\_SYNERGIES} ablation produces identical predictions, confirming zero circularity. Lineage-stratified evaluation reveals that all concordance derives from carcinomas (59.4\% any-overlap, 40.6\% pair-overlap; testable: 88.9\%/61.1\%), while hematologic ($n{=}7$), CNS ($n{=}2$), and sarcoma ($n{=}2$) entries achieve 0\% concordance (Table~\ref{tab:benchmark}, Supplementary Figure~S9). This heterogeneity is statistically significant ($\chi^2 = 11.7$, $p = 0.009$) and reflects two factors: (1)~hematologic targets (FLT3, BCL2, BTK, IDH1/2) are modality-specific and absent from CRISPR-essentiality-derived predictions, and (2)~CNS and sarcoma entries are too few for statistical power. The carcinoma-restricted recall (88.9\% testable any-overlap) represents the more appropriate benchmark for a CRISPR-based pipeline.

\begin{figure*}[htbp]
\centering
\includegraphics[width=0.85\linewidth]{figures/fig3_benchmark_comparison}
\caption{\textbf{Benchmark performance.} \textbf{(A)} Recall of ALIN ranked combinations against 43 independently curated gold-standard combinations spanning 25 cancer types: 7.0\% exact recall (3/43), 9.3\% superset recall (4/43), 30.2\% pair-overlap recall (13/43), 44.2\% any-overlap recall (19/43). On 25 testable entries: 64.0\% any-overlap, 44.0\% pair-overlap. \textbf{(B)} Match breakdown. \textbf{(C)} Baseline comparison (pair-overlap): ALIN 30.2\% surpasses driver-gene 16.3\%, random-global 8.8\%, random-candidate 0.0\%.}
\label{fig:benchmark}
\end{figure*}



Additional successful recoveries include BRAF+KRAS+STAT3 for PDAC (pair-overlap with Liaki et al.~\cite{Liaki2025}), BRAF+MAP2K1 for melanoma~\cite{Long2014}, and EGFR+MET for HCC. The 24 unrecovered combinations (55.8\%) reflect subtype-specific dependencies (HER2+ breast, BRAF V600E NSCLC), mutation-driven targets (FLT3 in AML), and therapeutic modalities absent from CRISPR screens (VEGFR2, mTOR).

\subsection*{Novel combinations for rare cancers}
Notable predictions for cancers with limited treatment options: Chondrosarcoma (MCL1 alone), Retinoblastoma (OTX2+STAT3), Nerve Sheath Tumor (HNRNPH1+STAT3), and Pleural Mesothelioma (CCND1+FGFR1+MDM2+STAT3, 4-target). Five of 10 priority combinations had no partial clinical trial matches, indicating complete novelty.

\subsection*{Validation and patient stratification}
Multi-source validation (PubMed, STRING, PRISM, ClinicalTrials.gov) was run on 10 priority combinations. All showed strong STRING PPI support. PRISM concordance scores ranged from 0.38 to 0.55. Patient stratification identified that 70--75\% of patients are addressable by existing biomarkers.

\subsection*{Component ablation and significance testing}
To quantify each pipeline module's contribution, we re-ran the full pipeline with one component disabled at a time and re-evaluated both benchmark concordance and top-1 triple stability (Table~\ref{tab:ablation}). We also tested the lineage-aware statistical alternative (Methods) to assess the impact of controlling for shared lineage dependencies.

\begin{table*}[htbp]
\centering
\caption{Component ablation: benchmark concordance (43 gold-standard entries) and top-1 triple stability after removing individual pipeline modules or substituting the lineage-aware statistical method. $\Delta$ denotes absolute change vs.\ full pipeline. \textbf{Top-1~$\Delta$} reports the number of cancer types whose top-ranked triple changed (including predictions lost entirely). \textbf{Preds} shows cancer types yielding ranked triples.}
\label{tab:ablation}
\small
\begin{tabular}{@{}lcccccc@{}}
\toprule
\textbf{Condition} & \textbf{AnyOvlp} & \textbf{PairOvlp} & \textbf{Exact} & \textbf{Preds} & \textbf{Top-1~$\Delta$} \\
\midrule
Full pipeline             & 46.5\% & 32.6\% & 7.0\% & 17 & --- \\
No OmniPath paths         & \phantom{0}7.0\% ($-$39.5) & \phantom{0}0.0\% ($-$32.6) & 0.0\% & \phantom{0}2 & 17/17 \\
No perturbation priors    & 23.3\% ($-$23.2) & 11.6\% ($-$21.0) & 0.0\% & 17 & 12/17 \\
No hub penalty            & 30.2\% ($-$16.3) & 20.9\% ($-$11.7) & 2.3\% & 17 & 17/17 \\
Lineage-aware statistical & 48.8\% ($+$2.3) & 34.9\% ($+$2.3) & 7.0\% & 17 & \phantom{0}5/17 \\
\bottomrule
\end{tabular}
\end{table*}

\begin{figure*}[!htb]
\centering
\includegraphics[width=0.88\linewidth]{figures/fig4_triple_patterns}
\caption{\textbf{Pathway shifting simulation (illustrative; parameters assumed, not fit to data).} ODE-based model comparing treatment strategies across five cancers (200-day simulations). \textbf{(A)} Final tumor viability for single-agent, intra-axial MHS, and tri-axial combination strategies. Tri-axial combinations achieve lower viability (mean 0.472 vs.\ 0.691). \textbf{(B)} Tri-axial advantage per cancer. Mean $\sim$30\%. \textbf{(C)} Compensatory pathway shift magnitude. Intra-axial strategies provoke stronger compensation (0.490) than tri-axial (0.434). All magnitudes are deductive consequences of assigned parameters.}
\label{fig:patterns}
\end{figure*}

OmniPath signaling paths are the single most important module: removing them collapses any-overlap from 46.5\% to 7.0\% and reduces predictions from 17 to 2 cancer types (all 17 top-1 triples affected), because most survival mechanisms are inferred via directed network paths. Perturbation priors contribute the second-largest effect ($-$23.2 pp any-overlap, $-$21.0 pp pair-overlap), changing 12/17 top-1 triples; without them, feedback-aware target selection is lost (e.g., Colorectal: EGFR+KRAS+MET $\to$ BRAF+CDK4+ERBB2) and exact concordance drops to zero. The hub-gene penalty is critical for target diversity: disabling it causes STAT3 to dominate all 17/17 triples, reducing any-overlap by 16.3~pp and pair-overlap by 11.7~pp. Substituting the lineage-aware statistical method (Methods) modestly improves concordance ($+$2.3 pp any-overlap, $+$2.3 pp pair-overlap) while changing only 5/17 triples; the two biologically notable changes are Colorectal (EGFR+KRAS+MET $\to$ BRAF+EGFR+KRAS, replacing MET with the established CRC target BRAF) and Breast (BCL2L1+EGFR+MAP2K1 $\to$ CDK4+ERBB2+PIK3CA, matching the clinical standard of care).

\textbf{Permutation test (protocol: Methods).} The observed any-overlap (44.2\%) far exceeded the null mean (0.08\%; $p < 0.001$; 1{,}000 iterations, 17 cancer types, median pool 1{,}717 selective-essential genes); pair-overlap (30.2\% vs.\ 0.03\%; $p < 0.001$) and exact concordance (7.0\% vs.\ 0.0\%; $p < 0.001$) were likewise significant. Zero of 1{,}000 null iterations matched or exceeded the observed values for any metric.

\textbf{Degree-preserving network null for STAT3 (protocol: Methods).} STAT3 appeared in 5/5 test cancers under the real OmniPath network but only $0.60 \pm 1.07$ cancers under the degree-preserved null (20 permutations, $10\times|E|$ edge swaps; range 0--4; $p < 0.05$). Under the null, CCND1 (96/100 cancer--permutation pairs) and CDK4 (58/100) dominated, consistent with their co-essentiality module membership being independent of network topology. STAT3 appeared in only 12/100 null pairs vs.\ 5/5 observed, confirming that its recurrence reflects biological signal beyond hub degree.

\subsection*{Pathway shifting simulation}
An ODE-based systems biology model illustrating the deductive consequences of the tri-axial hypothesis under assumed compensatory signaling parameters is presented in Supplementary Methods~S5 and Supplementary Figure~S4. Under these assumptions, tri-axial combinations yield $\sim$30\% lower modeled tumor viability than intra-axial combinations; however, this result is a logical consequence of the model's axioms, not an independent prediction (see Supplementary Methods~S5 for full details and epistemic caveats).

\section*{Discussion}

\subsection*{Biological significance}
A recurrent pattern across cancer types is a target architecture consistent with tri-axial organization: STAT3 as a candidate orthogonal node (78\% of cancers), cell cycle regulators as downstream effectors, and cancer-specific oncogenes as upstream drivers. This pattern is biologically plausible---STAT3 is constitutively activated in $>$70\% of solid tumors~\cite{Yu2004,Frank2007} and sits downstream of multiple oncogenic drivers---but it also reflects an algorithmic confound: as a high-degree network hub in OmniPath, STAT3 is mathematically favored in any minimum-cost hitting set that optimizes coverage with the fewest genes. Component ablation (Table~\ref{tab:ablation}) shows that removing OmniPath paths collapses any-overlap from 46.5\% to 7.0\%, confirming that signaling paths---not co-essentiality---drive STAT3 selection. A degree-preserving network null (20 edge-swap permutations, 5 test cancers) provides a direct test of hub-degree confounding (see Results). The recurrent \emph{architecture} (orthogonal + downstream + upstream) is consistent with the Liaki et al.~\cite{Liaki2025} PDAC-specific tri-axial principle; whether STAT3 \emph{specifically} occupies a functionally orthogonal position in each cancer type, and whether the three algorithmically selected nodes are truly functionally orthogonal axes rather than three nodes on a dense signaling graph, requires both the computational ablations above and disease-specific experimental validation.

\textbf{STAT3 as candidate orthogonal axis: algorithmic selection, hub correction, and clinical evidence.} STAT3 appears in 78\% (60/77) of MHS predictions despite being rarely mutationally activated. Its selection is biologically coherent: STAT3 functions as a convergent survival node downstream of multiple oncogenic drivers. Its high frequency reflects both genuine pan-cancer activation and algorithmic hub preference (see Limitations). However, STAT3's high connectivity in signaling networks makes it a \emph{mathematically optimal} hub for hitting-set coverage: it intersects many survival mechanisms at low marginal cost. A degree-preserving network null test (20 edge-swap permutations across 5 test cancers; see Results) shows that STAT3 appears in 5/5 observed cancers vs.\ a null mean of $0.60 \pm 1.07$ ($p < 0.05$), indicating that STAT3's recurrence significantly exceeds what network degree alone would produce. The remaining STAT3 discussion should be interpreted in light of these ablation and null results. Similarly, removing OmniPath paths entirely (retaining only co-essentiality and statistical dependency modules; see Table~\ref{tab:ablation}) collapses predictions to 2 cancers, confirming that signaling paths---not co-essentiality or statistical modules---drive STAT3 selection. To address the algorithmic confound, we (i)~removed a ``consensus path'' that previously included \emph{all} selective genes with maximum confidence, which artificially inflated STAT3's and other hub genes' mechanism frequency; and (ii)~introduced a \textbf{proportional hub-gene penalty} in ranked triple scoring (Equation~3) that penalizes genes whose survival-mechanism frequency exceeds the candidate median. With corrected survival mechanism inference and an evidence-aware exemption (retaining STAT3 only where Tier~1 experimental evidence exists, e.g.\ PDAC via Liaki et al.~\cite{Liaki2025}), STAT3 appears in only 1 of 17 benchmarked cancer-type ranked triples (PDAC: BRAF+KRAS+STAT3); the remaining 16 are dominated by cancer-specific oncogene targets (EGFR in 14/17, MET in 5/17, MAP2K1 in 7/17 predictions). The hub penalty does not affect MHS predictions (which remain a static combinatorial optimization) but ensures that ranked triples better reflect cancer-specific biology rather than network topology.

\textbf{CCND1 as a candidate pan-cancer cell cycle vulnerability.} The emergence of Cyclin D1 as the second most frequent MHS target (18 cancers, 23\%) is notable. While CCND1 amplification is known in individual cancers, its role as a minimal hitting set component across cancer types has not been described. This suggests a pan-cancer cell cycle entry dependence potentially exploitable via CDK4/6 inhibitors. However, cell cycle genes frequently appear in CRISPR essentiality screens because proliferation is the assay endpoint; thus, the apparent ``downstream effector'' role may partly reflect an assay artifact (proliferation-dependent scoring) rather than a cancer-specific therapeutic vulnerability. Controlling for generic proliferation dependencies (e.g., excluding pan-cancer cell cycle housekeeping genes or demonstrating cancer-selective CCND1 dependency) is needed to distinguish therapeutic targets from assay confounds. Notably, CDK4/6 inhibition can paradoxically \emph{protect} hematopoietic stem cells from chemotherapy-induced damage by inducing reversible G1 arrest~\cite{Roberts2012}, suggesting potential for sequential or intermittent dosing strategies that reduce myelosuppression.

\textbf{Static coverage vs.\ dynamic resistance.} While 65\% of cancers require only 2 targets for static survival mechanism coverage, the Liaki PDAC results demonstrate that the third axis is essential for preventing adaptive resistance. An ODE-based simulation (Supplementary Methods~S5) illustrates that, under assumed compensatory signaling parameters, same-axis targeting leaves orthogonal compensators intact, yielding $\sim$30\% higher modeled tumor viability than tri-axial combinations; however, this result is a deductive consequence of the model's axioms (see Supplementary Methods~S5 for epistemic caveats). The 2-target MHS represents a computational lower bound for static coverage, but the Liaki et al.\ experimental evidence~\cite{Liaki2025} argues that the number of targets for durable response should be guided by tri-axial coverage rather than the computational minimum.

\subsection*{Limitations}

\textbf{1. In silico nature and experimental gaps.} All predictions require preclinical validation. CRISPR knockout achieves complete gene ablation, differing from partial pharmacological inhibition~\cite{Lin2017,Goncalves2020}; only $\sim$25\% of drugs' killing patterns are phenocopied by CRISPR knockout of their targets. Single-gene screens also cannot capture emergent genetic interactions governing combination responses~\cite{Norman2019,Shen2017combCRISPR}. A co-essentiality interaction estimator and CRISPR--drug concordance filter partially mitigate these gaps.

\textbf{2. Cell line and subtype representation.} DepMap cell lines may not represent patient heterogeneity; some cancers have as few as 1--2 lines. Cell line genetic drift~\cite{BenDavid2018}, absence of tumor microenvironment (especially relevant for STAT3's dual immune role~\cite{Yu2009STAT3immunity,Kortylewski2005}), and 2D culture artifacts further limit generalizability. Pan-cancer-type analysis treats each disease as homogeneous; molecular subtypes (HER2+ vs.\ TNBC; EGFR-mutant vs.\ ALK+ NSCLC) have distinct dependencies, explaining benchmark misses for ERBB2 and ALK.

\textbf{3. PDAC divergence and ODE limitations.} The pipeline predicts a 2-target MHS for PDAC, while Liaki et al.~\cite{Liaki2025} achieved complete regression with 3 targets, demonstrating that static coverage underestimates targets needed for durable response. The ODE simulation illustrating tri-axial advantage is a deductive consequence of its assumed parameters, not independent validation (Supplementary Methods~S5).

\textbf{4. STAT3 hub bias.} STAT3's 78\% MHS frequency is partly algorithmic hub preference~\cite{VeraLicona2013}. After removing a spurious consensus path and implementing a hub-gene penalty (Equation~3), STAT3 appears in only 1/17 benchmarked ranked triples (PDAC, where Tier~1 evidence justifies retention), replaced elsewhere by cancer-specific targets. STAT3-directed agents have shown limited single-agent efficacy~\cite{Johnson2018,Jonker2018}, though STAT3 PROTACs~\cite{Bai2019} or biomarker-selected contexts may prove relevant.

\textbf{5. Benchmark limitations and circularity.} Target concordance measures partial overlap between predicted gene targets and historical drug targets; it does not measure clinical efficacy, pharmacological synergy, or resistance prevention. ``Any-overlap'' is permissive and can reward hub genes; ``exact match'' is brittle and can miss therapeutically relevant near-misses. Many approved combinations are not correctly represented as simple gene target sets due to polypharmacology and pathway-level effects. Of 24 unmatched gold-standard entries, 18 are structurally unmatchable (absent cancer types or non-CRISPR targets such as ESR1, KDR, PSMB5); the criteria for ``unmatchable'' were defined \textit{a priori} (see Methods). The gold standard was curated independently from clinical sources~\cite{Long2014,Kopetz2019,Turner2015,Daver2022}; a formal ablation confirms no circularity (disabling the KNOWN\_SYNERGIES lookup produces identical predictions for all 17 evaluable cancers).

\textbf{6. Methodological caveats.} Pipeline parameters (thresholds, weights, cluster count) were calibrated via systematic sweeps (500 configurations) but remain design choices~\cite{Dempster2021,Behan2019}. LOCO evaluation tests consistency, not generalization (predictions are not re-run with held-out data). Perturbation data covers only 13 curated signatures versus genome-wide resources like LINCS L1000~\cite{Subramanian2017}, creating knowledge bias toward well-studied kinases. Survival mechanisms lack formal pathway enrichment validation; ``100\% mechanism coverage'' is a mathematical guarantee of the MHS solver, not a claim that all survival mechanisms have been blocked. The MHS solver uses ILP for pools $\leq$500 genes but falls back to greedy approximation for larger pools.

\textbf{7. Network degree bias.} OmniPath is literature-curated, with known degree biases toward well-studied proteins (STAT3, TP53, AKT, MAPK family). The hub-gene penalty partially corrects this in ranked triples, but MHS predictions remain susceptible. A degree-preserving edge-swap null test (Results) directly quantifies the contribution of network topology vs.\ essentiality data to STAT3 selection; the OmniPath-removed ablation collapses predictions to 2 cancers, confirming the centrality of signaling paths.

\textbf{8. Proliferation assay confound.} CRISPR screens measure growth/viability, so genes essential for proliferation (cell cycle regulators, DNA replication factors) are inherently favored. The ``downstream effector'' role assigned to CCND1/CDK4/CDK6 may partly reflect this assay property rather than cancer-specific therapeutic vulnerability. Future analyses should control for core proliferation signatures.

\textbf{9. Druggability.} Several MHS components lack clinical-stage drugs, though 88\% (68/77) of predictions include at least one druggable target.

\subsection*{Comparison to existing approaches}
Unlike empirical screens (expensive, limited to available drugs), network-based methods (no mechanism coverage guarantee), ML approaches (limited interpretability), and fixed-size methods (biologically unjustified combination sizes), ALIN discovers the computational minimum via MHS optimization and then generates pathway-diverse triples guided by the tri-axial hypothesis. The candidate-pool random baseline achieves only 0.2\% any-overlap concordance, confirming that the scoring stages---not merely DepMap pre-filtering---drive performance. At pair-overlap level, ALIN (30.2\%) surpasses the driver-gene baseline (16.3\%; $1.85\times$ improvement) and is the only method in this comparison achieving exact-target concordance (Table~\ref{tab:benchmark}).

\section*{Future Work}

\textbf{Preclinical validation.} Priority cell-line experiments: (1)~Melanoma: BRAF + CDK4/6 inhibitor + STAT3 degrader~\cite{Bai2019}; (2)~PDAC: KRAS + CDK4/6 + STAT3 (MHS vs.\ tri-axial comparison); (3)~NSCLC: CDK + MCL1; (4)~Ewing Sarcoma: CDK4/6 + STAT3. Drug combination assays ($6 \times 6$ dose matrices, Bliss independence) comparing triple vs.\ best dual would quantify the third axis contribution. Organoid and xenograft models are needed for physiological relevance~\cite{BenDavid2018}.

\textbf{Computational validation and ablation.} (1)~\textbf{Lineage-aware evaluation}: now completed (Table~\ref{tab:benchmark}; Supplementary Figure~S9)---all concordance derives from carcinomas; subtype-level evaluation within carcinoma lineages (e.g., lung cancers only) remains; (2)~\textbf{proliferation control}: compare predicted targets against core proliferation gene sets to distinguish cancer-specific vulnerabilities from assay artifacts; (3)~\textbf{external validation}: use PRISM/GDSC drug sensitivity data to test whether predicted target pairs/triples correspond to drug sensitivity patterns; (4)~\textbf{bootstrap stability}: bootstrap across cell lines and report stability of top-$k$ targets and triples.

\textbf{Methodological.} (1)~Subtype-specific analysis (HER2+/TNBC, EGFR-mutant/ALK+ NSCLC) to address benchmark misses; (2)~integration of CCLE expression/CNV data; (3)~data-fitted ODE models with phosphoproteomics time-courses; (4)~expanded gold standard ($>$60 entries including KRAS G12D inhibitors)~\cite{Menden2019,Holbeck2017}; (5)~genome-wide pathway enrichment of survival mechanisms (MSigDB, Reactome); (6)~true held-out cross-validation; (7)~LINCS L1000 integration replacing curated perturbation data~\cite{Subramanian2017}; (8)~lineage-aware regression for cancer-specific dependency testing (now implemented; see Methods and Supplementary Figure~S10); (9)~Spearman rank-correlation co-essentiality as an additional alternative to Jaccard (Pearson comparison now completed; see Supplementary Methods~S2 and Supplementary Figure~S8).

\section*{Conclusion}

ALIN extends the tri-axial inhibition hypothesis from Liaki et al.'s PDAC system~\cite{Liaki2025} to 77 cancer types computationally, generating testable hypotheses rather than validated regimens. The central observation is a recurrent target architecture consistent with tri-axial organization: STAT3 as a candidate orthogonal node in 78\% of MHS predictions, cell cycle regulators as downstream effectors, and cancer-specific oncogenes as upstream drivers---though STAT3's dominance partly reflects algorithmic hub preference rather than validated functional orthogonality. A hub-gene penalty corrects this over-representation in ranked triples, producing cancer-specific predictions enriched for validated targets.

Against 43 gold-standard combinations, ALIN achieves 7.0\% exact target concordance, 30.2\% pair-overlap, and 44.2\% any-overlap ($p < 0.001$ vs.\ random-candidate null, permutation test), exceeding all baselines at pair-overlap level ($1.85\times$ vs.\ driver-gene). These are computational hypotheses for preclinical testing, not predictions of clinical efficacy; the gaps between CRISPR knockout and pharmacological inhibition~\cite{Lin2017,Goncalves2020}, between single-gene and combination screens~\cite{Norman2019}, and between algorithmic triads and mechanistically validated tri-axial combinations demand caution in interpretation.

\subsection*{Data and code availability}
Data: DepMap (\url{https://depmap.org}), PRISM (\url{https://depmap.org/repurposing}), OmniPath (\url{https://omnipathdb.org}), STRING~\cite{Szklarczyk2025} (\url{https://string-db.org}).
Code: \href{https://github.com/royerz2/Pan-Cancer-X-Node-Target-Discovery-System}{github.com/royerz2/Pan-Cancer-X-Node-Target-Discovery-System} with \texttt{run\_full\_pipeline.sh} for reproducibility.
Archived at Zenodo (DOI: \href{https://doi.org/10.5281/zenodo.18517646}{10.5281/zenodo.18517646}).
Detailed mathematical derivations and parameter justifications are provided in the Supplementary Methods document. See DATA\_AVAILABILITY.md for file URLs and licenses.

\subsection*{Competing interests}
The author declares no competing interests.

\subsection*{Author contributions (CRediT)}
\textbf{Roy Erzurumluoğlu}: Conceptualization, Methodology, Software, Validation, Formal Analysis, Investigation, Data Curation, Writing -- Original Draft, Writing -- Review \& Editing, Visualization.

\begin{acknowledgements}
We thank Liaki et al.~\cite{Liaki2025} for the foundational tri-axial combination therapy principle that motivates this work. DepMap, OmniPath, and PRISM consortia for open data.
\end{acknowledgements}

\section*{Bibliography}
% Use the provided .bib if present; otherwise fall back to a minimal manual bibliography
\IfFileExists{alin_refs.bib}{%
  \bibliographystyle{unsrt}
  \bibliography{alin_refs}
}{%
  \begin{thebibliography}{9}
  \bibitem{Liaki2025} Liaki V, Barrambana S, et al. 2025. A targeted combination therapy achieves effective pancreatic cancer regression. \textit{Proc Natl Acad Sci USA} 122(51):e2412835122.
  \bibitem{DepMap} DepMap: depmap.org
  \bibitem{OmniPath} OmniPath: omnipathdb.org
  \end{thebibliography}
}

\section*{Supplementary Materials}

\subsection*{Supplementary Methods}

\subsubsection*{S1. Data sources and version control}
All data were accessed in January 2026 and cached locally for reproducibility. DepMap release 24Q4 was downloaded from \url{https://depmap.org/portal/download/all/} (files: CRISPRGeneEffect.csv, Model.csv, OmicsDefaultModelProfiles.csv). The OmniPath directed signaling network was obtained via the OmniPath Python client (v1.0.8; \url{https://omnipathdb.org}). PRISM secondary repurposing screen data (19Q4) were downloaded from \url{https://depmap.org/repurposing/}. GDSC drug sensitivity data (IC$_{50}$ and AUC) were obtained from \url{https://www.cancerrxgene.org}. Exact file URLs, SHA-256 checksums, and download dates are recorded in the repository file DATA\_AVAILABILITY.md to enable precise data provenance tracking.

\subsubsection*{S2. Survival mechanism inference: mathematical definitions}

\paragraph{Essentiality threshold.} Gene $g$ is essential in cell line $\ell$ if its Chronos score satisfies $D_{g,\ell} < \theta_{\text{dep}}$ where $\theta_{\text{dep}} = -0.5$ (standard DepMap threshold for gene essentiality). Gene $g$ is selective for cancer type $c$ if it is essential in at least fraction $\theta_{\text{sel}} = 0.30$ of cell lines within $c$:
\[
\resizebox{\columnwidth}{!}{$\displaystyle
\text{Selective}(g, c) = \mathbb{1}\!\left[\frac{|\{\ell \in L_c : D_{g,\ell} < \theta_{\text{dep}}\}|}{|L_c|} \geq \theta_{\text{sel}}\right]
$}
\]
where $L_c$ is the set of cell lines annotated to cancer type $c$. Pan-essential genes, those essential in $>$90\% of all cell lines across all cancer types, are excluded to remove housekeeping dependencies.

\paragraph{Co-essentiality (Jaccard index).} For genes $g_i, g_j$ in cancer $c$, let $E_i = \{\ell \in L_c : D_{g_i,\ell} < \theta_{\text{dep}}\}$. The co-essentiality score is:
\[
\text{CoEss}(g_i, g_j) = \frac{|E_i \cap E_j|}{|E_i \cup E_j|}
\]
This Jaccard index ranges from 0 (no shared essentiality) to 1 (identical essentiality profiles). The resulting pairwise distance matrix ($1 - \text{CoEss}$) is clustered using Ward's agglomerative method with a dynamic tree cut producing 3--15 modules per cancer type. Each module constitutes a survival mechanism (referred to as a ``viability path'' in the MHS formulation for consistency with the hitting set literature).

\paragraph{Pearson co-essentiality comparison.} As an alternative to Jaccard binarization, we computed pairwise Pearson correlation on continuous Chronos dependency scores for each cancer type, converting to a distance matrix $d_{ij} = 1 - |r_{ij}|$. Both Jaccard and Pearson distance matrices were clustered with Ward's method for $k \in \{3, 4, \ldots, 15\}$, and agreement was measured by normalized mutual information (NMI) between the two partitions~\cite{Vinh2010}. Silhouette scores were computed for each clustering to assess internal cohesion. Across 10 cancer types with $\geq$74 cell lines (range 74--165; 949--1,340 selective genes), mean NMI was $0.20 \pm 0.07$, increasing modestly with $k$ (0.16 at $k{=}3$ to 0.22 at $k{=}15$). Per-cancer NMI ranged from 0.08 (Lung Neuroendocrine Tumor) to 0.30 (Diffuse Glioma). Jaccard clusterings yielded higher silhouette scores than Pearson at all $k$ values (mean 0.054 vs.\ 0.025), indicating tighter module boundaries under discretized profiles. These results confirm that Jaccard binarization and Pearson correlation capture complementary co-essentiality signals; the choice of Jaccard for ALIN is supported by its superior internal cohesion and interpretability for discrete module recovery.

\paragraph{Signaling path scoring.} For a directed path $p = (g_1 \to g_2 \to \cdots \to g_k)$ in the OmniPath network (maximum $k = 4$ hops), the dependency score is:
\[
S_{\text{path}}(p) = \frac{1}{k} \sum_{i=1}^{k} |\bar{D}_{g_i,c}|
\]
where $\bar{D}_{g_i,c}$ is the mean Chronos score of gene $g_i$ across cell lines of cancer $c$. Paths with $S_{\text{path}} < 0.3$ are pruned as insufficiently essential.

\paragraph{Cancer-specific statistical testing.} Gene $g$ is cancer-specific for cancer $c$ if it satisfies:
\begin{enumerate}
\item Welch's $t$-test: $q_{g,c} < 0.05$ (Benjamini--Hochberg FDR-corrected~\cite{Benjamini1995}; comparing Chronos scores of $g$ in $L_c$ vs.\ all other cell lines);
\item Effect size: Cohen's $d_{g,c} > 0.3$ (small-to-medium effect size threshold).
\end{enumerate}
Genes passing both criteria form an additional survival mechanism capturing lineage-specific vulnerabilities.

\paragraph{Lineage-aware regression alternative.} To control for shared lineage dependencies, we implemented an OLS regression model for each gene $g$:
\[
D_{g,\ell} = \beta_0 + \sum_{j=1}^{33} \beta_j \, \text{lin}_{j,\ell} + \beta_{\text{cancer}} \, \mathbb{1}_{\ell \in L_c} + \varepsilon_{g,\ell}
\]
where $\text{lin}_{j,\ell}$ are OncotreeLineage dummy variables (34 lineages, one dropped) and $\mathbb{1}_{\ell \in L_c} = 1$ if cell line $\ell$ belongs to the target cancer type. $\beta_{\text{cancer}}$ captures cancer-type-specific essentiality after removing lineage effects. Genes with BH-corrected $q_{\beta_{\text{cancer}}} < 0.05$ and $|\beta_{\text{cancer}}| > 0.3$ form the lineage-controlled cancer-specific survival mechanism.

Across 17 gold-standard cancer types, the lineage-aware model identifies substantially different gene sets from the Welch $t$-test (median Jaccard similarity = 0.051, mean = 0.118), reflecting removal of shared lineage dependencies. Nevertheless, final triple predictions change for only 2/17 cancer types (11.8\%): Colorectal Adenocarcinoma (EGFR+KRAS+MET $\to$ BRAF+EGFR+KRAS) and Invasive Breast Carcinoma (BCL2L1+EGFR+MAP2K1 $\to$ CDK4+ERBB2+PIK3CA). Benchmark concordance increases modestly from 46.5\% to 48.8\% any-overlap (Supplementary Figure~S10). The robustness of final predictions despite large gene-level changes confirms that the MHS + scoring modules are the dominant determinant of output quality.

\paragraph{Perturbation-response signatures (literature-curated).} We manually curate published perturbation data from $\sim$15 kinase-inhibitor studies (phosphoproteomics and transcriptional profiling after treatment) to identify feedback and bypass genes. For each inhibitor--gene pair, a response signature is defined as a set of genes whose expression or phosphorylation changes significantly upon treatment (e.g., EGFR phosphorylation increases upon KRAS inhibition, representing feedback reactivation). Where target-specific data are unavailable, isoform aliases share profiles from a related kinase (e.g., CDK6 shares the CDK4 profile; MAP2K1/MAP2K2 share MEK1; FYN shares SRC), a biologically dubious simplification given distinct substrate specificities (see Limitations~\#6). Essential genes overlapping with perturbation responders form additional survival mechanisms. Combinations that target perturbation-identified feedback genes receive a perturbation bonus $\beta_{\text{pert}} = 0.05$ per feedback gene covered, up to a maximum of 0.15. Confidence scores (0.85--0.95) assigned to each curated signature are author-assigned heuristics reflecting perceived study quality and replication, not formal statistical measures. This module does not integrate systematic high-throughput perturbation databases (e.g., LINCS L1000~\cite{Subramanian2017}; see Limitations~\#6).

\subsubsection*{S3. MHS cost function components}

The weighted cost for including gene $g$ in the MHS for cancer $c$ (Equation~2 in main text) comprises four terms:

\textbf{Toxicity} $\tau(g)$: Derived from OpenTargets safety data and DrugTargetDB adverse event profiles. Scores range from 0.0 (no known toxicity) to 1.0 (severe dose-limiting toxicities in clinical use). For genes without drug safety data, a default of 0.3 is assigned.

\textbf{Tumor specificity} $s(g,c)$: The selectivity of gene $g$ for cancer $c$ relative to all cancers:
\[
s(g,c) = \frac{f_{\text{ess}}(g,c) - \bar{f}_{\text{ess}}(g)}{\max_c f_{\text{ess}}(g,c) - \min_c f_{\text{ess}}(g,c) + \epsilon}
\]
where $f_{\text{ess}}(g,c)$ is the fraction of cell lines essential for $g$ in cancer $c$, $\bar{f}_{\text{ess}}(g)$ is the pan-cancer mean, and $\epsilon = 0.01$ prevents division by zero. Higher values indicate greater tumor selectivity.

\textbf{Druggability} $d(g)$: Binary-graded assessment: $d(g) = 1.0$ if an FDA-approved inhibitor exists; $d(g) = 0.6$ for compounds in Phase 2/3 trials; $d(g) = 0.3$ for Phase 1; $d(g) = 0.2$ for preclinical-only compounds; $d(g) = 0.0$ for undruggable targets.

\textbf{Pan-essentiality penalty} $\mathbb{1}_{\text{pan}}$: Equals 1 if the gene is essential in $>$90\% of all cell lines (indicating a housekeeping gene dependency); 0 otherwise. The $\times 2$ multiplier strongly penalizes selection of pan-essential genes.

\subsubsection*{S4. Ranked triple scoring: component definitions}

The composite ranking score (Equation~3 in main text) combines five components:

\textbf{Synergy score.} Derived from two sources: (1) known synergistic pairs from curated clinical data (e.g., BRAF+MEK: synergy = 0.9; BRAF+MAP2K1: synergy = 0.95; EGFR+MET: synergy = 0.90; CDK4+endocrine: synergy = 0.8) and (2) pathway diversity, quantified as the number of distinct biological pathways (MAPK, cell cycle, JAK/STAT, PI3K/AKT, apoptosis) covered by the triple, normalized to $[0, 1]$. When clinical evidence is available for at least one gene pair: weight 0.7 $\times$ known synergy + 0.3 $\times$ pathway diversity; otherwise: 0.6 $\times$ pathway diversity. This evidence-adaptive weighting ensures that FDA-approved and Phase~III-validated combinations receive appropriate credit relative to pathway-diversity heuristics.

\textbf{Resistance probability.} Estimated from curated resistance mechanisms: for each target in the triple, if known bypass mechanisms exist (e.g., EGFR $\to$ MET amplification upon EGFR inhibition), and the bypass gene is \emph{not} covered by another target in the triple, resistance probability increases by 0.15 per uncovered bypass. Triples covering their own bypass mechanisms score 0 (best).

\textbf{Combination toxicity (combo-tox).} Computed as:
\[
\resizebox{\columnwidth}{!}{$\displaystyle
\text{combo-tox} = \sum_{\text{DDI pairs}} w_{\text{DDI}} + \sum_{\text{shared tox}} w_{\text{overlap}}
$}
\]
where $w_{\text{DDI}} \in \{0.4 \text{ (major)}, 0.2 \text{ (moderate)}, 0.1 \text{ (minor)}\}$ and $w_{\text{overlap}} \in [0.08, 0.15]$ depending on the severity class of overlapping toxicities (e.g., myelosuppression = 0.15, hepatotoxicity = 0.12, QT prolongation = 0.15, dermatologic = 0.08).

\textbf{Path coverage.} Fraction of all inferred survival mechanisms for cancer $c$ that are intersected by at least one gene in the triple. Triples with coverage $< 0.70$ are excluded from ranking.

\textbf{Druggability count} $n_{\text{drugs}}$: Number of genes in the triple with FDA-approved or clinical-stage inhibitors (0, 1, 2, or 3).

\subsubsection*{S5. ODE model parameters and justification}

\begin{table*}[htbp]
\centering
\small
\caption{ODE model parameter values and biological justification.}
\begin{tabular}{@{}llll@{}}
\toprule
\textbf{Parameter} & \textbf{Symbol} & \textbf{Value} & \textbf{Justification} \\
\midrule
Basal production (drivers) & $b_i$ & 0.25--0.50 & Constitutive oncogene activation \\
Basal production (cascade) & $b_i$ & 0.03--0.08 & Dependent on upstream signaling \\
Basal production (orthogonal) & $b_i$ & 0.05--0.10 & Independent but regulatable \\
Degradation rate & $d_i$ & 0.05 h$^{-1}$ & Protein half-life $\sim$14 h \\
Hill coefficient & $n$ & 2 & Standard cooperativity \\
Hill half-max & $K$ & 0.5 & Normalized activity scale \\
Drug strength & $s_i$ & 0.92 & 92\% maximal inhibition \\
Drug onset rate & $\alpha$ & 0.15 h$^{-1}$ & $t_{1/2} \approx 4.6$ h to steady state \\
Compensatory gain (orthogonal) & $g_i$ & 0.4--0.7 & FYN$\to$STAT3 de-repression \\
Compensatory gain (cascade) & $g_i$ & 0.1--0.2 & Limited compensatory capacity \\
Simulation duration & $T$ & 4800 h & 200 days (Liaki observation window) \\
Integration method & ,  & RK45 & Adaptive step-size (scipy) \\
\bottomrule
\end{tabular}
\end{table*}

\textbf{Parameter sensitivity (robustness of the deductive structure).} Because the ODE parameters are assigned by biological role rather than fit to data, the sensitivity analysis characterizes the \emph{robustness of the tautology}, not the validity of the underlying assumptions. We performed univariate sensitivity analysis, varying each parameter $\pm$25\% from its nominal value. The qualitative ordering (tri-axial $<$ MHS $<$ single $<$ untreated) was preserved across all parameter perturbations. The steady-state viability difference between tri-axial and MHS strategies varied from 22.1\% (low compensatory gain $g_i = 0.3$) to 41.3\% (high compensatory gain $g_i = 0.9$), confirming that stronger assumed compensatory signaling increases the modeled advantage of tri-axial therapy. Drug strength had the largest absolute impact: $s_i = 0.80$ (80\% inhibition) reduced the tri-axial vs.\ MHS advantage to 18.5\%, while $s_i = 0.98$ (near-complete inhibition) increased it to 38.9\%. These ranges define the uncertainty envelope of the illustration, but do not constitute empirical validation of the compensatory gain assumptions themselves.

\subsubsection*{S6. Benchmark methodology}

The 43 multi-target ($\geq$2 gene) gold-standard combinations spanning 25 cancer types were independently curated from clinical sources without reference to ALIN outputs~\cite{Long2014,Planchard2016,Kopetz2019,Larkin2014,Park2021,Sequist2020,Baselga2012,Turner2015,Motzer2015,Daver2022,Ciruelos2024,Bhardwaj2019,Liaki2025}. Each entry requires at least two distinct gene targets to ensure combination-level evaluation. Evidence tiers:
\begin{enumerate}
\item \textbf{FDA-approved} (7 entries): Combinations with regulatory approval for the indicated cancer type (e.g., dabrafenib + trametinib for BRAF V600E melanoma; palbociclib + letrozole for ER+ breast cancer).
\item \textbf{Phase 2/3 clinical trials} (5 entries): Combinations with published positive efficacy data from randomized controlled trials (e.g., amivantamab + lazertinib for EGFR-mutant NSCLC; BEACON CRC).
\item \textbf{Preclinical validation} (1 entry): The Liaki et al.\ PDAC KRAS+EGFR+STAT3 triple~\cite{Liaki2025}, explicitly labeled as preclinical.
\end{enumerate}

\textbf{Match criteria.} The primary metric is \emph{any-overlap recall}: a predicted set $T$ matches gold-standard entry $G$ if $|G \cap T| \geq 1$~\cite{Julkunen2023}. A stricter \emph{pair-overlap recall} requires $|G \cap T| \geq 2$. Secondary metrics: ``Superset'' indicates $G \subseteq T$; ``Exact'' indicates $G = T$. Gene equivalences (MAP2K1$\leftrightarrow$MAP2K2, CDK4$\leftrightarrow$CDK6) are applied during matching.

\needspace{5\baselineskip}
\subsection*{Supplementary Figures}

\textbf{Fig.\ S1: ALIN pipeline detailed flowchart.}
End-to-end computational workflow from DepMap 24Q4 data ingestion (17,634 genes $\times$ 1,100 cell lines) and OmniPath network download through five survival mechanism inference methods (co-essentiality clustering, signaling topology, cancer-specific testing, perturbation signatures, driver landscaping), MHS optimization (greedy + ILP solvers), ranked triple scoring, and multi-source validation (PubMed, STRING, ClinicalTrials.gov, PRISM). Full pipeline executes in $\sim$45 minutes on a single CPU (Apple M2, 16~GB RAM).

\textbf{Fig.\ S2: Co-essentiality clustering methodology.}
\textbf{(A)} Binary essentiality matrix thresholded at Chronos $< -0.5$, filtered for selectivity ($\geq$30\% of lines) and pan-essentiality exclusion ($\leq$90\%). \textbf{(B)} Pairwise Jaccard co-essentiality computation. \textbf{(C)} Ward's hierarchical clustering with dynamic tree cut (3--15 modules per cancer). \textbf{(D)} Example melanoma modules: MAPK cascade (BRAF, MAP2K1, MAP2K2; mean Jaccard = 0.78), cell cycle (CDK4, CCND1, RB1; 0.65), JAK/STAT (STAT3, JAK1, FYN; 0.52).

\textbf{Fig.\ S3: Network path inference from OmniPath.}
\textbf{(A)} OmniPath directed subgraph for melanoma seeded from driver genes (BRAF, NRAS, EGFR), expanded up to 4 hops, pruned to nodes with mean Chronos $< -0.3$. Nodes colored by essentiality; edges weighted by OmniPath confidence ($\geq 0.5$ retained). \textbf{(B)} Dependency-weighted path scoring: $S_{\text{path}} = \frac{1}{|p|} \sum_{g \in p} |D_g|$; paths with $S_{\text{path}} < 0.3$ pruned. \textbf{(C)} Top melanoma paths: BRAF$\rightarrow$MEK$\rightarrow$ERK$\rightarrow$CCND1 ($S = 0.82$), EGFR$\rightarrow$KRAS$\rightarrow$BRAF ($S = 0.71$), JAK1$\rightarrow$STAT3 ($S = 0.58$).

\textbf{Fig.\ S4: Benchmark rank distribution.}
\textbf{(A)} Distribution of match types against 43 gold-standard entries: 3 exact, 1 superset, 9 pair-overlap, 6 any-overlap, 24 unmatched. \textbf{(B)} Baseline comparison: ALIN pair-overlap 30.2\% exceeds driver-gene 16.3\% ($1.85\times$) and global-frequency 27.9\%; ALIN is the only method with exact matches. \textbf{(C)} Unmatched entries involve subtype-specific dependencies, mutation-driven targets (FLT3), or modalities (VEGFR2, mTOR) not captured by pan-cancer DepMap data.

\textbf{Fig.\ S5: MHS combination size distribution.}
\textbf{(A)} MHS sizes across 77 cancers: 1~target (14.3\%), 2~targets (64.9\%), 3~targets (18.2\%), 4~targets (2.6\%); median = 2. \textbf{(B)} No significant difference across lineages (solid, hematologic, CNS, sarcoma; Kruskal--Wallis $p = 0.42$). \textbf{(C)} Positive correlation between cell line count and MHS size (Spearman $\rho = 0.31$, $p = 0.006$), reflecting increased statistical power rather than inherent complexity.

\textbf{Fig.\ S6: Perturbation-response path coverage (literature-curated).}
\textbf{(A)} Perturbation-derived survival mechanisms per cancer type (range: 0--8; mean: 3.2), reflecting knowledge bias toward well-studied kinases (13 curated signatures vs.\ $>$1.3M L1000 profiles~\cite{Subramanian2017}). \textbf{(B)} 35\% overlap with co-essentiality modules; 65\% capture unique feedback/bypass mechanisms. \textbf{(C)} Ablation: perturbation data changed MHS targets in 23\% (18/77) of cancers, adding resistance-relevant genes (EGFR, MET, CDK2).

\textbf{Fig.\ S7: Driver mutation landscape heatmap.}
Mutation frequencies of 10 driver genes across the 20 most well-powered cancer types ($\geq$30 cell lines). Data sourced from TCGA (cBioPortal), COSMIC, and OncoKB driver annotations. KRAS dominates PDAC (93.5\%); BRAF V600E dominates melanoma (68.7\%); TP53 LoF is the most frequent pan-cancer alteration (median 58\%). These frequencies inform upstream axis assignment in the ranked triple scoring step: driver genes with high mutation frequency in a cancer type receive scoring bonuses as candidate upstream targets. The integration is algorithmic (a scoring term proportional to mutation frequency), not a hard mapping rule.

\textbf{Fig.\ S9: Lineage-stratified benchmark concordance.}
\textbf{(A)} ALIN recall (any-overlap, pair-overlap, exact) stratified by lineage across all 43 gold-standard entries. All concordance derives from carcinomas ($n{=}32$); hematologic ($n{=}7$), CNS ($n{=}2$), and sarcoma ($n{=}2$) entries yield 0\% concordance. Dashed line: pooled any-overlap (44.2\%). \textbf{(B)} Testable entries only: carcinoma recall increases to 88.9\% any-overlap and 61.1\% pair-overlap ($n{=}18$ testable). \textbf{(C)} Per-cancer match level within each lineage, showing that misses concentrate in modality-specific targets (BCL2, FLT3, KDR) and small-sample lineages. Fisher's exact test: carcinoma vs.\ rest $p = 0.0008$; heterogeneity $\chi^2 = 11.7$, $p = 0.009$.

\textbf{Fig.\ S8: Jaccard vs.\ Pearson co-essentiality comparison.}
\textbf{(A)} NMI between Jaccard-binarized and Pearson-correlation Ward clusterings as a function of cluster count $k$ (3--15), averaged across 10 cancer types with $\geq$74 cell lines. Mean NMI increases modestly from 0.16 ($k{=}3$) to 0.22 ($k{=}15$), indicating low overall agreement. Shaded band shows $\pm$1 s.d.\ across cancer types. \textbf{(B)} Per-cancer mean NMI (across all $k$): Diffuse Glioma shows highest agreement (0.30); Lung Neuroendocrine Tumor lowest (0.08). \textbf{(C)} Silhouette score comparison: Jaccard consistently outperforms Pearson (mean 0.054 vs.\ 0.025), supporting tighter module boundaries under binary essentiality profiles.

\textbf{Fig.\ S10: Lineage-aware vs.\ Welch cancer-specific dependency comparison.}
\textbf{(A)} Per-cancer Jaccard similarity between cancer-specific gene sets identified by Welch $t$-test and lineage-aware OLS regression (median = 0.051, mean = 0.118). Low Jaccard indicates that the two methods identify substantially different gene sets, confirming that lineage effects dominate the Welch comparison. \textbf{(B)} Number of cancer-specific genes per method per cancer type. \textbf{(C)} Benchmark concordance: any-overlap increases from 46.5\% (Welch) to 48.8\% (lineage-aware); pair-overlap and exact remain unchanged. The two changed predictions (Colorectal: MET $\to$ BRAF; Breast: BCL2L1+EGFR+MAP2K1 $\to$ CDK4+ERBB2+PIK3CA) are both biologically improved.

\textbf{Fig.\ S11: MHS-to-triple target redistribution.}
\textbf{(A)} Overlap category distribution: 94.1\% of cancer types show fully disjoint MHS and triple target sets; 5.9\% show partial overlap. No cancer type has triple $\subseteq$ MHS or MHS $\subseteq$ triple. \textbf{(B)} STAT3 tracking across pipeline stages: present in 64.7\% of MHS predictions but only 5.9\% of ranked triples (removed in 11/17 cancer types). \textbf{(C)} Hub penalty comparison: genes removed from MHS carry higher hub penalties (violin plot) than those retained in triples. \textbf{(D)} MHS cost vs.\ triple combined score scatter (Pearson $r = 0.21$, $p = 0.43$): the two measures are uncorrelated, confirming that triple scoring captures distinct information from cost-minimization. \textbf{(E)} Per-cancer target changes: targets shared, added, and removed between MHS and triple for each cancer type. \textbf{(F)} Gene frequency comparison: top-12 genes ranked by total frequency across MHS and triples, showing systematic redistribution from cell cycle hubs (CCND1, STAT3) to cancer-specific druggable targets (EGFR, MAP2K1, CDK6).

\onecolumn
\subsection*{Supplementary Tables}

\noindent
\captionof{table}{\textbf{Table S1.} Representative MHS combinations for 15 of 77 cancer types, spanning diverse lineages and MHS sizes. STAT3 appears in 78\% (60/77) and cell cycle regulators in 65\% (50/77) of MHS sets. Full results in Supplementary Data File 1.}
\scriptsize
\setlength{\tabcolsep}{4pt}
\begin{tabular}{@{}p{2.8cm}lcp{5.8cm}r@{}}
\toprule
\textbf{Cancer Type} & \textbf{MHS Targets} & \textbf{Size} & \textbf{Druggable Targets} & \textbf{Cost} \\
\midrule

NSCLC & CCND1+CDK2+MCL1 & 3 & CDK2 (dinaciclib), MCL1 (AMG-176) & 4.15 \\
Melanoma & BRAF+CCND1+STAT3 & 3 & BRAF (vemurafenib), STAT3 (napabucasin) & 2.96 \\
Colorectal Adeno. & CDK4+CTNNB1+KRAS+STAT3 & 4 & CDK4 (palbociclib), KRAS (sotorasib), STAT3 (napabucasin) & 3.96 \\
PDAC & CCND1+KRAS & 2 & KRAS (sotorasib) & 1.94 \\
Breast (Invasive) & CDK4+PPP1R15B+STAT3 & 3 & CDK4 (palbociclib), STAT3 (napabucasin) & 3.17 \\
AML & CDK6+DNM2+STAT3 & 3 & CDK6 (palbociclib), STAT3 (napabucasin) & 2.89 \\
Diffuse Glioma & CDK6+CHMP4B+STAT3 & 3 & CDK6 (palbociclib), STAT3 (napabucasin) & 3.34 \\
Ewing Sarcoma & CDK4+FLI1+STAT3 & 3 & CDK4 (palbociclib), STAT3 (napabucasin) & 2.71 \\
Pleural Mesothel. & CCND1+FGFR1+MDM2+STAT3 & 4 & FGFR1 (erdafitinib), STAT3 (napabucasin) & 4.57 \\
Chondrosarcoma & MCL1 & 1 & MCL1 (AMG-176) & 0.96 \\
MPN & ABL1+CDK4+STAT3 & 3 & CDK4 (palbociclib), STAT3 (napabucasin) & 3.10 \\
Retinoblastoma & OTX2+STAT3 & 2 & STAT3 (napabucasin) & 2.03 \\
HCC & GRB2+STAT3 & 2 & STAT3 (napabucasin) & 2.28 \\
Prostate Adeno. & CDK4+STAT3 & 2 & CDK4 (palbociclib), STAT3 (napabucasin) & 1.68 \\
Cervical (Mixed) & ERBB2 & 1 & ERBB2 (trastuzumab) & 0.90 \\
\multicolumn{5}{@{}l}{\textit{$\ldots$ 62 additional cancer types; full table in supplementary CSV}} \\
\bottomrule
\end{tabular}
\normalsize
\setlength{\tabcolsep}{6pt}

\vspace{1.5em}
\noindent
\captionof{table}{\textbf{Table S2.} Gold-standard benchmark: 43 independently curated multi-target clinically validated combinations spanning 25 cancer types~\cite{Long2014,Planchard2016,Kopetz2019,Larkin2014,Park2021,Sequist2020,Baselga2012,Turner2015,Motzer2015,Daver2022,Ciruelos2024,Bhardwaj2019,Liaki2025}. Gene equivalences (MAP2K1$\leftrightarrow$MAP2K2, CDK4$\leftrightarrow$CDK6) applied during matching.}
\footnotesize
\begin{tabular}{@{}p{2.5cm}p{2.7cm}lp{3.8cm}p{2.4cm}@{}}
\toprule
\textbf{Cancer} & \textbf{Gold Targets} & \textbf{Evidence} & \textbf{Ranked Triple} & \textbf{Match} \\
\midrule

Melanoma & BRAF+MAP2K1 & FDA & BRAF+MAP2K1+STAT3 & Yes (superset) \\
Melanoma & BRAF+MAP2K2 & FDA & BRAF+MAP2K1+STAT3 & Yes (superset) \\
NSCLC & EGFR+MET & Breakthrough & \textemdash & No \\
NSCLC & ALK & FDA & \textemdash & No \\
Lung Neuroendocrine & KRAS & FDA & KRAS+MET+STAT3 & Yes (superset) \\
Breast (Invasive) & CDK4+CDK6 & FDA & CDK4+KRAS+STAT3 & Yes (superset) \\
Breast (Invasive) & CDK4+KRAS & Trial & CDK4+KRAS+STAT3 & Yes (superset) \\
Breast & ERBB2 & FDA & \textemdash & No \\
Colorectal & EGFR & FDA & \textemdash & No \\
Colorectal & BRAF+EGFR & Trial & \textemdash & No \\
Colorectal & KRAS & Trial & BRAF+KRAS+STAT3 & Yes (superset) \\
Ampullary & KRAS+STAT3 & Preclin & EGFR+KRAS+STAT3 & Yes (superset) \\
Adenosq.\\ PDAC & FYN+SRC+\\mbox{STAT3} & Preclin & CDK6+FYN+STAT3 & Yes (pair-overlap) \\
Ampullary & KRAS & FDA & EGFR+KRAS+STAT3 & Yes (superset) \\
AML & FLT3 & FDA & \textemdash & No \\
AML & CDK6+KRAS & Trial & CDK6+KRAS+STAT3 & Yes (superset) \\
RCC & MTOR & FDA & \textemdash & No \\
RCC & MTOR+VEGFR2 & Trial & \textemdash & No \\
HNSCC & EGFR+MET & Trial & CDK6+EGFR+MET & Yes (superset) \\
Diffuse Glioma & CDK4+CDK6 & Trial & CDK6+FYN+STAT3 & Yes (superset) \\
Synovial Sarcoma & CDK4+CDK6 & Trial & CDK6+FYN+STAT3 & Yes (superset) \\
Ampullary & EGFR & Trial & EGFR+KRAS+STAT3 & Yes (superset) \\
HCC & EGFR+MET & Trial & \textemdash & No \\
\bottomrule
\end{tabular}
\normalsize

\vspace{1.5em}
\noindent
\captionof{table}{\textbf{Table S3.} Priority MHS combinations for preclinical validation. Selection criteria: biological novelty, low MHS cost, availability of clinical-stage drugs, and cancer types where tri-axial inhibition can be tested against dual-therapy standards of care. For PDAC, we recommend adding STAT3 as a third axis per Liaki et al.~\cite{Liaki2025}.}
\vspace{0.5em}
\scriptsize
\setlength{\tabcolsep}{4pt}
\begin{tabular}{@{}p{2.2cm}lrp{6.8cm}@{}}
\toprule
\textbf{Cancer Type} & \textbf{MHS Targets} & \textbf{Cost} & \textbf{Rationale} \\
\midrule
Chondrosarcoma & MCL1 & 0.96 & Simplest MHS; single target; chemo-resistant tumor type \\
Ewing Sarcoma & CDK4+FLI1+STAT3 & 2.71 & Includes cancer-defining EWS-FLI1 fusion target \\
Melanoma & BRAF+CCND1+STAT3 & 2.96 & Extends BRAF inhibition; 68.7\% BRAF V600E \\
PDAC & CCND1+KRAS & 1.94 & 2-target MHS; add STAT3 as third axis per Liaki et al.~\cite{Liaki2025} \\
Retinoblastoma & OTX2+STAT3 & 2.03 & Cancer-specific OTX2; rare pediatric tumor \\
\bottomrule
\end{tabular}
\setlength{\tabcolsep}{6pt}
\normalsize

\begin{table*}[t]
\caption{\textbf{Table S4.} MHS-to-triple target redistribution across 17 gold-standard cancer types. For each cancer, the table shows the best MHS solution (cost-minimizing mechanism coverage), the top ranked triple (scored by Equation~3), and the target changes between them. STAT3 status indicates whether STAT3 appears in MHS only, triple only, both, or neither. 94.1\% of cancer types show fully disjoint MHS and triple targets, reflecting the hub-gene penalty redistribution from network hubs to cancer-specific druggable targets.}
\label{tab:s4_mhs_triple}
\resizebox{\textwidth}{!}{%
\begin{tabular}{lcrlllll}
\toprule
\textbf{Cancer Type} & \textbf{MHS Targets} & \textbf{Cost} & \textbf{Top Triple} & \textbf{Added} & \textbf{Removed} & \textbf{Overlap} & \textbf{STAT3} \\
\midrule
AML & CDK6+SOS1+SPI1+STAT3 & 4.16 & CDK4+CDK6+MCL1 & CDK4, MCL1 & SOS1, SPI1, STAT3 & partial & mhs only \\
Anapl.\ Thyroid & BRAF+STAT3 & 2.06 & CDK6+EGFR+MAP2K1 & CDK6, EGFR, MAP2K1 & BRAF, STAT3 & disjoint & mhs only \\
Bladder & CCND1+STAT3 & 2.52 & CDK2+EGFR+MAP2K1 & CDK2, EGFR, MAP2K1 & CCND1, STAT3 & disjoint & mhs only \\
Colorectal & CCND1+STAT3 & 2.52 & EGFR+KRAS+MET & EGFR, KRAS, MET & CCND1, STAT3 & disjoint & mhs only \\
Diff.\ Glioma & CCND1+STAT3 & 2.52 & CDK2+EGFR+MAP2K1 & CDK2, EGFR, MAP2K1 & CCND1, STAT3 & disjoint & mhs only \\
Endometrial & CCND1+STAT3 & 2.52 & CDK2+EGFR+MET & CDK2, EGFR, MET & CCND1, STAT3 & disjoint & mhs only \\
Esophagogastric & CCND1+STAT3 & 2.34 & CDK6+EGFR+PIK3CA & CDK6, EGFR, PIK3CA & CCND1, STAT3 & disjoint & mhs only \\
HNSCC & EGFR+STAT3 & 2.14 & CDK4+CDK6+ERBB2 & CDK4, CDK6, ERBB2 & EGFR, STAT3 & disjoint & mhs only \\
HCC & CCND1+CDK4 & 2.28 & BCL2L1+EGFR+MAP2K1 & BCL2L1, EGFR, MAP2K1 & CCND1, CDK4 & disjoint & neither \\
Breast & CCND1+STAT3 & 2.30 & BCL2L1+EGFR+MAP2K1 & BCL2L1, EGFR, MAP2K1 & CCND1, STAT3 & disjoint & mhs only \\
Liposarcoma & CCND1+CDK4 & 1.84 & EGFR+FGFR1+MET & EGFR, FGFR1, MET & CCND1, CDK4 & disjoint & neither \\
Melanoma & CCND1+CDK4 & 2.13 & CDK6+EGFR+MAP2K1 & CDK6, EGFR, MAP2K1 & CCND1, CDK4 & disjoint & neither \\
NSCLC & CCND1+STAT3 & 2.46 & CDK6+EGFR+MAP2K1 & CDK6, EGFR, MAP2K1 & CCND1, STAT3 & disjoint & mhs only \\
Ovarian & CCND1+CDK4+STAT3 & 3.59 & CDK6+EGFR+MET & CDK6, EGFR, MET & CCND1, CDK4, STAT3 & disjoint & mhs only \\
PDAC & CCND1+CDK6 & 2.24 & FYN+KRAS+STAT3 & FYN, KRAS, STAT3 & CCND1, CDK6 & disjoint & triple only \\
Prostate & CCND1+CDK4 & 1.54 & CDK2+EGFR+MAP2K1 & CDK2, EGFR, MAP2K1 & CCND1, CDK4 & disjoint & neither \\
Renal & CCND1+ZNF316 & 2.32 & CDK6+EGFR+MAP2K1 & CDK6, EGFR, MAP2K1 & CCND1, ZNF316 & disjoint & neither \\
\bottomrule
\end{tabular}}
\end{table*}

\end{document}
