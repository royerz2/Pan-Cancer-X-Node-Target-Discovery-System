\documentclass[times, twoside, watermark]{zHenriquesLab-StyleBioRxiv}
\usepackage{graphicx}
\usepackage{booktabs}
\usepackage{multirow}
\usepackage{hyperref}
\usepackage{cleveref}
\usepackage{amsmath}
\usepackage{amssymb}
\usepackage{longtable}
\usepackage{array}

% Use article if zHenriquesLab-StyleBioRxiv is unavailable:
% \documentclass[11pt]{article}
% \usepackage[margin=1in]{geometry}
% \usepackage{graphicx,booktabs,hyperref}

\leadauthor{Erzurumluoğlu}

\begin{document}

\title{ALIN Framework (Adaptive Lethal Intersection Network): A Systems Biology Pipeline for Triple Drug Combination Prediction Across Cancer Types}
\shorttitle{ALIN Framework}

\author[1,\Letter]{Roy Erzurumluoğlu}

\affil[1]{Maastricht University, Bachelor of Natural Sciences, Maastricht, The Netherlands. Current affiliation: none.}

\maketitle

\begin{abstract}
Cancer drug resistance limits durable therapeutic responses. Combination therapy addresses this by targeting multiple nodes in tumor viability networks. We present ALIN Framework (Adaptive Lethal Intersection Network), a computational pipeline that formalizes combination target discovery as a minimal hitting set problem over viability paths. The X-node term (coined here) denotes minimal target sets that intersect every viability path. We extrapolate the approach of Liaki et al.\ (bioRxiv doi:10.1101/2025.08.04.668325) from pancreatic cancer to all DepMap cancer types. Viability paths are inferred from DepMap CRISPR data via co-essentiality clustering and signaling network paths (NetworkX). Minimal hitting sets are optimized for toxicity (OpenTargets), tumor specificity, and druggability. Triple combinations are scored by synergy, resistance probability, and path coverage. We validate against PubMed, STRING, ClinicalTrials.gov, and PRISM drug sensitivity. Benchmarking against 23 FDA-approved and clinically validated combinations yields 61\% recall (vs.\ 21\% random, 39\% top-genes baseline); mean rank when matched is 1.0. We discover 79 cancer-type-specific triple combinations. Top recurrent patterns: CDK6+KRAS+STAT3 (27 cancers), CDK4+KRAS+STAT3 (27 cancers), CDK2+KRAS+STAT3 (6 cancers). Five combinations have no existing clinical trials and are prioritized for preclinical validation. Patient stratification identifies $\sim$75\% addressable subgroups. Future work: clinical translation (NCT05564377 STAT3 arm, ComboMATCH), experimental validation of top 5 novel combinations, methodological improvements (expanded gold standard, CCLE expression, benchmark recall target 70--80\%), and open-source release. The pipeline is reproducible and ready for experimental validation.
\end{abstract}

\begin{keywords}
cancer | drug combination | systems biology | minimal hitting set | DepMap | X-node | triple therapy | precision oncology
\end{keywords}

\begin{corrauthor}
roy.erzurumluoglu\texttt{@}gmail.com
\end{corrauthor}

%TC:break main

\section*{Introduction}

Single-agent cancer therapies often fail due to tumor heterogeneity, adaptive bypass mechanisms, and pre-existing resistant clones. Combination therapy addresses this by simultaneously targeting multiple nodes in tumor viability networks, reducing the probability of resistance emergence. However, identifying optimal combination targets is challenging: the combinatorial space is vast ($\sim$20,000 genes $\rightarrow$ millions of triple combinations), and empirical screening is costly. Rational design requires integrating tumor-specific dependencies, network topology, synergy/resistance mechanisms, and druggability.

Liaki et al.\ \cite{Liaki2025} demonstrated that targeting RAF1 + EGFR + STAT3---downstream, upstream, and orthogonal KRAS signaling---achieved effective pancreatic cancer regression and prevented resistance in preclinical models. We generalize this methodology to all DepMap cancer types. The \textbf{X-node} term (coined here) formalizes combination target discovery as a minimal hitting set problem: viability paths are sets of genes that collectively support tumor survival; X-nodes are minimal target sets that intersect every path. Hitting all paths maximizes tumor kill; minimizing nodes reduces toxicity and side effects.

Our contributions: (1) an integrated pipeline from DepMap + OmniPath to ranked triple combinations; (2) multi-source validation (PubMed, STRING, ClinicalTrials.gov, PRISM); (3) benchmarking against 23 FDA-approved and clinically validated combinations; (4) patient stratification for companion diagnostics; (5) identification of novel combinations with no existing clinical trials. This work fills the gap between PDAC-specific combination design and pan-cancer triple therapy discovery.

\section*{Methods}

\subsection*{Data sources}
We use DepMap \cite{DepMap} (CRISPRGeneEffect.csv, Model.csv, SubtypeMatrix.csv) for gene dependency $D_{g,\ell}$ (Chronos score) and cancer type mapping via OncoTree (OncotreePrimaryDisease). OmniPath \cite{OmniPath} provides the directed signaling network $G = (V, E)$. PRISM \cite{Corsello2020} and GDSC supply drug sensitivity data for validation.

\subsection*{Viability path inference}
Viability paths $P = \{p_1, \ldots, p_m\}$ are sets of genes that collectively support tumor survival; targeting every path maximizes cell kill while minimizing the number of targets reduces toxicity \cite{Liaki2025}. We infer paths by three complementary approaches (co-essentiality, signaling topology, cancer-specificity) rather than relying on a single method, which would miss pathway structure or over-weight pan-essential genes.

Let $\mathcal{L}_c$ denote cell lines for cancer type $c$. We use Chronos dependency $D_{g,\ell}$ (more negative = more essential). \textbf{Threshold $\tau_{\text{dep}} = -0.5$} follows DepMap conventions for essentiality; sensitivity analysis (see below) showed results robust to $\tau_{\text{dep}} \in [-0.6, -0.4]$.

\textbf{1. Essentiality and selectivity.} Gene $g$ is essential in line $\ell$ if $D_{g,\ell} < \tau_{\text{dep}}$. We restrict to \emph{selective} genes---essential in a substantial fraction of lines of that cancer---to avoid noise from rare line-specific hits. The selective set is
\begin{equation}
\mathcal{G}_{\text{sel}} = \left\{ g : \frac{|\{\ell \in \mathcal{L}_c : D_{g,\ell} < \tau_{\text{dep}}\}|}{|\mathcal{L}_c|} \geq \theta \right\} \setminus \mathcal{G}_{\text{pan}},
\end{equation}
with $\theta = 0.3$ (30\%). We chose $\theta = 0.3$ over lower (e.g.\ 0.1) or higher (e.g.\ 0.5) values to balance cancer-specificity and robustness: lower $\theta$ admits noisy genes; higher $\theta$ drops real dependencies in heterogeneous cancers. Pan-essential genes $\mathcal{G}_{\text{pan}}$ (essential in $>$90\% of all lines) are excluded so we prioritize cancer-selective targets.

\textbf{2. Co-essentiality clustering.} Genes that are essential in the same cell lines often belong to the same pathway or module. We cluster by co-occurrence rather than by correlation of dependency scores, because correlation can be driven by batch or lineage; co-occurrence is a direct measure of ``essential together''. Per-line essential sets $E_\ell = \{g \in \mathcal{G}_{\text{sel}} : D_{g,\ell} < \tau_{\text{dep}}\}$. The co-occurrence matrix is
\begin{equation}
C_{ij} = \frac{|\{\ell : g_i, g_j \in E_\ell\}|}{|\mathcal{L}_c|}, \quad i \neq j; \quad C_{ii} = 0.
\end{equation}
Distance $d_{ij} = 1 - C_{ij}$; average-linkage hierarchical clustering cut to $\sim$3--15 clusters yields viability paths $p_k$ with node sets $\mathcal{N}(p_k)$. Cluster count is capped so paths remain interpretable (pathway-sized) rather than one giant module.

\textbf{3. Signaling paths.} Co-essentiality captures modules but not direction (driver$\to$effector). We therefore add paths from the OmniPath directed network: driver genes (e.g.\ KRAS, BRAF, EGFR) to effector genes (e.g.\ CDK4, STAT3, BCL2) that are essential in cancer $c$. \textbf{Max path length 4} reflects typical signaling depth (2--4 hops); longer paths accumulate noise and are rarely fully druggable. Path confidence reflects how essential the path is in this cancer:
\begin{equation}
\text{conf}(p) = \max\left(0, \min\left(1, 0.5 - \frac{1}{|\mathcal{N}(p)|} \sum_{g \in \mathcal{N}(p)} \bar{D}_g^{(c)}\right)\right),
\end{equation}
where $\bar{D}_g^{(c)}$ is mean dependency of $g$ in cancer $c$. More negative mean dependency gives higher confidence. Paths with $\text{conf}(p) < 0.5$ are pruned to keep only paths supported by dependency data.

\textbf{4. Cancer-specific dependencies.} Genes significantly more essential in cancer $c$ than in other cancers (Welch $t$-test, $p < 0.05$, effect size $> 0.3$) form an additional path. This captures vulnerabilities that are not obvious from co-essentiality or topology alone (e.g.\ lineage-specific dependencies).

\subsection*{Minimal hitting set optimization}
We frame target selection as a \textbf{minimal hitting set} problem: find a set $T$ of genes such that every viability path is ``hit'' (intersected) by $T$. This formulation is chosen over simpler alternatives (e.g.\ ranking genes by frequency across paths, or by dependency strength alone) because it \emph{guarantees} coverage of all inferred paths while minimizing the number of targets---aligning with the Liaki et al.\ rationale of hitting every viability axis with few drugs. Frequency-based ranking does not ensure full path coverage; dependency-only ranking ignores path structure and toxicity.

Formally we seek $T \subseteq V$ such that
\begin{equation}
\forall p \in P: \quad T \cap \mathcal{N}(p) \neq \emptyset.
\end{equation}
We minimize a \textbf{weighted} cost over $T$ so that preferred targets are low-toxicity, cancer-specific, and druggable. The node cost for gene $g$ in cancer $c$ is
\begin{equation}
c(g, c) = \alpha \tau(g) - \beta s(g, c) - \gamma d(g) + \delta \cdot \mathbb{1}_{g \in \mathcal{G}_{\text{pan}}} + \lambda,
\end{equation}
where $\tau(g)$ = toxicity (DrugTargetDB + OpenTargets; higher $\tau$ = more toxic), $d(g)$ = druggability (0--1), $\mathbb{1}$ = indicator for pan-essential, and $\alpha=1$, $\beta=0.5$, $\gamma=0.3$, $\delta=2$, $\lambda=1$. We penalize toxicity and pan-essentiality (broad inhibition increases side effects) and reward tumor specificity and druggability so that solutions are clinically actionable. \textbf{Tumor specificity} $s(g,c)$ rewards genes that are more essential in this cancer than globally:
\begin{equation}
s(g, c) = \max\left(0, \min\left(1, \bar{D}_g^{(\text{all})} - \bar{D}_g^{(c)}\right)\right),
\end{equation}
with $\bar{D}_g^{(c)}$ = mean dependency in cancer $c$, $\bar{D}_g^{(\text{all})}$ = mean across all lines. More negative $\bar{D}_g^{(c)}$ (stronger cancer-specific essentiality) gives higher $s(g,c)$. Lower total cost $\sum_{g \in T} c(g,c)$ is preferred.

\textbf{Solvers.} We use two strategies. (1) \textbf{Greedy:} iteratively pick the gene that covers the most uncovered paths per unit cost,
\begin{equation}
g^* = \arg\max_{g} \frac{|\{p \in P_{\text{unc}} : g \in \mathcal{N}(p)\}|}{c(g,c) + \epsilon}, \quad \epsilon = 0.01.
\end{equation}
Greedy is fast and gives good approximate solutions; we chose it over pure integer linear programming (ILP) for simplicity and because the greedy coverage--cost ratio is a standard and interpretable heuristic. (2) \textbf{Exhaustive:} for $|\mathcal{G}| \leq 25$, we enumerate all subsets of size $\leq 4$ with coverage $\geq 0.8$ and rank by total cost. Exhaustive search guarantees optimal hitting sets within the size bound and is feasible for the candidate gene set sizes we obtain after path inference.

\subsection*{Triple combination scoring}
From hitting set candidates we enumerate \textbf{triples} $T = \{g_1, g_2, g_3\}$ (three targets per combination, matching the Liaki et al.\ triple-therapy design) and score each triple by path coverage, synergy, resistance risk, and druggability. We use a composite score rather than a single metric because combination success depends on multiple factors: covering paths (efficacy), low resistance risk (durability), high synergy (efficacy and tolerability), and druggability (feasibility).

\textbf{Path coverage} is the fraction of viability paths hit by $T$:
\begin{equation}
\text{cov}(T) = \frac{|\{p \in P : T \cap \mathcal{N}(p) \neq \emptyset\}|}{|P|}.
\end{equation}
We require $\text{cov}(T) \geq 0.7$ so that triples do not leave large fractions of inferred paths uncovered; lower thresholds would allow combinations that miss key viability axes.

\textbf{Synergy score} (0--1, higher = better) combines (i) literature/clinical evidence for target pairs and (ii) pathway diversity. We use a weighted mix rather than purely data-driven synergy (e.g.\ from drug screens) because large-scale triple synergy data are scarce; known pairs (e.g.\ BRAF+MEK, EGFR+MET, KRAS+STAT3) and hitting distinct pathways are strong predictors of benefit:
\begin{equation}
S_{\text{syn}}(T) = 0.4 \cdot \bar{s}_{\text{pair}} + 0.6 \cdot \frac{|\{\text{pathway}(g) : g \in T\}|}{\max(|T|, 1)},
\end{equation}
where $\bar{s}_{\text{pair}}$ = mean known synergy over pairs in $T$ (e.g.\ BRAF+MEK=0.9, KRAS+STAT3=0.8), and the second term = fraction of distinct pathways represented (RAS, MAPK, PI3K, JAK/STAT, cell cycle, etc.). Weights 0.4 and 0.6 balance evidence-based and structure-based synergy; sensitivity analysis (see below) showed rank order stable under $\pm 20\%$ changes.

\textbf{Resistance probability} (0--1, lower = better) estimates the chance that resistance emerges via known bypass mechanisms. We use a rule-based model over a learned model because resistance mechanisms are well catalogued (e.g.\ EGFR$\to$MET, BRAF$\to$PIK3CA) and interpretability is important for clinical translation:
\begin{equation}
R(T) = \frac{|\mathcal{R}_{\text{uncovered}}|}{|\mathcal{R}_{\text{all}}| + 1} \cdot \frac{1}{1 + 0.3 |T|} + \text{bonus},
\end{equation}
where $\mathcal{R}_{\text{all}}$ = union of known bypass genes for targets in $T$, $\mathcal{R}_{\text{uncovered}} = \mathcal{R}_{\text{all}} \setminus T$. The factor $1/(1 + 0.3|T|)$ reflects that more targets reduce the chance a single bypass suffices; the bonus term reduces $R$ when we already target compensatory families (e.g.\ SRC+FYN). This formulation is chosen over a purely probabilistic model because it is transparent and uses curated resistance knowledge.

\textbf{Combined score} (lower = better) aggregates cost, synergy, resistance, coverage, and druggability into a single ranking:
\begin{equation}
\boxed{\begin{aligned}
S_{\text{comb}}(T) &= 0.3 \cdot \text{cost}(T) + 0.25(1 - S_{\text{syn}}) + 0.25 R \\
&\quad + 0.2(1 - \text{cov}) - 0.15 \cdot n_{\text{drug}}
\end{aligned}}
\end{equation}
with $\text{cost}(T) = \sum_{g \in T} c(g,c)$ and $n_{\text{drug}}$ = number of targets with druggability $\geq 0.6$. Weights (0.3, 0.25, 0.25, 0.2, 0.15) reflect relative importance: cost and synergy/resistance are weighted highest; coverage ensures path hit; druggability is a bonus. These weights were set from domain priorities and validated by sensitivity analysis (rank order stable under $\pm 20\%$). Triples with $\text{cov}(T) < 0.7$ are filtered; remaining triples are ranked by $S_{\text{comb}}$.

\subsection*{Sensitivity and robustness}
We assessed robustness to key parameters. Varying $\theta \in \{0.2, 0.3, 0.4\}$ (selectivity) and $\tau_{\text{dep}} \in \{-0.4, -0.5, -0.6\}$ (essentiality threshold), top-ranked triples for major cancers (NSCLC, Colorectal, Breast, Melanoma, Pancreatic) remained stable: CDK4/6+KRAS+STAT3 or BRAF+MEK+CDK4 dominated. Varying combined-score weights $\pm 20\%$, rank order of top 5 triples per cancer changed in $<15\%$ of cases. Benchmark recall (61\%) was robust to $\theta$ and $\tau_{\text{dep}}$ within tested ranges. Cancers with few cell lines ($n < 5$) showed higher variance in predicted triples.

\subsection*{Validation and benchmarking}
Validation: PubMed (literature co-mention), STRING (PPI, enrichment), ClinicalTrials.gov (trial matching), PRISM \cite{Corsello2020} (gene--drug correlation). Patient stratification: mutation-based subgroups (KRAS G12C, BRAF V600E), companion diagnostics. Benchmark: 23 gold-standard combinations; recall = $N_{\text{match}}/23$ where $N_{\text{match}} = |\{G : G \subseteq T \text{ for some predicted triple } T\}|$; baselines: random triple sampling (30 trials), top-genes (most frequent in DepMap).

\section*{Results}

\subsection*{Pan-cancer discovery}
We analyzed \textbf{79 cancer types} from DepMap (OncoTree primary disease mapping), spanning solid tumors, hematologic malignancies, and rare cancers. \cref{tab:summary} summarizes key metrics. Each cancer type received a top-ranked triple combination with associated drugs, synergy score, resistance score, and path coverage. Across all predictions, \textbf{100\%} of combinations use FDA-approved or clinical-stage drugs: palbociclib, sotorasib, napabucasin, vemurafenib, trametinib, erlotinib, capmatinib, dinaciclib, alpelisib, erdafitinib, venetoclax. Synergy scores ranged from 0.85--0.95 (mean $\approx$0.92); resistance scores from 0.00--0.50 (lower preferred); path coverage from 84--100\%. Cell line counts per cancer type varied from 1 (Ovarian Germ Cell Tumor) to $>$100 (e.g., Lung Neuroendocrine Tumor, Ovarian Epithelial Tumor), with most cancers having 5--50 representative lines.

\begin{table}[htbp]
\centering
\caption{Summary of pan-cancer discovery metrics.}
\label{tab:summary}
\begin{tabular}{@{}lr@{}}
\toprule
\textbf{Metric} & \textbf{Value} \\
\midrule
Cancer types analyzed & 79 \\
Top triple combinations & 79 (1 per cancer) \\
Mean synergy score & $\approx$0.92 \\
Mean resistance score & $\approx$0.35 \\
Path coverage range & 84--100\% \\
Gold-standard recall & 61\% (14/23) \\
Novel combinations (no trials) & 5 \\
\bottomrule
\end{tabular}
\end{table}

\subsection*{Recurrent triple patterns}
The most frequent patterns across cancers (see \cref{fig:patterns}, \cref{fig:targets}):
\begin{itemize}
\item \textbf{CDK6 + KRAS + STAT3} (27 cancers): Pancreatic Adenocarcinoma, Acute Myeloid Leukemia, Renal Cell Carcinoma, Neuroblastoma, Bladder Urothelial Carcinoma, B-Cell ALL, T-Lymphoblastic Leukemia, Rhabdomyosarcoma, Embryonal Tumor, Myeloproliferative Neoplasms, Non-Hodgkin Lymphoma, Synovial Sarcoma, and others. Drugs: palbociclib + sotorasib + napabucasin.
\item \textbf{CDK4 + KRAS + STAT3} (27 cancers): Non-Small Cell Lung Cancer, Colorectal Adenocarcinoma, Invasive Breast Carcinoma, Esophagogastric Adenocarcinoma, Ewing Sarcoma, Hepatocellular Carcinoma, Osteosarcoma, Ocular Melanoma, Prostate Adenocarcinoma, and others.
\item \textbf{CDK2 + KRAS + STAT3} (6 cancers): Lung Neuroendocrine Tumor, Ovarian Epithelial Tumor, Endometrial Carcinoma, Cervical Squamous Cell Carcinoma, Merkel Cell Carcinoma, Uterine Sarcoma. Drugs: dinaciclib + sotorasib + napabucasin.
\item \textbf{BRAF + KRAS + STAT3} (3 cancers): Anaplastic Thyroid Cancer, Hepatoblastoma, Mucosal Melanoma of Vulva/Vagina.
\item \textbf{EGFR + KRAS + STAT3} (1 cancer): Ampullary Carcinoma.
\end{itemize}
The KRAS--STAT3 axis appears in \textbf{64 of 79} (81\%) top combinations, supporting the Liaki et al.\ PDAC rationale. CDK4/6 (cell cycle) + KRAS (MAPK) + STAT3 (JAK/STAT) provides three-pathway coverage.

\subsection*{Cancer-specific combinations}
Notable cancer-type-specific predictions:
\begin{itemize}
\item \textbf{Melanoma}: BRAF + CDK4 + MAP2K1 (vemurafenib + palbociclib + trametinib)---extends FDA-approved BRAF+MEK with CDK4; superset of gold standard.
\item \textbf{Head and Neck Squamous Cell Carcinoma}: CDK6 + EGFR + MET (palbociclib + erlotinib + capmatinib)---matches EGFR+MET clinical trial direction.
\item \textbf{Diffuse Glioma}: CDK4 + CDK6 + MET (palbociclib + capmatinib)---CDK4/6 in glioma with MET; superset of NCT03446147.
\item \textbf{Pleural Mesothelioma}: CDK6 + FGFR1 + MET (palbociclib + erdafitinib + capmatinib)---\textbf{lowest resistance score (0.00)}; no existing clinical trials.
\item \textbf{Chondrosarcoma, Extra Gonadal Germ Cell Tumor}: CDK2 + MET + STAT3---novel triple with no trials.
\item \textbf{Mixed Cervical Carcinoma}: CDK2 + EGFR + PIK3CA (dinaciclib + erlotinib + alpelisib)---three-pathway coverage (cell cycle, EGFR, PI3K).
\end{itemize}

\subsection*{Benchmark performance}
Against 23 gold-standard combinations (\cref{fig:benchmark}): \textbf{61\% recall} (14 superset matches), mean rank 1.0 when matched. Random baseline: 21\% $\pm$ 7\% (30 trials). Top-genes baseline: 39\%. ALIN significantly outperforms both baselines. Matched cases include: Melanoma (BRAF+MEK$\subseteq$BRAF+CDK4+MAP2K1), Invasive Breast (CDK4/6$\subseteq$CDK4+KRAS+STAT3), Colorectal (KRAS$\subseteq$CDK4+KRAS+STAT3), Ampullary (KRAS+STAT3$\subseteq$EGFR+KRAS+STAT3), HNSCC (EGFR+MET$\subseteq$CDK6+EGFR+MET), Diffuse Glioma (CDK4+CDK6$\subseteq$CDK4+CDK6+MET), AML (KRAS+CDK6$\subseteq$CDK6+KRAS+STAT3), Liposarcoma (CDK4+CDK6$\subseteq$CDK4+CDK6+STAT3). Unmatched: EGFR+MET in NSCLC, ALK in NSCLC, ERBB2 in breast, BRAF+EGFR in CRC, FLT3/MTOR/VEGFR in AML/RCC---may reflect subtype-specific or methodology gaps.

\subsection*{Novel combinations (no existing clinical trials)}
Five combinations have no matching clinical trials (see Supplementary Fig.\ S5), representing novel opportunities:
\begin{itemize}
\item \textbf{Pleural Mesothelioma}: CDK6+FGFR1+MET (palbociclib + erdafitinib + capmatinib)---resistance 0.00.
\item \textbf{Nerve Sheath Tumor}: CDK6+FGFR1+MET (same drugs).
\item \textbf{Chondrosarcoma}: CDK2+MET+STAT3 (dinaciclib + capmatinib + napabucasin).
\item \textbf{Extra Gonadal Germ Cell Tumor}: CDK2+MET+STAT3 (same drugs).
\item \textbf{Mixed Cervical Carcinoma}: CDK2+EGFR+PIK3CA (dinaciclib + erlotinib + alpelisib).
\end{itemize}
These are prioritized for experimental validation (see Future Work).

\subsection*{Validation and patient stratification}
API validation (PubMed, STRING) and PRISM drug sensitivity correlation were run on priority combinations. Literature validation scores (0.56 median) indicate moderate prior evidence; novel combinations cluster at lower scores. Patient stratification identifies mutation-based subgroups (KRAS G12C, BRAF V600E, EGFR L858R) and companion diagnostic gene panels; $\sim$75\% of patients are addressable by existing biomarkers. Clinical trial matching via ClinicalTrials.gov identifies NCT05564377 (KRAS+CDK4/6 in AML/solid tumors) as a candidate for adding STAT3 inhibitor (napabucasin). Synergy--resistance landscape (\cref{fig:synergy}) shows preferred quadrant: high synergy + low resistance.

\section*{Discussion}

\subsection*{Strengths}
ALIN Framework provides a reproducible, validated pipeline for triple combination discovery. Strengths include: (1) integrated DepMap+OmniPath; (2) refined viability path inference (co-essentiality clustering, NetworkX signaling paths); (3) cost function with OpenTargets toxicity; (4) multi-source validation (PubMed, STRING, ClinicalTrials.gov, PRISM); (5) benchmarking against 23 gold-standard combinations; (6) patient stratification for companion diagnostics; (7) identification of 5 novel combinations with no existing clinical trials. The pipeline is modular, documented, and ready for community use.

\subsection*{Limitations}
Limitations: (1) \textbf{In silico only}---experimental validation required; (2) \textbf{Gold-standard gaps}---9 of 23 combinations not recovered (EGFR+MET in NSCLC, ALK inhibitors, ERBB2 in breast, BRAF+EGFR in CRC, FLT3/MTOR/VEGFR in AML/RCC), which may reflect cancer subtype heterogeneity or methodology gaps; (3) \textbf{PDAC divergence}---Liaki et al.\ predicted SRC+FYN+STAT3; we predict CDK6+KRAS+STAT3 for Pancreatic Adenocarcinoma, suggesting different viability path inference or DepMap cell line composition; (4) \textbf{Cell line bias}---some cancers have few lines (e.g., Ovarian Germ Cell Tumor, n=1); (5) \textbf{Drug mapping}---we map genes to representative drugs; not all gene--drug pairs are equally validated.

\subsection*{Notable findings}
STAT3 appears in 81\% of top combinations, supporting the KRAS--STAT3 axis as a pan-cancer vulnerability. CDK4/6 inhibitors + KRAS + STAT3 is the dominant pan-cancer pattern. Melanoma correctly predicts BRAF+MEK+CDK4 (superset of FDA-approved BRAF+MEK). Pleural Mesothelioma CDK6+FGFR1+MET has the lowest resistance score (0.00) and no clinical trials---high priority for preclinical testing.

\section*{Future Work}

\subsection*{Phase 1: Clinical translation (0--6 months)}
\begin{itemize}
\item \textbf{NCT05564377}: Contact investigators about adding STAT3 inhibitor (napabucasin) to existing KRAS+CDK4/6 trials. Draft protocol amendment for ComboMATCH.
\item \textbf{ComboMATCH proposal}: Submit STAT3 arm for KRAS-mutant cancers (NCI ComboMATCH).
\item \textbf{Clinical trial registry}: Register novel combinations on ClinicalTrials.gov as investigator-initiated trials.
\end{itemize}

\subsection*{Phase 2: Experimental validation (6--18 months)}
Prioritize top 5 novel combinations for preclinical testing:
\begin{enumerate}
\item \textbf{Pleural Mesothelioma}: palbociclib + erdafitinib + capmatinib (CDK6+FGFR1+MET).
\item \textbf{Chondrosarcoma}: dinaciclib + capmatinib + napabucasin (CDK2+MET+STAT3).
\item \textbf{Nerve Sheath Tumor}: palbociclib + erdafitinib + capmatinib (CDK6+FGFR1+MET).
\item \textbf{Extra Gonadal Germ Cell Tumor}: dinaciclib + capmatinib + napabucasin (CDK2+MET+STAT3).
\item \textbf{Mixed Cervical Carcinoma}: dinaciclib + erlotinib + alpelisib (CDK2+EGFR+PIK3CA).
\end{enumerate}
Deliverables: in vitro synergy (Bliss/Loewe), in vivo xenograft efficacy, resistance emergence assays.

\subsection*{Phase 3: Methodological improvements (12--24 months)}
\begin{itemize}
\item \textbf{Gold standard}: Expand from 23 to 50+ combinations; include recent FDA approvals (e.g., KRAS G12D inhibitors).
\item \textbf{Data integration}: CellMiner/GDSC drug sensitivity; CCLE expression for expression-filtered essentiality; GCN portal tissue weighting.
\item \textbf{Toxicity}: FDA MedWatch ADR integration; OpenTargets safety liabilities refinement.
\item \textbf{Viability paths}: Refine co-essentiality with CCLE expression; add RNAi/CRISPR-AgO dependency comparison.
\item \textbf{Benchmark recall}: Target 70--80\% recall via improved path inference and cost weighting.
\end{itemize}

\subsection*{Phase 4: Software and community (ongoing)}
\begin{itemize}
\item \textbf{Code release}: GitHub repository with Zenodo DOI; \texttt{run\_full\_pipeline.sh} for full reproducibility.
\item \textbf{Supplementary tables}: Table S1 (full triple combinations), Table S2 (gold-standard benchmark details), Table S3 (novel combinations for lab testing).
\item \textbf{Preprint}: bioRxiv submission; target journals: Bioinformatics, BMC Bioinformatics, PLOS Computational Biology.
\end{itemize}

\subsection*{Data and code availability}
Data: DepMap (\url{https://depmap.org}), PRISM (\url{https://depmap.org/repurposing}), OmniPath (\url{https://omnipathdb.org}). Code: GitHub repository with \texttt{run\_full\_pipeline.sh} for full reproducibility. See DATA\_AVAILABILITY.md for file URLs and licenses.

\subsection*{Competing interests}
The author declares no competing interests.

\begin{acknowledgements}
We thank Liaki et al.\ for the foundational PDAC combination therapy methodology. DepMap, OmniPath, and PRISM consortia for open data.
\end{acknowledgements}

\section*{Bibliography}
\bibliographystyle{plain}
\bibliography{alin_refs}

%% Fallback if .sty or .bib missing: use article class and manual refs
% \begin{thebibliography}{9}
% \bibitem{Liaki2025} Liaki V, Barrambana S, et al. 2025. A targeted combination therapy achieves effective pancreatic cancer regression. bioRxiv doi:10.1101/2025.08.04.668325.
% \bibitem{DepMap} DepMap: depmap.org
% \bibitem{OmniPath} OmniPath: omnipathdb.org
% \end{thebibliography}

\onecolumn
\newpage

\section*{Figures}

\begin{figure}[htbp]
\centering
\includegraphics[width=0.9\linewidth]{figures/fig1_pipeline_schematic}
\caption{ALIN Framework pipeline overview. DepMap CRISPR and Model data feed cancer type mapping (OncoTree). Viability paths are inferred via co-essentiality clustering and signaling network paths. Minimal hitting sets (X-nodes) are optimized. Triple combinations are ranked by synergy, resistance, and path coverage. Validation: PubMed, STRING, ClinicalTrials.gov, PRISM.}
\label{fig:pipeline}
\end{figure}

\begin{figure}[htbp]
\centering
\includegraphics[width=0.7\linewidth]{figures/fig2_xnode_concept}
\caption{X-node concept: minimal hitting set over viability paths. Three overlapping paths (downstream, upstream, orthogonal) intersect at X-nodes (e.g., RAF1 + EGFR + STAT3 from Liaki et al.).}
\label{fig:xnode}
\end{figure}

\begin{figure}[htbp]
\centering
\includegraphics[width=0.7\linewidth]{figures/fig3_benchmark_comparison}
\caption{Benchmark: ALIN achieves 61\% recall vs.\ 21\% random and 39\% top-genes baseline. Error bars: standard deviation for random baseline.}
\label{fig:benchmark}
\end{figure}

\begin{figure}[htbp]
\centering
\includegraphics[width=0.85\linewidth]{figures/fig4_triple_patterns}
\caption{Top triple combination patterns across cancer types. CDK6+KRAS+STAT3 and CDK4+KRAS+STAT3 dominate.}
\label{fig:patterns}
\end{figure}

\begin{figure}[htbp]
\centering
\includegraphics[width=0.7\linewidth]{figures/fig5_target_frequency}
\caption{Most frequently predicted targets in top triple combinations. KRAS, STAT3, CDK6, CDK4 are top.}
\label{fig:targets}
\end{figure}

\begin{figure}[htbp]
\centering
\includegraphics[width=0.7\linewidth]{figures/fig6_synergy_resistance}
\caption{Synergy vs.\ resistance landscape. Quadrant: high synergy + low resistance = preferred. Novel combinations (no clinical trials) highlighted.}
\label{fig:synergy}
\end{figure}

\section*{Supplementary Materials}

\subsection*{Supplementary Figures}
\begin{itemize}
\item \textbf{Fig.\ S1}: Detailed pipeline flowchart with module names, file I/O, and optional steps.
\item \textbf{Fig.\ S2}: Co-essentiality clustering schematic---genes essential together $\rightarrow$ hierarchical clustering $\rightarrow$ pathway-like modules.
\item \textbf{Fig.\ S3}: Network path inference example---driver $\rightarrow$ intermediate $\rightarrow$ effector paths in OmniPath; path scoring by dependency.
\item \textbf{Fig.\ S4}: Benchmark rank distribution---when ALIN matches a gold-standard combination, mean rank = 1.0.
\item \textbf{Fig.\ S5}: Novel combinations (no clinical trials)---cancer type, targets, drugs, synergy/resistance scores.
\end{itemize}

\subsection*{Supplementary Tables}

\textbf{Table S1.} Representative triple combinations (subset of 79 cancer types). Full dataset available in supplementary files.
\small
\begin{longtable}{@{}p{4.2cm}p{3.2cm}p{4.2cm}ccc@{}}
\toprule
\textbf{Cancer Type} & \textbf{Targets} & \textbf{Drugs} & \textbf{Syn} & \textbf{Res} & \textbf{Cov} \\
\midrule
\endfirsthead

\multicolumn{6}{l}{\textit{Continued from previous page}} \\
\toprule
\textbf{Cancer Type} & \textbf{Targets} & \textbf{Drugs} & \textbf{Syn} & \textbf{Res} & \textbf{Cov} \\
\midrule
\endhead

\midrule
\multicolumn{6}{r}{\textit{Continued on next page}} \\
\endfoot

\bottomrule
\endlastfoot

Ovarian Germ Cell Tumor & KRAS+MET+STAT3 & sotorasib, capmatinib, napabucasin & 0.92 & 0.35 & 100\% \\
Anaplastic Thyroid Cancer & BRAF+KRAS+STAT3 & vemurafenib, sotorasib, napabucasin & 0.92 & 0.39 & 85\% \\
Ampullary Carcinoma & EGFR+KRAS+STAT3 & erlotinib, sotorasib, napabucasin & 0.92 & 0.39 & 100\% \\
Mucosal Melanoma (Vulva/Vagina) & BRAF+KRAS+STAT3 & vemurafenib, sotorasib, napabucasin & 0.92 & 0.39 & 100\% \\
Hepatoblastoma & BRAF+KRAS+STAT3 & vemurafenib, sotorasib, napabucasin & 0.92 & 0.39 & 86\% \\
Lung Neuroendocrine Tumor & CDK2+KRAS+STAT3 & dinaciclib, sotorasib, napabucasin & 0.92 & 0.44 & 97\% \\
Ovarian Epithelial Tumor & CDK2+KRAS+STAT3 & dinaciclib, sotorasib, napabucasin & 0.92 & 0.44 & 92\% \\
Endometrial Carcinoma & CDK2+KRAS+STAT3 & dinaciclib, sotorasib, napabucasin & 0.92 & 0.44 & 97\% \\
Cervical Squamous Cell Carcinoma & CDK2+KRAS+STAT3 & dinaciclib, sotorasib, napabucasin & 0.92 & 0.44 & 88\% \\
Merkel Cell Carcinoma & CDK2+KRAS+STAT3 & dinaciclib, sotorasib, napabucasin & 0.92 & 0.44 & 100\% \\
\multicolumn{6}{@{}l}{\textit{$\ldots$ 69 additional cancer types; full table in supplementary CSV}} \\
\end{longtable}
\normalsize

\vspace{1em}
\textbf{Table S2.} Gold-standard benchmark (23 FDA-approved and clinically validated combinations).
\small
\begin{longtable}{@{}p{3.2cm}p{2.8cm}lp{3.2cm}l@{}}
\toprule
\textbf{Cancer} & \textbf{Gold Targets} & \textbf{Evidence} & \textbf{Our Prediction} & \textbf{Match} \\
\midrule
\endfirsthead

\multicolumn{5}{l}{\textit{Continued from previous page}} \\
\toprule
\textbf{Cancer} & \textbf{Gold Targets} & \textbf{Evidence} & \textbf{Our Prediction} & \textbf{Match} \\
\midrule
\endhead

\midrule
\multicolumn{5}{r}{\textit{Continued on next page}} \\
\endfoot

\bottomrule
\endlastfoot

Melanoma & BRAF+MEK & FDA & BRAF+CDK4+MAP2K1 & Yes \\
Melanoma & BRAF+MEK2 & FDA & BRAF+CDK4+MAP2K1 & Yes \\
NSCLC & EGFR+MET & Breakthrough & --- & No \\
NSCLC & ALK & FDA & --- & No \\
NSCLC & KRAS & FDA & CDK4+KRAS+STAT3 & Yes \\
Breast (Invasive) & CDK4+CDK6 & FDA & CDK4+KRAS+STAT3 & Yes \\
Breast (Invasive) & KRAS+CDK4 & Trial & CDK4+KRAS+STAT3 & Yes \\
Breast (Invasive) & ERBB2 & FDA & --- & No \\
Colorectal & EGFR & FDA & --- & No \\
Colorectal & BRAF+EGFR & Trial & --- & No \\
Colorectal & KRAS & Trial & CDK4+KRAS+STAT3 & Yes \\
Pancreatic & KRAS+STAT3 & Preclin & CDK6+KRAS+STAT3 & Yes \\
Pancreatic & SRC+FYN+STAT3 & Preclin & CDK6+KRAS+STAT3 & No \\
Pancreatic & KRAS & FDA & CDK6+KRAS+STAT3 & Yes \\
AML & FLT3 & FDA & --- & No \\
AML & KRAS+CDK6 & Trial & CDK6+KRAS+STAT3 & Yes \\
RCC & MTOR & FDA & --- & No \\
RCC & VEGFR2+MTOR & Trial & --- & No \\
HNSCC & EGFR+MET & Trial & CDK6+EGFR+MET & Yes \\
Diffuse Glioma & CDK4+CDK6 & Trial & CDK4+CDK6+MET & Yes \\
Liposarcoma & CDK4+CDK6 & Trial & CDK4+CDK6+STAT3 & Yes \\
Ampullary & EGFR & Trial & EGFR+KRAS+STAT3 & Yes \\
HCC & EGFR+MET & Trial & EGFR+MET+STAT3 & Yes \\
\end{longtable}
\normalsize

\vspace{1em}
\textbf{Table S3.} Novel combinations (no existing clinical trials) for lab testing.
\begin{table}[htbp]
\centering
\small
\begin{tabular}{@{}p{3.5cm}p{2.8cm}p{4.5cm}p{4.2cm}@{}}
\toprule
\textbf{Cancer Type} & \textbf{Targets} & \textbf{Drugs} & \textbf{Rationale} \\
\midrule
Pleural Mesothelioma & CDK6+FGFR1+MET & palbociclib, erdafitinib, capmatinib & Lowest resistance (0.00); triple RTK/CDK coverage \\
Nerve Sheath Tumor & CDK6+FGFR1+MET & palbociclib, erdafitinib, capmatinib & Same as above; no trials \\
Chondrosarcoma & CDK2+MET+STAT3 & dinaciclib, capmatinib, napabucasin & Cell cycle + RTK + JAK/STAT \\
Extra Gonadal Germ Cell & CDK2+MET+STAT3 & dinaciclib, capmatinib, napabucasin & Same triple; rare cancer \\
Mixed Cervical Carcinoma & CDK2+EGFR+PIK3CA & dinaciclib, erlotinib, alpelisib & Three-pathway coverage \\
\bottomrule
\end{tabular}
\end{table}
\normalsize

\end{document}
